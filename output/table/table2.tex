\begin{table}[htbp]
\centering
\caption{Industry-Level Summary Statistics}
\label{tab:industry_stats}
\begin{tabular}{*{5}{l}}
\toprule
Industry (TEAOR08) & \shortstack{Firm-year\\obs.} & \shortstack{Distinct\\firms} & \shortstack{Distinct\\managers} & Status \\
\midrule
A: Agriculture, Forestry, Fishing &      321,548 &       26,912 &       27,553 & Included \\
C: Manufacturing &    1,023,232 &       93,305 &      100,699 & Included \\
G,H: Wholesale, Retail, Transportation &    2,901,620 &      312,287 &      320,322 & Included \\
J,M: Telecom, Business Services &    1,983,170 &      194,172 &      195,321 & Included \\
F: Construction &      972,444 &      120,850 &      109,948 & Included \\
Other: Nontradable Services &    2,795,689 &      290,914 &      276,494 & Included \\
\midrule
B: Mining, Quarrying (Excluded) &       13,427 &        1,190 &            8 & Excluded \\
K,L: Finance, Insurance, Real Estate (Excluded) &      202,990 &       23,542 &          177 & Excluded \\
\bottomrule
\end{tabular}
\begin{minipage}{\textwidth}
\footnotesize
\textit{Notes:} This table presents industry-level summary statistics using the TEAOR08 classification system. Column (1) shows the industry name and corresponding TEAOR08 sector codes. Column (2) shows the total number of firm-year observations in the balance sheet data (1992-2022). Column (3) shows the number of distinct firms with balance sheet data. Column (4) shows the number of distinct managers (CEOs) from the firm registry data. Column (5) indicates whether the industry is included in or excluded from the main analysis. Mining (sector B) and Finance/Insurance/Real Estate (sectors K,L) are excluded from the main analysis due to different production function characteristics. The TEAOR08 classification follows the Hungarian adaptation of the NACE Rev. 2 system. Source: Hungarian administrative data combining firm balance sheets and CEO registry.
\end{minipage}
\end{table}
