\begin{table}[htbp]
\centering
\caption{Industry Breakdown}
\label{tab:industry_stats}
\begin{tabular}{l*{5}{r}}
\toprule
Industry (NACE) & \shortstack{Obs.} & \shortstack{Firms} & \shortstack{CEOs} & \shortstack{Surplus\\share (\%)} \\
\midrule
Agriculture, Forestry, Fishing (A) &      322,298 &       26,972 &       55,545 & -125.5 \\
Manufacturing (C) &    1,026,916 &       93,549 &      179,204 & -63.8 \\
Wholesale, Retail, Transportation (G,H) &    2,906,789 &      312,644 &      550,160 & -58.4 \\
Telecom, Business Services (J,M) &    1,978,917 &      193,905 &      345,313 & -16.6 \\
Construction (F) &      972,173 &      120,839 &      183,138 & -76.6 \\
Nontradable Services (Other) &    2,791,099 &      290,554 &      527,671 & -83.2 \\
Mining, Quarrying (B)* &       13,490 &        1,194 &        2,923 & -94.1 \\
Finance, Insurance, Real Estate (K,L)* &      202,868 &       23,515 &       48,156 & -19.0 \\
\bottomrule
\end{tabular}
\begin{minipage}{\textwidth}
\footnotesize
\textit{Notes:} This table presents industry-level summary statistics using the TEAOR08 classification system. Column (1) shows the industry name and corresponding NACE sector codes. Column (2) shows the total number of firm-year observations in the balance sheet data (1992-2022). Column (3) shows the number of distinct firms with balance sheet data. Column (4) shows the number of distinct managers (CEOs) from the firm registry data. Column (5) shows the average EBITDA as a percentage of revenue. Mining (sector B) and Finance/Insurance/Real Estate (sectors K,L) are excluded from the main analysis due to different production function characteristics. The NACE classification follows the Hungarian adaptation of the NACE Rev. 2 system. \end{minipage}
\end{table}
