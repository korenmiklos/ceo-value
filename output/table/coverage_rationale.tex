\begin{table}[htbp]\centering
\caption{Coverage Rationale: Why Revenue is Primary}
\begin{tabular}{lcc}
\toprule
Criterion & Revenue & EBITDA \\
\midrule
Valid observations &    8,116,380 &    6,047,316 \\
Coverage rate & 84\% & 62.6\% \\
Always positive & Yes & No \\
Required by law & Yes & No \\
Available across all years & Yes & Partial \\
Internationally comparable & Yes & Yes \\
\bottomrule
\end{tabular}
\begin{minipage}{\textwidth}
\footnotesize
\textit{Notes:} This table compares revenue and EBITDA as primary outcome measures. 
Revenue has    3,613,777 more valid observations than EBITDA due to 
negative EBITDA values that cannot be log-transformed. Revenue reporting is 
mandatory for all firms, ensuring comprehensive coverage. EBITDA, while economically 
meaningful, can be negative during losses, creating sample selection issues in 
logarithmic specifications. Coverage rates are calculated from the analysis sample.
\end{minipage}
\end{table}
