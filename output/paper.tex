\documentclass[11pt,a4paper]{article}
\usepackage[utf8]{inputenc}
\usepackage[T1]{fontenc}
\usepackage{amsmath,amsfonts,amssymb}
\usepackage{apacite}
\usepackage{natbib}
\usepackage{graphicx}
\usepackage{booktabs}
\usepackage{url}
\usepackage{hyperref}
\usepackage[margin=2.5cm]{geometry}
\usepackage{setspace}
\onehalfspacing

\title{Estimating the Value of CEOs in Privately Held Businesses}

\author{Miklós Koren\thanks{HUN-REN Centre for Economic and Regional Studies, Institute of Economics and Corvinus University of Budapest. Email: koren.miklos@krtk.hun-ren.hu. Project no. 144193 has been implemented with the support provided by the Ministry of Culture and Innovation of Hungary from the National Research, Development and Innovation Fund, financed under the KKP\_22 funding scheme. This project was funded by the European Research Council (ERC Advanced Grant agreement number 101097789). The views expressed in this research are those of the authors and do not necessarily reflect the official view of the European Union or the European Research Council.} \\
        Krisztina Orbán\thanks{HUN-REN Centre for Economic and Regional Studies, Institute of Economics.} \\
        Bálint Szilágyi\thanks{HUN-REN Centre for Economic and Regional Studies, Institute of Economics.} \\
        Álmos Telegdy\thanks{HUN-REN Centre for Economic and Regional Studies, Institute of Economics.} \\
        András Vereckei\thanks{HUN-REN Centre for Economic and Regional Studies, Institute of Economics.}}

\date{\today}

\begin{document}

\maketitle

\begin{abstract}
Most evidence on CEO value comes from publicly traded firms in developed markets, limiting its relevance for developing economies where privately held businesses predominate. We develop a framework to estimate CEO value in private firms using standard financial statement data and administrative registers, overcoming traditional measurement challenges. Our theoretical model builds on Lucas (1978) and assumes managers differ in skills that directly affect total factor productivity, with firms retaining economic rents due to decreasing returns to scale. Applied to comprehensive Hungarian administrative data spanning 1992-2022, our approach tracks CEO changes and measures their impact on firm surplus generation. Results from this analysis will quantify the economic value CEOs create in private business contexts. Understanding managerial contributions in private firms has broad implications for entrepreneurship policy and economic development in emerging markets.
\end{abstract}

\textbf{Keywords:} CEO value, private firms, productivity

\textbf{JEL Classification:} [To be added]

\newpage

\section{Introduction}

Managers play a crucial role in determining firm performance, as documented across various institutional settings \citep{Bertrand2003-io, Fisman2014-pw, bandiera2020ceo, bennedsen2020ceos}. Most existing studies in this literature focus on publicly listed firms in developed markets such as the United States. Evidence from developing countries suggests that management practices are equally important for firm performance in these contexts \citep{Bloom2014-ux}. However, the relevance of existing evidence on CEO value remains limited for developing markets for two key reasons. First, in developing markets, most firms are privately held and do not have publicly traded shares or readily available information about executive compensation of the sort surveyed in \citet{frydman2010executive}. This creates a fundamental measurement challenge for researchers. Second, in private businesses, owners typically retain direct control over the firm, which may generate different incentive structures and performance outcomes compared to publicly listed firms with dispersed ownership.

In this paper, we develop a framework to estimate the value of CEOs in privately held businesses using standard financial statement data and administrative registers, data sources that are commonly collected by governments in developing countries. This approach enables the measurement of CEO value in settings where traditional methods based on stock market valuations or executive compensation data are not feasible.

Our theoretical model builds on \citet{Lucas1978-rp} and assumes that managers differ in their skills, which directly affect the total factor productivity of the firm. Following \citet{AtkesonKehoe2005JPE, McGrattan2012RED}, we assume that firms retain economic rents due to decreasing returns to scale in the presence of manager skills, organizational capital, and intangible assets. This framework allows us to identify the marginal contribution of CEO skills to firm surplus while controlling for other sources of firm heterogeneity.

We apply this framework to comprehensive administrative data from Hungary covering the period 1992-2022. The dataset enables us to track CEO changes across firms and measure their impact on firm performance, specifically on the economic surplus generated by the firm. This empirical strategy allows us to identify CEO value through variation in managerial appointments while controlling for firm-specific and time-varying factors that might otherwise confound the analysis.


\section{Modeling Framework}
Firms produce output using a Cobb-Douglas production function that incorporates both fixed and variable inputs. Owing to the presence of fixed inputs, technology exhibits decreasing returns to scale. This will pin down the scale of the firm even when markets are perfectly competitive and the firm is a price taker in both input and output markets \citep{AtkesonKehoe2005JPE,McGrattan2012RED}.\footnote{Alternatively, we could assume that firms face downward sloping residual demand curves, which would make the \emph{revenue production function} decreasing returns to scale. As long as residual demand is isoelastic, the analytical derivation of the model remains unchanged. The only difference is that the parameters have a different interpretation: the revenue elasticity of an input is the product of the input's share in revenue and $1-1/\sigma$, where $\sigma$ is the elasticity of residual demand \citep{DeLoecker2011Econometrica}.}

The production function for firm $i$ with manager $m$ at time $t$ is:
\begin{equation}
Q_{imt} = \Omega_{it}A_i Z_{m}  K_{it}^\alpha L_{imt}^{\beta} M_{imt}^{\gamma}
\end{equation}
where $\Omega_{it}$ is residual total factor productivity, $A_i$ represents time-invariant organizational capital and immaterial assets (location, brand value), $Z_m$ captures manager skill, $K_{it}$ is physical capital, $L_{imt}$ is labor input, $M_{imt}$ is intermediate input usage. The parameters $\alpha$, $\beta$ and $\gamma$ represent the elasticities with respect to physical capital, labor and material inputs, respectively. We denote $\chi := 1 - \beta - \gamma$. Conditional on productivity, organizational capital and manager skill, the production function exhibits decreasing returns to scale, $\alpha + \beta + \gamma < 1$. In a traditional production function with only capital, labor and material as inputs, $\Omega$, $A$ and $Z$ would all be lumped together as \emph{total factor productivity}.

We assume managers optimize variable inputs $L_{imt}$ and $M_{imt}$ while taking fixed inputs $A_{i}$ and $Z_m$ and physical capital $K_{it}$ as given. In private businesses, owners typically have direct control over fixed inputs, including large-scale investments in organizational and physical capital \citep{Navaretti2010EFIGE}. Managers, on the other hand, are responsible for day-to-day operations and variable input choices.

Output is sold at sector-specific price $P_{st}$, making the revenue of the firm $R_{imst} = P_{st}Q_{imt}$. The firm faces a wage rate $W_{st}$ for labor input, price $\varrho_{st}$ for intermediate inputs. After straightforward algebra solving for the optimal labor and intermediate input choices, the firm's revenue can be expressed as:
\begin{equation}\label{eq:revenue}
R_{imst} = (P_{st}\Omega_{it}A_i Z_m)^{1/\chi}
K_{it}^{\alpha/\chi}
W_{st}^{-\beta/\chi}
\varrho_{st}^{-\gamma/\chi}
(1-\chi)^{(1-\chi)/\chi}.
\end{equation}
Revenue is increasing in fixed inputs $A_i$ and $Z_m$, physical capital $K_{it}$, and decreasing in the wage rate $W_{st}$ and material input price $\varrho_{st}$. Higher prices $P_{st}$ and productivity $\Omega_{it}$ also increase revenue. Note that because $\chi<1$, the elasticity of revenue with respect to fixed inputs is greater than the elasticity in the production function, i.e. $\alpha/\chi > \alpha$. This is because the firm can leverage its fixed inputs to increase revenue more than proportionally by hiring more variable inputs.

As is usual under Cobb-Douglas production functions, the share of revenue accruing to each input is constant over time and across firms, equal to their elasticity in the production function. We define the rent accruing to fixed factors (including physical capital) 
\begin{equation}\label{eq:rent}
S_{imst} = R_{imst} - W_{st}L_{imt} - \varrho_{st}M_{imt} = \chi R_{imst}.
\end{equation}
Taking logarithms of equations \eqref{eq:revenue} and \eqref{eq:rent}, we can express the log surplus as:      
\begin{equation}\label{eq:log_surplus}
s_{imst} = C+\frac\alpha\chi k_{it} + \frac1\chi {z}_{m} + \frac1\chi p_{st} + \frac1\chi{\omega}_{it}+\frac1\chi a_i 
- \frac\beta\chi w_{st} - \frac\gamma\chi \rho_{st},
\end{equation}
where $C$ is a constant only depending on fixed parameters, $k_{it} = \ln K_{it}$, ${z}_{m} = \ln Z_m$ , $ p_{st} = \ln P_{st}$, ${\omega}_{it} = \ln\Omega_{it}$, $a_i = \ln A_i$, and $w_{st} = \ln W_{st}$, $\rho_{st} = \ln \varrho_{st}$. 

Equation \eqref{eq:log_surplus} shows how surplus depends on manager skills, holding fixed the inputs chosen by the owner and the input and output prices prevailing in the sector. Taking two managers $m$ and $m'$ with skills ${z}_m$ and ${z}_{m'}$ at the same firm, the change in surplus attributable to the new manager is:
\begin{equation}\label{eq:manager_change}
s_{im'st} - s_{imst} = \frac1\chi({z}_{m'} - {z}_{m}).
\end{equation}
The \emph{value} of the new manager to the owners of the firm is the change in surplus. This value is proportional to the difference in manager skills, scaled by the inverse of the elasticity of revenue with respect to fixed inputs $\chi$. In what follows, we aim to measure this value by estimating the change in surplus following a manager change.

\paragraph{Estimable equation.} In absence of observing organization capital and input prices, we can substitute these out with fixed effects, leading to the following estimable equation:
\begin{equation}\label{eq:estimation}
s_{imst} = \frac\alpha\chi k_{it}  + \frac1\chi\tilde{z}_m + \lambda_i + \mu_{st} + \tilde \omega_{it}
\end{equation}
where $\lambda_i = a_i/\chi$ is a firm fixed effect capturing time-invariant organizational capital, $\mu_{st} = C + p_{st}/\chi - \beta w_{st}/\chi - \gamma\rho_{st}/\chi$ is an industry-time fixed effect capturing sector-specific prices and wages, and $\tilde\omega_{it} = \omega_{it}/\chi$ is a rescaled time-varying firm productivity shock. 

Assuming that residual productivity $\tilde\omega_{it}$ is uncorrelated with manager skills and physical capital, we can estimate the model using ordinary least squares with fixed effects (OLSFE). Note that we do \emph{not} assume that manager skills are uncorrelated with physical capital, organizational capital or sectoral prices. It may well be the case that better firms with good price conditions hire better managers and invest more. 

Given our estimated parameters and fixed effects, we can recover manager skills as:
\begin{equation}\label{eq:estimated}
\hat\chi s_{imst} -  \hat\alpha k_{it}  -\hat\chi \lambda_i -\hat\chi \mu_{st} := \tilde s_{imst} = \hat z_m + \hat\omega_{it}. 
\end{equation}
We remove the contribution of physical capital, firm and industry-year fixed effects from log surplus to obtain a \emph{residualized surplus} $\tilde s_{imst}$. Because $\omega_{it}$ is assumed to be mean zero independent of $m$, we can estimate $\hat z_m$ as the average of $\tilde s_{imst}$ across all observations for manager $m$. This gives us a consistent estimate of manager skill, $\hat z_m = \frac1{N_m}\sum_{i,t} \tilde s_{imst}$, where $N_m$ is the number of observations for manager $m$.\footnote{This is equivalent to including a manager fixed effect in the regression, similar in spirit to \citet{Abowd1999Econometrica} and \citet{Card2018JoLE}. This notation emphasizes that manager effects estimated from fewer observations are noisier.}

\section{Data and Measurement}
\paragraph{Main data sources.} Our analysis uses comprehensive administrative data on Hungarian firms during 1992-2022, created by merging balance sheet and financial statement data with firm registry information. The balance sheet data come from \citet{merleg2024} and contains financial information for essentially all Hungarian firms required to file annual reports. The firm registry data come from \citet{cegjegyzek2024} and includes information on firm registration, ownership structure, and director appointments. Both datasets are distributed by HUN-REN KRTK and originally published by Opten Zrt.\footnote{The data cannot be publicly shared due to privacy and licensing restrictions. The replication package available at https://github.com/korenmiklos/ceo-value describes how to get access to the data.}

The balance sheet data include all firms required to file financial statements with Hungarian authorities, covering essentially the entire formal business sector except for the smallest corporations not engaged in double-entry bookkeeping and individual entrepreneurs. The dataset contains detailed financial information including sales revenue, export revenue, employment, tangible and intangible assets, raw material and intermediate input costs, personnel expenses, and ownership indicators for state and foreign control.

Registry information is collected by the Hungarian Corporate Court, which maintains legally mandated public records on firms \citep{cegtv}. These records include information on company representatives---individuals authorized to act on behalf of the firm in legal and business matters. Representatives may include CEOs and other executives, but also lower-level employees with signatory rights. We exclude the rare instances where the representative is a legal entity. The dataset is structured as a temporal database: each entry has an effective date interval and reflects the state of representation at a given time. Updates occur not only when positions change but also when personal identifiers (e.g., address) are modified or when reporting standards evolve. Start and end dates are often missing, and prior to 2010, the data does not contain unique numerical identifiers for individuals.

We resolve individual identities by linking records based on name, birth date, mother's name, and home address, creating a unique identifier for each person. This entity resolution step enables tracking of representatives over time and across firms. To construct an annual panel of top managers, we infer the period of service for each representative using available date bounds and sequential information. A representative is considered active in a given year if their tenure includes June 21 of each year.

Because job titles are not standardized, identifying the CEO requires heuristic rules. When an explicit title such as \emph{managing director} is available, we classify the individual accordingly. For firms lacking such labels, we assume that all representatives are CEOs if the number of representatives is three or fewer. If there are more than three and one of them was previously identified as a CEO, we assign the CEO role based on continuity. This approach allows us to systematically identify the firm's top executive across years.

\paragraph{Sample construction.} We construct our analytical sample through several filtering steps. We restrict our analysis to the period 1992 to 2022 to focus on the post-transition Hungarian economy. This removes 136,141 observations from years prior to 1992, when the economic and institutional environment was fundamentally different. Our sample contains 10,214,120 firm-year observations spanning 31 years. Table \ref{tab:sample} shows the temporal distribution of observations in our final sample. The sample exhibits steady growth from 98,780 observations in 1992 to 454,106 in 2022. This expansion reflects the growth of entrepreneurship in Hungary following the transition to a market economy.

\begin{table}[htbp]
\centering
\caption{Sample Distribution by Year}
\label{tab:sample}
\begin{tabular}{rr|rr|rr}
\toprule
Year & Observations & Year & Observations & Year & Observations \\
\midrule
1992 & 98,780 & 2002 & 301,278 & 2012 & 397,131 \\
1993 & 122,677 & 2003 & 305,947 & 2013 & 437,692 \\
1994 & 153,639 & 2004 & 319,750 & 2014 & 427,494 \\
1995 & 171,759 & 2005 & 326,905 & 2015 & 433,371 \\
1996 & 198,558 & 2006 & 334,498 & 2016 & 431,041 \\
1997 & 219,751 & 2007 & 345,134 & 2017 & 424,184 \\
1998 & 246,660 & 2008 & 362,920 & 2018 & 425,601 \\
1999 & 256,992 & 2009 & 370,788 & 2019 & 419,883 \\
2000 & 280,386 & 2010 & 384,570 & 2020 & 424,501 \\
2001 & 302,894 & 2011 & 402,636 & 2021 & 432,594 \\
 &  &  &  & 2022 & 454,106 \\
\midrule
\multicolumn{6}{c}{Total: 10,214,120} \\
\bottomrule
\end{tabular}
\footnotesize
Notes: Sample distribution after applying time period restrictions (1992-2022) and data quality filters.
\end{table}

\textbf{CEO panel construction.} We construct a panel of chief executive officers from the firm registry data, restricting the sample to the same 1992-2022 time period. The initial CEO panel contains information on 996,387 observations that are excluded due to the time restriction. The final CEO panel includes variables identifying the firm (frame\_id\_numeric), person (person\_id), year, as well as CEO characteristics including gender (male), birth year, manager category, and ownership status.

The CEO data reveals substantial variation in the number of CEOs per firm-year. Among the 12,726,597 firm-year observations with CEO information, the vast majority (82.24\%) have a single CEO. However, 15.32\% of firm-years have two CEOs, 1.98\% have three CEOs, and small fractions have even larger numbers of CEOs, with some firms reporting up to 52 CEOs in a single year. This distribution reflects the complexity of executive structures in Hungarian firms, including cases where firms may have multiple managing directors or where CEO transitions occur within a year.

\textbf{Sample merging and match rates.} We merge the CEO panel with the balance sheet data using firm identifiers and year. The merge process reveals important patterns in data availability across sources. Of the 15,980,738 total observations from both datasets, 11,886,636 observations (74.4\%) successfully match between CEO and balance sheet data. The remaining observations consist of 3,507,466 CEO observations without corresponding balance sheet data and 586,636 balance sheet observations without CEO information.

At the firm level, the match rates are more favorable. Among the 1,200,145 unique firms in our combined dataset, 942,684 firms (78.55\%) have information in both datasets. The remaining firms are split between 238,852 firms (19.90\%) that appear only in the CEO registry and 18,609 firms (1.55\%) that appear only in the balance sheet data. This pattern suggests that CEO information is available for most active firms but may be missing for very small firms or those with simplified reporting requirements.

\textbf{Industry composition.} We classify firms into broad industry sectors using the TEAOR08 classification system. The final analytical sample of 8,872,039 firm-year observations spans diverse industries, with notable concentration in service sectors. Wholesale, retail, and transportation activities account for the largest share at 3,430,342 observations (28.86\%). Nontradable services represent 3,176,339 observations (26.72\%), while telecom and business services contribute 2,249,271 observations (18.92\%). Manufacturing firms account for 1,254,792 observations (10.56\%), construction for 1,100,022 observations (9.25\%), and agriculture for 411,226 observations (3.46\%). Mining represents the smallest sector with 16,926 observations (0.14\%). Finance sector firms are separately identified with 247,718 observations (2.08\%).

\textbf{CEO turnover and tenure patterns.} The data reveals substantial heterogeneity in CEO turnover across firms. We construct CEO spell variables to track the sequence of different CEO appointments within each firm. Among firm-year observations, 66.72\% represent the first CEO spell, meaning these are either firms with their original CEO or the first year of data for that CEO. Second CEO spells account for 22.90\% of observations, while 6.88\% represent third spells. The distribution has a long tail, with some firms experiencing up to 25 different CEO spells during the observation period.

At the firm level, 62.97\% of the 1,012,113 firms in our sample experience only one CEO spell during the observation window. However, 24.07\% of firms have exactly two CEO spells, indicating at least one CEO change. The remaining 12.96\% of firms experience multiple CEO changes, with some firms having up to 25 CEO transitions. This pattern suggests that while many firms maintain stable CEO leadership, a substantial minority experience frequent executive turnover.

\textbf{Sample restrictions and final dataset.} We apply several filters to focus on firms most suitable for productivity analysis. First, we exclude firms that ever have more than two CEOs in a single year, removing 1,519,524 observations. This filter eliminates firms with potentially complex or unstable governance structures that may confound productivity estimates. Second, we drop firms with more than six CEO spells over the observation period, removing an additional 45,216 observations to focus on firms with more stable executive structures.

We also exclude certain industries that may have different production functions or regulatory environments. Agriculture, mining, construction, and finance sectors are dropped, removing 1,494,057 observations. The final restriction removes firms from these sectors because agricultural production functions differ fundamentally from other sectors, mining operations face unique resource constraints, construction has project-based rather than continuous production, and financial services operate under distinct regulatory frameworks that affect standard productivity measures.

\textbf{Variable construction.} 
\paragraph{Measurement of Model Variables.} We measure the key variables from the theoretical framework as follows:

\textit{Physical capital} ($K_{it}$): Tangible assets from balance sheet data, including machinery, equipment, and buildings, measured in logarithmic form as $k_{it} = \ln K_{it}$.

\textit{Surplus} ($S_{imst}$): EBITDA (Earnings Before Interest, Taxes, Depreciation, and Amortization), calculated as sales revenue minus personnel expenses minus material costs, measured in logarithmic form as $s_{imst} = \ln S_{imst}$.

\textit{Manager skill} ($Z_m$): CEO fixed effects $\tilde{z}_m$ estimated from the regression in equation (5), capturing time-invariant managerial ability.

\textit{Labor input} ($L_{imt}$): Employment measured as the number of employees, transformed to logarithmic form as $l_{imt} = \ln L_{imt}$.

\textit{Manager compensation} ($W_{imst}$): CEO wages including base salary and bonuses from administrative records (not yet available in current analysis).

\textit{Organizational capital} ($A_i$): Time-invariant firm characteristics including location, brand value, and market position, captured by firm fixed effects $\lambda_i$ and not directly observed.

\textit{Sector-time variation}: Industry-specific prices and wages controlled through industry-time fixed effects $\mu_{st}$ using TEAOR08 sector classifications.



Missing values in financial variables are systematically recoded to zero, following standard practice in administrative data analysis where missing values typically indicate zero rather than unknown values. The extent of missing data varies considerably across variables, reflecting different reporting requirements and business activities. Export data has the highest rate of missing values, with 5,456,815 observations recoded, reflecting that many firms do not engage in export activities. Employment data required recoding for 1,138,791 observations, while sales revenue had relatively few missing values with only 486,197 observations recoded.

For employment, we make an additional adjustment by setting values below one to equal one. This transformation affects 3,655,899 observations and acknowledges that active firms filing administrative reports must have positive employment. Zero or negative employment values likely reflect administrative reporting inconsistencies rather than true zero employment.

We also address issues with wage bill and personnel expense variables, where 3,931,270 and 1,117,283 observations respectively are recoded from missing to zero. For asset variables, tangible assets required recoding for 1,014,331 observations while intangible assets had 4,299,589 missing values recoded, reflecting that many firms do not report significant intangible assets.

We construct several derived variables for the analysis. EBITDA is calculated as sales minus personnel expenses minus materials. Log transformations are applied to key variables including sales (lnR), EBITDA (lnEBITDA), employment (lnL), and tangible assets (lnK). CEO tenure is measured as years since first appointment, while CEO age and firm age are calculated from birth year and founding year respectively. We also create indicator variables for expatriate CEOs (those with missing gender information, suggesting non-Hungarian names) and ownership status.

The final analytical sample contains 8,872,039 firm-year observations representing 960,464 unique firms over the 1992-2022 period. This sample focuses on manufacturing, wholesale/retail/transportation, telecom/business services, and other nontradable services sectors, with firms having relatively stable CEO structures suitable for productivity analysis.

\begin{table}[htbp]
\centering
\caption{Industry Composition of Final Sample}
\label{tab:industry}
\begin{tabular}{lrr}
\toprule
Industry Sector & Observations & Percent \\
\midrule
Wholesale, Retail, Transportation & 3,430,342 & 28.86 \\
Nontradable Services & 3,176,339 & 26.72 \\
Telecom and Business Services & 2,249,271 & 18.92 \\
Manufacturing & 1,254,792 & 10.56 \\
Construction\textsuperscript{*} & 1,100,022 & 9.25 \\
Agriculture\textsuperscript{*} & 411,226 & 3.46 \\
Finance\textsuperscript{*} & 247,718 & 2.08 \\
Mining\textsuperscript{*} & 16,926 & 0.14 \\
\midrule
Total (before restrictions) & 11,886,636 & 100.00 \\
Final analytical sample & 8,872,039 & -- \\
\bottomrule
\end{tabular}
\footnotesize
Notes: \textsuperscript{*}Industries excluded from final analytical sample. Industry classification based on TEAOR08 system.
\end{table}

\begin{table}[htbp]
\centering
\caption{CEO Structure and Turnover Patterns}
\label{tab:ceo_structure}
\begin{tabular}{lrr}
\toprule
\multicolumn{3}{c}{Panel A: Number of CEOs per Firm-Year} \\
\midrule
Number of CEOs & Observations & Percent \\
\midrule
1 & 10,466,412 & 82.24 \\
2 & 1,949,370 & 15.32 \\
3 & 251,882 & 1.98 \\
4+ & 58,933 & 0.46 \\
\midrule
Total & 12,726,597 & 100.00 \\
\\[0.5em]
\multicolumn{3}{c}{Panel B: CEO Spells per Firm-Year} \\
\midrule
CEO Spell & Observations & Percent \\
\midrule
1 (First CEO) & 6,423,429 & 66.72 \\
2 & 2,204,806 & 22.90 \\
3 & 662,846 & 6.88 \\
4 & 205,665 & 2.14 \\
5+ & 130,738 & 1.36 \\
\midrule
Total & 9,627,484 & 100.00 \\
\\[0.5em]
\multicolumn{3}{c}{Panel C: Maximum CEO Spells per Firm} \\
\midrule
Max CEO Spells & Firms & Percent \\
\midrule
1 & 637,287 & 62.97 \\
2 & 243,609 & 24.07 \\
3 & 84,184 & 8.32 \\
4-6 & 42,788 & 4.23 \\
7+ & 4,245 & 0.42 \\
\midrule
Total & 1,012,113 & 100.00 \\
\bottomrule
\end{tabular}
\footnotesize
Notes: Panel A shows distribution of concurrent CEOs per firm-year. Panel B shows CEO spell distribution among successfully matched firm-years. Panel C shows maximum number of CEO changes per firm over entire observation period.
\end{table}

\section{Methodology}

[Methodology to be written]

\section{Results}

\section{Conclusion}

[Conclusion to be written]

\bibliographystyle{apacite}
\bibliography{references}

\appendix
\section{Robustness Checks}

\begin{table}[htbp]\centering
\def\sym#1{\ifmmode^{#1}\else\(^{#1}\)\fi}
\caption{The revenue function in various samples}
\begin{tabular}{l*{5}{c}}
\hline\hline
                    &\multicolumn{1}{c}{(1)}&\multicolumn{1}{c}{(2)}&\multicolumn{1}{c}{(3)}&\multicolumn{1}{c}{(4)}&\multicolumn{1}{c}{(5)}\\
                    &\multicolumn{1}{c}{Full}&\multicolumn{1}{c}{sample}&\multicolumn{1}{c}{First CEO spell}&\multicolumn{1}{c}{Single CEO spell}&\multicolumn{1}{c}{Multiple CEO spells}\\
\hline
Tangible and intangible assets (log)&       0.254\sym{***}&       0.255\sym{***}&       0.256\sym{***}&       0.249\sym{***}&       0.275\sym{***}\\
                    &     (0.001)         &     (0.001)         &     (0.001)         &     (0.001)         &     (0.002)         \\
[1em]
Intangible assets share&      -0.028\sym{***}&      -0.027\sym{***}&      -0.038\sym{***}&      -0.016\sym{*}  &      -0.040\sym{***}\\
                    &     (0.007)         &     (0.009)         &     (0.011)         &     (0.010)         &     (0.014)         \\
[1em]
Foreign owned       &       0.012         &       0.012         &      -0.004         &       0.018\sym{*}  &       0.022         \\
                    &     (0.008)         &     (0.011)         &     (0.014)         &     (0.010)         &     (0.013)         \\
\hline
Observations        &     6634335         &     4404163         &     3073377         &     3560899         &     1797728         \\
\hline\hline
\multicolumn{6}{l}{\footnotesize Controls: firm-CEO-spell fixed effects; industry-year fixed effects.}\\
\end{tabular}
\end{table}


\begin{table}[htbp]\centering
\def\sym#1{\ifmmode^{#1}\else\(^{#1}\)\fi}
\caption{The revenue function by sector}
\begin{tabular}{l*{4}{c}}
\hline\hline
                    &\multicolumn{1}{c}{(1)}&\multicolumn{1}{c}{(2)}&\multicolumn{1}{c}{(3)}&\multicolumn{1}{c}{(4)}\\
                    &\multicolumn{1}{c}{Manufacturing}&\multicolumn{1}{c}{Wholesale, Retail, Transportation}&\multicolumn{1}{c}{Telecom and Business Services}&\multicolumn{1}{c}{Nontradable services}\\
\hline
Tangible and intangible assets (log)&       0.302\sym{***}&       0.262\sym{***}&       0.242\sym{***}&       0.212\sym{***}\\
                    &     (0.003)         &     (0.002)         &     (0.002)         &     (0.002)         \\
[1em]
Intangible assets share&       0.011         &      -0.010         &      -0.065\sym{***}&      -0.027\sym{*}  \\
                    &     (0.025)         &     (0.014)         &     (0.012)         &     (0.015)         \\
[1em]
Foreign owned       &       0.046\sym{*}  &       0.011         &       0.088\sym{***}&      -0.014         \\
                    &     (0.023)         &     (0.015)         &     (0.022)         &     (0.014)         \\
[1em]
State owned         &       0.083\sym{*}  &      -0.029         &      -0.133\sym{*}  &       0.006         \\
                    &     (0.048)         &     (0.042)         &     (0.068)         &     (0.040)         \\
\hline
Observations        &      783394         &     1978400         &     1280504         &     1804551         \\
\hline\hline
\multicolumn{5}{l}{\footnotesize Controls: firm-CEO-spell fixed effects; industry-year fixed effects.}\\
\end{tabular}
\end{table}



\end{document}

% Available BibTeX keys from references.bib:
% McGrattan2012RED
% DeLoecker2011Econometrica
% AtkesonKehoe2005JPE
% Navaretti2010EFIGE
% Abowd1999Econometrica
% Card2018JoLE
% Bertrand2003-io
% Halpern2015-se
% Fisman2014-pw
% reghdfe
% greene
% cegtv
% frydman2010executive
% bandiera2020ceo
% Arnold2009-so
% bennedsen2020ceos
% Koren2017-rs
% Lucas1978-rp
% Syverson2011-ti
% cegjegyzek2024
% merleg2024
% McGrattanPrescott2008WP396
% Bloom2014-ux
