\documentclass[11pt,a4paper]{article}
\usepackage[utf8]{inputenc}
\usepackage[T1]{fontenc}
\usepackage{amsmath,amsfonts,amssymb}
\usepackage{apacite}
\usepackage{natbib}
\usepackage{graphicx}
\usepackage{booktabs}
\usepackage{threeparttable}
\usepackage{url}
\usepackage{hyperref}
\usepackage[margin=2.5cm]{geometry}
\usepackage{setspace}
\usepackage{comment}
\onehalfspacing

\newcommand{\Var}{\text{Var}}
\newcommand{\Cov}{\text{Cov}}

% Define \sym command for significance stars from esttab
\newcommand{\sym}[1]{{#1}}

\title{Estimating the Value of CEOs in Privately Held Businesses\thanks{Project no. 144193 has been implemented with the support provided by the Ministry of Culture and Innovation of Hungary from the National Research, Development and Innovation Fund, financed under the KKP\_22 funding scheme. This project was funded by the European Research Council (ERC Advanced Grant agreement number 101097789). The views expressed in this research are those of the authors and do not necessarily reflect the official view of the European Union or the European Research Council. \emph{Author contributions:} Conceptualization and study design: Koren, Orbán and Telegdy. Data curation, integration and quality assurance: Szilágyi and Vereckei. Statistical analysis: Koren and Telegdy. Writing the original draft: Koren. Review and editing: Koren, Orbán and Telegdy.}}

\author{Miklós Koren\thanks{Central European University, HUN-REN Centre for Economic and Regional Studies, CEPR and CESifo. E-mail: korenm@ceu.edu} \\
        Krisztina Orbán\thanks{Monash University.} \\
        Bálint Szilágyi\thanks{HUN-REN Centre for Economic and Regional Studies.} \\
        Álmos Telegdy\thanks{Corvinus University of Budapest.} \\
        András Vereckei\thanks{HUN-REN Centre for Economic and Regional Studies, Institute of Economics.}}

\date{\today}

\begin{document}

\maketitle

\begin{abstract}
How much do CEOs matter for firm performance? We estimate the causal effect of CEO quality on productivity using comprehensive administrative data covering the universe of Hungarian firms and CEOs from 1992--2022. We develop a production function framework that separates owner-controlled strategic decisions from CEO-controlled operational decisions. To address the severe measurement error in CEO fixed effects arising from short tenures, we introduce a placebo-controlled event study design: we compare actual CEO transitions to randomly assigned fake transitions in firms with stable leadership. While raw estimates suggest 22 percent performance differences between good and bad CEOs, our placebo control reveals that about two-thirds of this apparent variation is mechanical noise. The true causal effect is 7--8 percent; economically meaningful but only one-third of the raw correlation. The placebo control methodology provides a general solution for estimating individual effects in short-panel settings where traditional bias-correction methods fail.
\end{abstract}

\textbf{Keywords:} CEO value, private firms, productivity

\textbf{JEL Classification:} D24, G34

\newpage

%%%%%%%%%%%%%%%%%%%%%%
\section{Introduction}
%%%%%%%%%%%%%%%%%%%%%%

%Why are CEOs important?
To what extent do CEOs contribute to firm performance? Good management can improve firm productivity, as suggested by intervention studies that improve management practices \citep{bloom2013does}, training programs that improve managerial skills \citep{mckenzie2021small}, or leadership changes that reshape organizations \citep{Bertrand2003-io,bennedsen2020ceos,metcalfe2023managers}. 

%Why is it hard to measure CEO contribution?
Quantifying the contribution of an individual CEO, however, is a difficult task for a number of reasons. Modern businesses rely on a wide variety of inputs to produce, including tangible and intangible capital (plants, production lines, technical and organizational know-how, brand value), labor, intermediate inputs, and raw materials. Among these, only one is managerial capital and its effects are difficult to isolate from those of other inputs. It is also difficult to know who decides about these inputs: the CEO, the owner of the firm or they make decisions together? Even if the owner decides, she will take into account the CEO's capabilities.\footnote{Accurately measuring CEO contribution in private businesses is important for studying the manager market and executive pay practices. Recent work has compared CEO pay to CEO value in public firms \citep{tervio2008difference,gabaix2008ceo}, but similar analysis for private firms remains limited.} 

%what we do
In this paper, we model and quantify the value of CEOs. We make three contributions to the literature. First, we develop a model that accounts for the division of control between owners who set fixed inputs (such as tangible and intangible capital) and CEOs who optimize variable inputs (labor, materials). This framework clarifies what CEOs can and cannot affect, enabling cleaner identification of their contribution. Second, we assemble comprehensive administrative data covering the universe of Hungarian firms and their CEO networks over three decades (1992-2022). Third, we employ a regression design similar to what \cite{abowd1999high} introduced and has been used to uncover managerial effects since the beginning of this strand of literature \citep{Bertrand2003-io}. While we use this framework, we introduce a placebo-controlled event study design that separates true CEO effects from the mechanical noise that contaminates fixed effects estimates when managers have short tenures.

%Why is our approach unique
Papers have used three types of approaches when analyzing the effect of CEOs. One group studied the effects of a distinct, measurable variable, such as managerial practices used in the firm \citep{} or a CEO attribute (age, gender, tenure, education or country of origin \citep{}). These studies are useful to show the effects of the studied variables, but they are naturally limited in looking at CEO effects overall. Another group used an external event to assess the effect of CEO turnover (e.g., \citet{bennedsen2020ceos}). While this approach brought about credible causal estimates, lack of data means they cannot be used in most situations. 

The third group employed CEO fixed effects to separate the quality of firm from that of the CEOs \citep{Bertrand2003-io,crossland2011differences,quigley2015has}. While these studies separate the overall effect of CEOs, the generalizability of their results is challenged by the data used: almost all of these papers base their analysis on public companies.\footnote{\citep{quigley2022ceo} is a notable exception studying CEO effects in private businesses.} While our empirical approach is similar to the one used in these studies, our data has information on the universe of Hungarian businesses and their CEOs for the period between 1992 and 2022, tracking over 1 million firms and 300,000 CEOs. The overwhelming majority of the firms in our sample are small and medium sized enterprises (SMEs). This adds a new dimension to the analysis that was largely missing to date.

Our modeling approach treats CEOs in private firms as plant managers responsible for operational decisions. This simplification underestimates their strategic importance but provides a better approximation than models of superstar CEOs in public companies. This characterization aligns with evidence from family firms where professional managers face constraints on strategic decisions while maintaining operational autonomy \citep{zellweger2012managing}, and with studies of multi-plant firms where headquarters retains capital allocation authority while delegating production decisions to plant managers \citep{bloom2012americans}.

Comprehensive administrative data from Hungary offers several advantages. Mandatory registration of all company directors, including CEOs, ensures complete coverage without selection bias. The transition economy context likely features greater variation in managerial quality than mature markets, enhancing statistical power. The large sample size and long time span enable construction of CEO mobility networks essential for identification. The institutional context matters: \citet{crossland2011differences} show that national institutions affect CEO discretion, with civil-law countries like Hungary typically granting less managerial autonomy than common-law systems, making our owner-CEO division particularly relevant.

Because CEO tenures in our data are short, estimated CEO effects contain substantial noise from averaging residual productivity shocks over few observations. This small-sample bias, known in the labor economics literature as ``limited mobility bias'' \citep{andrews2008high}, complicates interpretation of fixed effects estimates. To address this, we introduce a placebo-controlled event study design.

To illustrate our approach, consider a firm with the same CEO from 2000-2010. We randomly assign a fake transition in 2005, creating two pseudo-CEOs. If estimated `effects' for these pseudo-CEOs diverge substantially, this reveals the noise problem in fixed effects estimation. By comparing actual CEO transitions to these placebo transitions, we can correct the fixed effects estimates for noise, isolating the true CEO contribution.

Our headline result challenges conventional wisdom about CEO importance. The naive comparison suggests firms hiring good CEOs outperform those hiring bad CEOs by 22.1 percent—a large effect consistent with the view that leadership quality is paramount. However, our placebo control reveals 14.8 percent of this difference arises from noise in the estimation process, not skill differences. The true causal effect of CEO quality is 7.4 percent: economically meaningful but only 33 percent of what raw correlations suggest. Three-quarters of apparent variation in CEO quality is spurious.

Our work connects to the broader literature on management practices and firm productivity. Randomized controlled trials demonstrate that management training and consulting improve firm performance \citep{bloom2013does}, but these interventions change practices rather than people. Whether replacing managers generates similar gains remains contentious. Evidence from public sector organizations suggests modest manager effects \citep{fenizia2022managers, janke2024role}, while studies of family firms find larger impacts when professional managers replace family members \citep{bennedsen2007inside}. Our results for private firms fall between these extremes: CEOs matter, but less than raw correlations suggest.

Methodologically, our paper builds on the two-way fixed effects literature in labor economics that decomposes wages into worker and firm components \citep{Abowd1999Econometrica, Card2018JoLE}. These studies face similar challenges from limited mobility creating small-sample bias \citep{andrews2008high} and have developed bias-correction methods \citep{Bonhomme2023-dx, gaure2014correlation}. We adapt this framework to the CEO-firm setting but add placebo controls to separate signal from noise. This approach is valuable when studying managers who, unlike workers, have few observations per individual, making traditional bias-correction methods less effective. Recent work has documented apparently increasing CEO effects over time \citep{quigley2015has}, but these studies do not account for the mechanical noise we identify. \citet{lippi2014corporate} find that concentrated ownership in Italian firms distorts executive selection and reduces productivity by 10\%, providing motivation for our framework separating owner and CEO decisions.

%%%%%%%%%%%%%%%%%%%%%%%%%%%%%%%%%%
\section{A Model of Production with Owner- and Manager-Controlled Inputs}
%%%%%%%%%%%%%%%%%%%%%%%%%%%%%%%%%%%

Our model highlights a key institutional feature of private firms: the separation of strategic decisions (owner-controlled) from operational decisions (manager-controlled).\footnote{Both theoretical results \citep{fama1983separation, jensen1976theory, burkart2003family, schulze2021property} and empirical evidence \citep{durand2003ownership, gao2015comparison, quigley2022does, cole2008privately, nakazato2011executive, gompers2023market, bloom2012organization, wang2019decentralization, buffington2017mops} confirm that in privately held firms and firms with concentrated ownership, managerial discretion on strategic decisions is limited and owner control dominates.} This division of decision rights, common in private businesses, affects how we identify manager effects: we cannot control for variable inputs when estimating manager productivity because variable input levels are an outcome of manager decisions and are correlated with productivity.\footnote{A similar point was made by \citet{Gandhi2020-nu}, who emphasize that freely adjustable inputs are ``bad controls'' because they are determined by productivity.} 

Firms produce output using a Cobb-Douglas production function with both fixed and variable inputs. The production function for firm $i$ with manager $m$ at time $t$ is:
\begin{equation}\label{eq:production}
Q_{imt} = A_i Z_{m} \varepsilon_{it} K_{it}^\alpha L_{imt}^{\beta} M_{imt}^{\gamma}
\end{equation}
where $A_i$ represents time-invariant organizational capital (location, brand value, customer relationships), $Z_m$ captures manager skill, $\varepsilon_{it}$ is residual productivity, $K_{it}$ is physical capital, $L_{imt}$ is labor input, and $M_{imt}$ is intermediate input usage. 

Standard production function estimation combines the first three components into a single measure: $\Omega_{it} = A_i Z_m \varepsilon_{it}$, called total factor productivity (TFP). Our framework decomposes TFP into firm-specific advantages ($A_i$), manager-specific skill ($Z_m$), and residual productivity shocks ($\varepsilon_{it}$) to identify the manager contribution.

We assume physical capital investment ($K_{it}$) and organizational assets ($A_i$), including location choices, brand development, and CEO hiring decisions are predetermined. Managers control labor hiring ($L_{imt}$) and input purchases ($M_{imt}$). 

The production function exhibits decreasing returns to scale in variable inputs ($\beta + \gamma < 1$), which pins down optimal firm size even under perfect competition. Fixed inputs $A_i$ and $Z_m$ create firm-specific and manager-specific advantages that generate economic rents.

Managers maximize profit by choosing variable inputs optimally given the predetermined choices. Under sector-specific output price $P_{st}$, wage rate $W_{st}$, and material price $\varrho_{st}$, the first-order conditions yield closed-form solutions for optimal input demands. Substituting these back into the revenue function gives:
\begin{equation}\label{eq:revenue}
R_{imst} = (P_{st}A_i Z_m \varepsilon_{it})^{1/\chi}
K_{it}^{\alpha/\chi}
W_{st}^{-\beta/\chi}
\varrho_{st}^{-\gamma/\chi}
(1-\chi)^{(1-\chi)/\chi}
\end{equation}
Revenue increases in manager skill $Z_m$, organizational capital $A_i$, and physical capital $K_{it}$, while decreasing in input prices. The elasticity of revenue with respect to manager skill is $1/\chi > 1$, with $\chi := 1 - \beta - \gamma$, the share of surplus in revenue. 

The surplus accruing to fixed factors (factors other than labor and materials), which owners and managers jointly maximize, equals revenue minus payments to variable inputs:
\begin{equation}\label{eq:surplus}
S_{imst} = R_{imst} - W_{st}L_{imt} - \varrho_{st}M_{imt} = \chi R_{imst}
\end{equation}
Under Cobb-Douglas technology, surplus is a constant fraction $\chi$ of revenue. This proportionality allows us to work directly with revenue, simplifying estimation while preserving economic insights. Taking logarithms of the revenue equation:
\begin{equation}\label{eq:log_revenue}
r_{imst} = C+\frac{\alpha}{\chi} k_{it} + \frac{1}{\chi} z_{m} + \frac{1}{\chi} a_i + \frac{1}{\chi} p_{st} + \frac{1}{\chi}\epsilon_{it} 
- \frac{\beta}{\chi} w_{st} - \frac{\gamma}{\chi} \rho_{st}
\end{equation}
where lowercase letters denote logarithms (e.g., $\epsilon_{it} = \log \varepsilon_{it}$) and $C$ is a constant.

The value of replacing manager $m$ with manager $m'$ at the same firm is the extent to which surplus changes as the manager changes:
\begin{equation}\label{eq:manager_value}
r_{im'st} - r_{imst} = \frac{1}{\chi}(z_{m'} - z_{m})
\end{equation}
Manager value equals the skill difference scaled by $1/\chi$. This scaling reflects a leverage effect: a 1\% better manager hires more variable inputs, thereby increasing revenue and surplus by more than 1\%.

Our model abstracts from corporate governance frictions that could lead managers to make suboptimal decisions. These frictions are less problematic in concentrated ownership settings where owners retain control over strategic choices. While agency problems may affect long-term vision, risk-taking, and entrepreneurship, the incentive to increase revenue and reduce operating costs remains aligned between owners and managers in most private firms.

The model excludes dynamic considerations such as adjustment costs or forward-looking behavior. Strategic decisions are forward-looking due to their persistence, but we treat them as predetermined from the manager's perspective. Even though the optimization framework is static, our empirical application allows for arbitrary time-series correlation within and across variables. The data validate this choice: strategic variables like capital evolve slowly with little correlation to manager changes, while operational variables adjust immediately following CEO transitions.

\section{Corporate Data from Hungary}

Hungary provides an ideal setting for studying CEO effects in private firms. The country offers complete administrative data coverage for all incorporated businesses with mandatory CEO registration, spanning 30 years from the transition economy of the 1990s through EU accession in 2004 to the present. This coverage enables us to track CEO careers across firms and construct mobility networks necessary for identification.

Our analysis combines two administrative datasets. The firm registry, maintained by Hungarian corporate courts, contains records on all company representatives—individuals authorized to act on behalf of firms. These records include CEOs and other executives with signatory rights, tracked through a temporal database where each entry reflects representation status over specific time intervals. Updates occur when positions change, personal identifiers are modified, or reporting standards evolve. The registry provides names, addresses, dates of birth (from 2010), and mother's names (from 1999), though numerical identifiers exist only from 2013 onward.

The balance sheet dataset contains annual financial reports for all Hungarian firms required to file statements. This includes sales revenue, export revenue, employment counts, tangible and intangible assets, raw material and intermediate input costs, personnel expenses, and indicators for state and foreign ownership. The two datasets cover 1,063,172 firms over 31 years, yielding 9,627,484 firm-year observations before sample restrictions.

Identifying individual CEOs poses some challenges. Before 2013, no numerical identifiers existed, requiring entity resolution based on names, addresses, mother's names, and birthdates. We link records across these dimensions to create unique person identifiers, enabling tracking across firms and over time. Matching quality improves after 1999 (when mother's names reporting begins) and 2010 (when birthdates reporting starts), though the 1990s data achieves reasonable coverage through name and address matching. 

CEO identification within firms requires heuristics since job titles are inconsistently recorded. When explicit ``managing director'' titles exist, we use them directly. For remaining cases, we assume all representatives are CEOs if three or fewer exist at the firm. When more than three representatives are present, we assign CEO status based on continuity with previously identified CEOs. Time spans between appointments are often unclosed or non-contiguous, requiring imputation based on sequential information, assuming representatives remain active if their tenure includes June 21 of each year.

We apply several restrictions to create a sample suitable for productivity analysis. First, we exclude mining and finance sectors due to specialized accounting frameworks and regulatory environments. Second, we drop firms ever having more than 2 simultaneous CEOs (removing observations for firms with complex governance) to avoid complex governance structures that complicate identification. Third, we exclude firms with more than 10 CEO changes over the sample period (removing observations for unstable firms) to reduce noise from misclassified transitions. Fourth, we remove all state-owned enterprises, as their objectives and constraints differ from private businesses \citep{orban2019inception}.

We restrict attention to firms that employ at least 5 workers at some point in the firm lifecycle. This filter removes a substantial portion of observations but eliminates shell companies, tax optimization vehicles, and self-employment arrangements masquerading as corporations. The remaining firms represent genuine businesses with meaningful economic activity where management quality affects performance.

We exclude public firms and joint-stock companies from our analysis. The few companies traded on the Budapest Stock Exchange operate under different governance structures, compensation schemes, and disclosure requirements than private businesses. We also exclude cooperatives and other non-standard corporate forms where multiple managing directors share executive authority, as these organizational structures complicate identification of individual CEO effects.

\begin{table}[htbp]
\centering
\caption{Sample Over Time}
\label{tab:sample}
\begin{tabular}{*{6}{c}}
\toprule
Year & \shortstack{Total\\firms} & \shortstack{Sample\\firms} & CEOs & \multicolumn{2}{c}{Connected component} \\
\cmidrule(lr){5-6}
 & & & & Firms & CEOs \\
\midrule
1992 &       98,780 &       28,554 &       34,432 &           25 &           28 \\
1995 &      171,759 &       46,524 &       54,118 &           46 &           45 \\
2000 &      280,386 &       63,376 &       72,636 &           67 &           64 \\
\midrule
Total &      431,178 &       86,111 &      122,424 &           84 &          105 \\
\bottomrule
\end{tabular}
\begin{minipage}{12cm}
\footnotesize
\textit{Notes:} This table presents the evolution of the sample from 1992 to 2022. Column (1) shows the total number of distinct firms with balance sheet data. Column (2) shows the number of distinct firms after applying data quality filters. Column (3) shows the number of distinct CEOs. Columns (4) and (5) show the subset of distinct firms and CEOs that belong to the largest connected component of the manager network, where managers are connected if they have worked at the same firm. The table shows every fifth year plus the first year (1992), last year (2022), and totals of distinct counts. \end{minipage}
\end{table}


Given the 1990s' rapid economic liberalization, transition to market economy, and foreign direct investment inflows, we conducted robustness checks restricting the sample to post-2004 data following Hungary's EU accession. The results remain unchanged, confirming our findings are not driven by the transition period's institutional environment (see Figure 1 below).

CEO mobility creates the manager-firm network essential for fixed effects identification. The largest connected component of the bipartite firm-CEO network contains 22,001 managers and a comparable number of firms, 6.5 percent of all managers but representing a substantial share of economic activity.

CEO tenure varies across firms. While 63 percent of firms retain the same CEO throughout their observed lifetime, the remaining firms experience leadership transitions that enable identification. Among firm-years with CEO information, 80 percent have single CEOs, 17 percent have two, and 3 percent have more (before our sample restrictions). CEO spell lengths follow an exponential distribution with a 20 percent annual hazard rate, implying typical tenures of 3-7 years.

\begin{table}[htbp]
\centering
\caption{CEO Patterns and Spell Length Analysis}
\label{tab:ceo_patterns}

\textbf{Panel A: Number of CEOs per Firm}
\begin{tabular}{lcc}
\toprule
CEOs & Firm-Year & Firm \\
\midrule
1 & 80\% & 63\% \\
2 & 17\% & 24\% \\
3 & 2\% & 8\% \\
4+ & 1\% & 5\% \\
Total &    9,627,484 &    1,012,113 \\
\bottomrule
\end{tabular}

\vspace{0.5cm}

\textbf{Panel B: CEO Spell Length Distribution}
\begin{tabular}{lcc}
\toprule
Length & Actual & Placebo \\
(Years) & Spells & Spells \\
\midrule
1 & 22\% & 27\% \\
2 & 15\% & 18\% \\
3 & 12\% & 14\% \\
4+ & 52\% & 41\% \\
Total &       99,328 &       13,860 \\
\bottomrule
\end{tabular}

\begin{minipage}{12cm}
\footnotesize
\textit{Notes:} Panel A shows the distribution of CEO counts per firm across firm-years and firms. Panel B compares the spell length distribution between actual CEO transitions and synthetically generated placebo transitions. The similar distributions validate our placebo methodology.
\end{minipage}
\end{table}


\section{Estimation}

Our estimation proceeds in four steps: measuring the surplus share (the share of revenue accruing to fixed factors), estimating the revenue function, recovering manager fixed effects, and validating causality through event studies. Each step builds toward separating true CEO effects from the noise that contaminates raw estimates.

\paragraph{Step 1: Measuring the Surplus Share.} The parameter $\chi$---the share of surplus in revenue---determines how manager skill translates into firm performance. Under Cobb-Douglas technology, this share equals one minus the combined revenue shares of labor and materials. Following \citet{Gandhi2020-nu}, we measure $\chi$ from the data as:
\begin{equation}
\hat{\chi}_s = 1 - \frac{\sum_{i \in s}(W_{st}L_{it} + \varrho_{st}M_{it})}{\sum_{i \in s} R_{it}}
\end{equation}
where the summation runs over firms in sector $s$. This approach yields sector-specific estimates of $\chi$. The $1/\chi$ scaling in our framework creates a leverage effect where manager skill is amplified in its impact on firm performance.

\paragraph{Step 2: Estimating the Revenue Function.} With $\hat{\chi}$ measured, we estimate the revenue function to recover the capital elasticity and control for observable factors. Using lowercase letters for logarithms, the estimating equation is the empirical counterpart of (\ref{eq:log_revenue}):
\begin{equation}
r_{imst} = \frac{\alpha}{\chi} k_{it} + \frac{1}{\chi}z_m + \lambda_i + \mu_{st} + \tilde{\epsilon}_{it}
\end{equation}
where $r_{imst} = \log R_{imst}$ is log revenue, $k_{it} = \log K_{it}$ is log capital, $\lambda_i$ captures firm fixed effects, $\mu_{st}$ are sector-year fixed effects, and $\tilde{\epsilon}_{it} = \epsilon_{it}/\chi$ is rescaled residual productivity. We include controls for firm age, intangible asset presence, and foreign ownership, though these minimally affect the capital coefficient.

The key assumptions are: (1) all firms within a sector face the same prices, and (2) for each manager, residual productivity $\tilde{\epsilon}_{it}$ has zero mean when averaged across all their firms and time periods: $E[\bar{\epsilon}_m] = 0$ where $\bar{\epsilon}_m = \frac{1}{N_m}\sum_{i,t \in m} \tilde{\epsilon}_{it}$ and the sum runs over all firm-year observations under manager $m$.

We do not require random manager mobility or that residual productivity has zero mean at the point of CEO transition. Manager assignment can be endogenous: good managers may systematically move to firms experiencing positive shocks or be hired when firms anticipate improvements. We only require that these shocks average to zero over a manager's entire career. This is a weaker assumption than random assignment but still substantive: it rules out managers who systematically arrive at permanently improving (or declining) firms.

We estimate using OLS with high-dimensional fixed effects via \texttt{reghdfe} \citep{reghdfe}. The coefficient on log capital is $\hat{\alpha}/\hat{\chi}$, which we multiply by $\hat{\chi}$ to recover $\hat{\alpha}$.

\paragraph{Step 3: Recovering Manager Fixed Effects.} After estimating the revenue function, we compute log total factor productivity by removing the contribution of capital and sectoral prices from revenue:
\begin{equation}
\omega_{imst} = \hat{\chi} r_{imst} - \hat{\alpha} k_{it} - \hat{\mu}_{st} = z_m + \lambda_i + \epsilon_{it}
\end{equation}
This measure of log TFP contains manager skill, firm effects, and residual productivity. In standard production function estimation, this entire term would be treated as a single TFP measure. Our decomposition separates the manager contribution from other sources of productivity.

We estimate a two-way fixed effects model with firm and manager fixed effects. Our identification relies on the zero-mean condition described above: residual productivity must average to zero for each manager across their career. This allows for various forms of endogenous mobility but rules out systematic patterns where managers consistently join permanently improving or declining firms. 

The event study provides a diagnostic test for this identification assumption. Pre-trends in productivity before CEO transitions would suggest (though not prove) that the zero-mean assumption is violated. If productivity systematically rises before good CEOs arrive, we worry that the positive trend continues post-transition, violating $E[\bar{\epsilon}_m] = 0$. Conversely, the absence of pre-trends makes it harder to construct plausible endogeneity stories. While we cannot rule out contemporaneous shocks that coincide exactly with CEO changes (e.g., owners simultaneously firing the CEO and adopting new technology), such precise timing is less plausible than gradual changes that would manifest as pre-trends. Our event studies show no significant pre-trends, supporting but not proving our identification assumption.\footnote{The absence of strong pre-trends in our data contrasts with evidence from \citet{cornelli2013monitoring} showing boards actively monitor and replace CEOs when performance deteriorates in public firms, suggesting our private firm transitions may be less performance-driven. While \citet{jenter2021performance} find 38-55\% of turnovers are performance-induced in U.S. public firms, our private firm setting likely features more random CEO transitions given the absence of market pressures and board oversight.}

The system of fixed effects is identified only within connected components: groups of firms and managers linked through mobility. Two managers can be compared if they worked at the same firm or connect through a chain of shared employment relationships. We can estimate $\hat z_m$ for every manager, but only up to a constant term that may vary across connected components. Our largest connected component contains 22,001 managers, enabling comparisons within this network. We normalize the manager effect to be mean zero in the largest connected component.

The connected component represents a non-random subset of the economy, with member firms larger than average. This selection is not problematic for our identification strategy since we allow manager effects to correlate arbitrarily with observable characteristics. We cannot extend the analysis to singleton firms or disconnected components where manager fixed effects lack a common reference point for interpretation.

\paragraph{Step 4: Placebo-Controlled Event Studies.} Even when $\epsilon$ is orthogonal to $z$, estimated fixed effects contain substantial small-sample noise when manager transitions are infrequent and manager tenures are short.\footnote{Worker-firm fixed effect studies face similar challenges called ``limited mobility bias'' \citep{andrews2008high} and have developed bias-correction methods \citep{Bonhomme2023-dx, gaure2014correlation}.}

To understand the sources of small-sample bias and how we address it, we remove the firm fixed effect from TFP by subtracting the firm average:
\begin{equation}
\Delta\omega_{imt} = \Delta z_{m_{it}} + \Delta\epsilon_{it},
\end{equation}
where $\Delta x_{it} := x_{it} - \frac{1}{N_i}\sum_{\tau} x_{i\tau}$ denotes the deviation of a variable from its within-firm mean. When a firm changes CEOs, the change in log TFP captures both the true skill difference and accumulated noise. The noise component---the average of residual productivity shocks during each manager's tenure---dominates the signal when tenures are short. 

Our solution leverages a simple insight: when CEOs do not change, we still observe variation in log TFP driven purely by noise:
\begin{equation}
\Delta\omega_{imt} = \Delta\epsilon_{it}.
\end{equation}
By applying the same estimation procedure to ``non-changes,'' we can measure and filter out the noise component.

We construct placebo CEO transitions in three steps. First, we estimate the time-varying hazard of actual CEO changes. Second, we identify firms with long CEO tenures (7+ years) where no actual change occurs. Third, we randomly assign placebo transitions to these stable firms using the estimated hazard function.

For instance, a firm with the same CEO from 1992-2000 might receive a placebo transition in 1996, artificially splitting the tenure into two pseudo-CEOs. The timing follows the empirical hazard, ensuring the mechanical noise properties—averaging residual productivity over varying tenure lengths—match between actual and placebo groups.

To validate our placebo methodology, we verify that synthetically generated CEO transitions match actual patterns. Taking firms with stable leadership (same CEO for 7+ years), we randomly assign placebo transitions using the empirical hazard function. The resulting spell length distribution mirrors actual CEO changes (Panel B of Table 2).

We implement the event study around CEO transitions at time $g$, comparing actual changes (treatment) to placebo changes (control):
\begin{equation}
\omega_{imt} = a_i + \gamma_{t-g} + \epsilon_{it}
\end{equation}
The coefficients $\gamma_{t-g}$ capture the evolution of log TFP in event time, where $t-g \in [-4, 3]$ and we normalize $\gamma_{-1} = 0$.

Because both groups receive a ``treatment'' (actual or placebo), standard difference-in-differences cannot be used. We adapt the \citet{Callaway2021JoLE} estimator for two treatment types using the \texttt{xt2treatments} package \citep{Koren2024xt2treatments}. The key innovation is precisely aligning transitions in event time—both actual and placebo changes occur at $t=0$, enabling a clean comparison of dynamics between treated and control firms.


%%%%%%%%%%%%%%%%%
\section{Results}
%%%%%%%%%%%%%%%%%

\subsection{Production Function Estimates}

Table \ref{tab:surplus_share} in the Appendix reports our estimates of the surplus share $\chi$ by industry. The estimates range from 0.06 to 0.19 across included sectors, with wholesale, retail, and transport showing the lowest values and professional services the highest. These estimates imply that a 1\% increase in TFP increases revenue by 5 to 16 percent through the leverage effect of scaling variable inputs.

Table \ref{tab:revenue_function} presents the revenue function estimates. We include controls for log capital, firm age, presence of intangible assets, and foreign ownership, along with firm-CEO and sector-year fixed effects. The capital coefficient is precisely estimated at 0.333 (standard error 0.001), consistent with capital's limited but significant role in private businesses. The intangible asset dummy shows a positive coefficient of 0.277, while foreign ownership contributes 0.027 to log revenue. Firm age exhibits diminishing returns with a coefficient of -0.094 on log age. These controls explain 76\% of the variation in log revenue. Robustness checks varying the control set and sample restrictions, reported in the Appendix, yield similar estimates.

Founder-managed firms exhibit lower revenue conditional on owner inputs, implying lower total factor productivity—consistent with evidence that family firms sacrifice growth for control \citep{bennedsen2007inside}. The performance penalty from family succession is well-documented, with heir CEOs reducing ROA by 6 percentage points in Danish firms \citep{bennedsen2007family}. 

\begin{table}[htbp]\centering
\def\sym#1{\ifmmode^{#1}\else\(^{#1}\)\fi}
\caption{Surplus Function Estimation Results}
\begin{tabular}{l*{6}{c}}
\toprule
                    &\multicolumn{1}{c}{(1)}&\multicolumn{1}{c}{(2)}&\multicolumn{1}{c}{(3)}&\multicolumn{1}{c}{(4)}&\multicolumn{1}{c}{(5)}&\multicolumn{1}{c}{(6)}\\
                    &\multicolumn{1}{c}{Revenue}&\multicolumn{1}{c}{EBITDA}&\multicolumn{1}{c}{Wagebill}&\multicolumn{1}{c}{Materials}&\multicolumn{1}{c}{Revenue}&\multicolumn{1}{c}{Revenue}\\
\midrule
Fixed assets (log)  &       0.336\sym{***}&       0.340\sym{***}&       0.304\sym{***}&       0.386\sym{***}&       0.294\sym{***}&       0.298\sym{***}\\
                    &     (0.001)         &     (0.001)         &     (0.001)         &     (0.002)         &     (0.001)         &     (0.004)         \\
\addlinespace
Has intangible assets&       0.280\sym{***}&       0.173\sym{***}&       0.281\sym{***}&       0.327\sym{***}&       0.209\sym{***}&       0.259\sym{***}\\
                    &     (0.003)         &     (0.003)         &     (0.003)         &     (0.004)         &     (0.003)         &     (0.010)         \\
\addlinespace
Foreign owned       &       0.023\sym{*}  &       0.019         &       0.061\sym{***}&       0.008         &       0.025\sym{**} &       0.035         \\
                    &     (0.012)         &     (0.012)         &     (0.013)         &     (0.015)         &     (0.011)         &     (0.025)         \\
\addlinespace
Founding owner      &      -0.097\sym{***}&      -0.064\sym{***}&      -0.039\sym{***}&      -0.115\sym{***}&      -0.016\sym{***}&      -0.021\sym{**} \\
                    &     (0.005)         &     (0.005)         &     (0.005)         &     (0.006)         &     (0.003)         &     (0.008)         \\
\addlinespace
Non-founding owner  &      -0.008         &      -0.008         &      -0.012\sym{*}  &      -0.013         &      -0.011\sym{***}&      -0.018\sym{*}  \\
                    &     (0.007)         &     (0.006)         &     (0.007)         &     (0.009)         &     (0.004)         &     (0.010)         \\
\midrule
Observations        &     3004184         &     2326192         &     2949024         &     3059662         &     2967233         &      374084         \\
\bottomrule
\multicolumn{7}{l}{\footnotesize Standard errors in parentheses}\\
\multicolumn{7}{l}{\footnotesize All models include firm-CEO-spell fixed effects and industry-year fixed effects. Outcome variables are}\\
\multicolumn{7}{l}{\footnotesize log-transformed. Models (5) and (6) include quadratic controls for firm age and CEO tenure.}\\
\multicolumn{7}{l}{\footnotesize Model (6) restricts to largest connected component.}\\
\multicolumn{7}{l}{\footnotesize \sym{*} \(p<0.10\), \sym{**} \(p<0.05\), \sym{***} \(p<0.01\)}\\
\end{tabular}
\end{table}


The stability of our estimates across samples validates the connected component analysis. Revenue function coefficients are nearly identical when estimated on the connected component versus the full sample (Table 3, Model 6), suggesting production technology and manager effects operate similarly across both groups. While selection into the connected component is non-random—these firms are larger and hire external managers more often—the economic relationships appear invariant to this selection, supporting our use of the connected component to identify manager fixed effects.

\subsection{The Placebo Test}

Figure \ref{fig:event_study_main} presents a four-panel analysis. Panel (a) compares TFP of firms receiving a better CEO (improving $Z_m$) to those receiving a worse CEO (declining $Z_m$). The two lines correspond to actually treated firms and placebo-treated firms, allowing us to isolate true CEO effects from mechanical noise. There are substantial pre-trends in the raw data, but these are mirrored in the placebo group, indicating they arise from noise rather than true skill differences.

\begin{figure}[htbp]
\centering
\includegraphics[width=\textwidth]{figure/event_study.pdf}
\caption{Placebo-Controlled Event Studies of CEO Transitions}
\label{fig:event_study_main}
\end{figure}

Using the placebo-treated firms as control (Panel b), we can isolate the true CEO effect. The difference between the two lines in Panel (b) shows that better CEOs increase TFP by 4.4 percent, whereas worse CEOs decrease it by 3.3 percent, yielding a difference of 7.7 percent. This is about a third of the raw estimate. The absence of pre-trends suggests that TFP changes are driven by CEO changes rather than the other way around.

Panel (c) confirms robustness in the post-2004 period, excluding the first decade of transition. If anything, the CEO effects are slightly larger in this more stable institutional environment. Panel (d) shows the effects of outsider-to-outsider transitions, excluding founder-managed firms. This setting provides the cleanest identification of CEO effects separate from ownership changes, albeit in a smaller sample, and, hence, larger standard errors.

Appendix Figure A2 presents complementary evidence using the variance of outcomes, which increases sharply at CEO transitions. This is consistent with real heterogeneity in CEO quality, which implies greater outcome dispersion when CEOs change.

\subsection{Validation: Differential Effects on Manager vs Owner Variables}

Our model predicts that CEOs should primarily affect outcomes they control (labor, materials) rather than those controlled by owners (capital, organizational structure). Figure \ref{fig:owner_manager_control} presents event studies for owner-controlled variables (fixed assets, intangibles, foreign ownership) and manager-controlled variables (employment, materials, revenue).

\begin{figure}[htbp]
\centering
\begin{minipage}{0.45\textwidth}
\includegraphics[width=\textwidth]{figure/event_study_owner_controlled.pdf}
\end{minipage}
\begin{minipage}{0.45\textwidth}
\includegraphics[width=\textwidth]{figure/event_study_manager_controlled.pdf}
\end{minipage}
\label{fig:owner_manager_control}
\caption{Evolution of Owner- and Manager-Controlled Variables Under New CEOs}
\end{figure}

%% FIXME: this will have to be properly formatted and use the same yscale


Good CEOs have immediate and substantial effects on manager-controlled variables. Log employment increases by 18.7 percent, log materials by 28.0 percent, and log revenue by 28.1 percent, all effects highly significant. (See Appendix A5 for estimated treatment effects and standard errors.) These effects appear immediately in year 0 and persist throughout the post-period, consistent with new CEOs quickly adjusting operational scale.

In contrast, owner-controlled variables show different patterns. Fixed assets exhibit no significant change (5.5 percent, not statistically different from zero). Foreign ownership probability and intangible asset use increase gradually over time, rather than jumping immediately. While the average effect is significant for these inputs, the slow buildup is consistent with our assumption that managers cannot quickly change these inputs.


\section{Conclusion}

This paper provides causal estimate of CEO value in private firms that accounts for the mechanical noise contaminating manager fixed effects. Using comprehensive administrative data from Hungary covering over one million firms and 340,000 CEOs, we develop a novel placebo-controlled event study design that separates true managerial impact from spurious variation. Our central finding is that two-thirds of the raw difference between the TFP under good and bad CEOs arises from noise; the remaining one-third reflects true skill differences. 

An alternative way to deal with the noise problem is to use observable manager characteristics as measures of skill. Observable characteristics such as education and work experience \citep{DePirro2025}, foreign name as a proxy for international exposure \citep{Koren2023expat}, and the selectiveness of entry cohorts \citep{koren2024managers} offer more reliable, though narrower, measures of specific dimensions of managerial quality. These observables capture only partial aspects of CEO ability but avoid the mechanical noise that contaminates fixed effects estimates.


\bibliographystyle{apalike}
\bibliography{references}

\appendix
\section{Online Appendix: Additional Tables and Figures}
\renewcommand{\thefigure}{A\arabic{figure}}
\renewcommand{\thetable}{A\arabic{table}}
\setcounter{figure}{0}
\setcounter{table}{0}

\subsection{Production Manager Autonomy in Family Firms}

\input{table/tableA0}

\subsection{Industry Breakdown}

\begin{table}[htbp]
\centering
\caption{Industry Breakdown}
\label{tab:industry_stats}
\begin{tabular}{l*{5}{r}}
\toprule
Industry (NACE) & \shortstack{Obs.} & \shortstack{Firms} & \shortstack{CEOs} & \shortstack{Surplus\\share (\%)} \\
\midrule
Agriculture, Forestry, Fishing (A) &       83,241 &       15,895 &       24,160 &   6.0 \\
Manufacturing (C) &      288,713 &       54,413 &       84,583 &  14.5 \\
Wholesale, Retail, Transportation (G,H) &      697,452 &      146,814 &      218,758 &   9.0 \\
Telecom, Business Services (J,M) &      348,738 &       72,107 &      101,622 &  29.6 \\
Construction (F) &      178,023 &       39,387 &       54,358 &  11.1 \\
Nontradable Services (Other) &      432,172 &       97,318 &      132,954 &  20.1 \\
Mining, Quarrying (B)* &        2,874 &          594 &        1,001 &  17.3 \\
Finance, Insurance, Real Estate (K,L)* &       20,883 &        4,650 &        8,099 &  27.2 \\
\bottomrule
\end{tabular}
\begin{minipage}{\textwidth}
\footnotesize
\textit{Notes:} This table presents industry-level summary statistics using the TEAOR08 classification system. Column (1) shows the industry name and corresponding NACE sector codes. Column (2) shows the total number of firm-year observations in the balance sheet data (1992-2022). Column (3) shows the number of distinct firms with balance sheet data. Column (4) shows the number of distinct managers (CEOs) from the firm registry data. Column (5) shows the average EBITDA as a percentage of revenue. Mining (sector B) and Finance/Insurance/Real Estate (sectors K,L) are excluded from the main analysis due to different production function characteristics. The NACE classification follows the Hungarian adaptation of the NACE Rev. 2 system. \end{minipage}
\end{table}


\begin{table}[htbp]\centering
\def\sym#1{\ifmmode^{#1}\else\(^{#1}\)\fi}
\caption{The revenue function by sector}
\begin{tabular}{l*{4}{c}}
\hline\hline
                    &\multicolumn{1}{c}{(1)}&\multicolumn{1}{c}{(2)}&\multicolumn{1}{c}{(3)}&\multicolumn{1}{c}{(4)}\\
                    &\multicolumn{1}{c}{Manufacturing}&\multicolumn{1}{c}{Wholesale, Retail, Transportation}&\multicolumn{1}{c}{Telecom and Business Services}&\multicolumn{1}{c}{Nontradable services}\\
\hline
Tangible and intangible assets (log)&       0.302\sym{***}&       0.262\sym{***}&       0.242\sym{***}&       0.212\sym{***}\\
                    &     (0.003)         &     (0.002)         &     (0.002)         &     (0.002)         \\
[1em]
Intangible assets share&       0.011         &      -0.010         &      -0.065\sym{***}&      -0.027\sym{*}  \\
                    &     (0.025)         &     (0.014)         &     (0.012)         &     (0.015)         \\
[1em]
Foreign owned       &       0.046\sym{*}  &       0.011         &       0.088\sym{***}&      -0.014         \\
                    &     (0.023)         &     (0.015)         &     (0.022)         &     (0.014)         \\
[1em]
State owned         &       0.083\sym{*}  &      -0.029         &      -0.133\sym{*}  &       0.006         \\
                    &     (0.048)         &     (0.042)         &     (0.068)         &     (0.040)         \\
\hline
Observations        &      783394         &     1978400         &     1280504         &     1804551         \\
\hline\hline
\multicolumn{5}{l}{\footnotesize Controls: firm-CEO-spell fixed effects; industry-year fixed effects.}\\
\end{tabular}
\end{table}



\subsection{Manager Skill Distributions}

\begin{figure}[htbp]
\centering
\begin{minipage}{0.48\textwidth}
\centering
\includegraphics[width=\textwidth]{figure/manager_skill_within.pdf}
\end{minipage}
\hfill
\begin{minipage}{0.48\textwidth}
\centering
\includegraphics[width=\textwidth]{figure/manager_skill_connected.pdf}
\end{minipage}
\caption{Manager Skill Distributions}
\label{fig:manager_skills_appendix}
\footnotesize
Notes: Panel A shows the distribution of within-firm manager skill variation for firms with multiple CEOs. Panel B shows the distribution of manager skills in the largest connected component of managers. Both distributions show manager skills in log points after normalization and scaling.
\end{figure}


% Additional tables and figures to be inserted here as needed

\subsection{Variance-Based Evidence for CEO Heterogeneity}

\begin{figure}[htbp]
\centering
\includegraphics[width=0.8\textwidth]{figure/event_study_panel_c.pdf}
\caption{TFP Moments Around CEO Transitions}
\label{fig:intangibles}
\end{figure}

Figure \ref{fig:variance_event_study} presents complementary evidence using the variance of log TFP around CEO transitions. Under our framework, if CEO changes introduce real heterogeneity in managerial quality, the cross-sectional variance of outcomes should increase at the transition point—some firms get better CEOs, others get worse ones. In contrast, pure noise or firm-specific trends would not systematically alter variance.

The variance analysis shows that actual CEO transitions are associated with increased dispersion in outcomes, while placebo transitions show no such effect. This provides model-free evidence that CEO transitions introduce real heterogeneity in firm performance, supporting our identification strategy.


\subsection{Treatment Effects on Owner vs Manager Variables}

{
\def\sym#1{\ifmmode^{#1}\else\(^{#1}\)\fi}
\begin{tabular}{l*{3}{c}}
\hline\hline
                    &\multicolumn{1}{c}{(1)}&\multicolumn{1}{c}{(2)}&\multicolumn{1}{c}{(3)}\\
                    &\multicolumn{1}{c}{Fixed assets (log)}&\multicolumn{1}{c}{Has intangible assets}&\multicolumn{1}{c}{Foreign owned}\\
\hline
Better CEO          &       0.171         &       0.063\sym{*}  &       0.017         \\
                    &     (0.131)         &     (0.035)         &     (0.021)         \\
\hline
Observations        &        7801         &        7801         &        7801         \\
\hline\hline
\end{tabular}
}


{
\def\sym#1{\ifmmode^{#1}\else\(^{#1}\)\fi}
\begin{tabular}{l*{2}{c}}
\hline\hline
                    &\multicolumn{1}{c}{(1)}&\multicolumn{1}{c}{(2)}\\
                    &\multicolumn{1}{c}{Wagebill (log)}&\multicolumn{1}{c}{Materials (log)}\\
\hline
Better CEO          &       0.251\sym{***}&       0.335\sym{***}\\
                    &     (0.045)         &     (0.047)         \\
\hline
Observations        &       58025         &       58025         \\
\hline\hline
\end{tabular}
}



\end{document}