\documentclass[11pt,a4paper]{article}
\usepackage[utf8]{inputenc}
\usepackage[T1]{fontenc}
\usepackage{amsmath,amsfonts,amssymb}
\usepackage{apacite}
\usepackage{natbib}
\usepackage{graphicx}
\usepackage{booktabs}
\usepackage{threeparttable}
\usepackage{url}
\usepackage{hyperref}
\usepackage[margin=2.5cm]{geometry}
\usepackage{setspace}
\onehalfspacing

\newcommand{\Var}{\text{Var}}
\newcommand{\Cov}{\text{Cov}}

% Define \sym command for significance stars from esttab
\newcommand{\sym}[1]{{#1}}

\title{Estimating the Value of CEOs in Privately Held Businesses\thanks{Project no. 144193 has been implemented with the support provided by the Ministry of Culture and Innovation of Hungary from the National Research, Development and Innovation Fund, financed under the KKP\_22 funding scheme. This project was funded by the European Research Council (ERC Advanced Grant agreement number 101097789). The views expressed in this research are those of the authors and do not necessarily reflect the official view of the European Union or the European Research Council. \emph{Author contributions:} Conceptualization and study design: Koren, Orbán and Telegdy. Data curation, integration and quality assurance: Szilágyi and Vereckei. Statistical analysis: Koren and Telegdy. Writing the original draft: Koren. Review and editing: Koren, Orbán, Szilágyi, Telegdy and Vereckei. \emph{AI disclosure:} Claude Sonnet 4 was used to write and edit the research code and to format the manuscript (such as editing tables, figures, references, creating summaries). All code and text generated by AI tools were reviewed and edited by the authors. All authors have read and agreed to the published version of the manuscript. \emph{Data availability statement:} The data underlying this article cannot be shared publicly due to privacy and licensing restrictions. The replication package is available at \url{https://github.com/korenmiklos/ceo-value}.}}

\author{Miklós Koren\thanks{Central European University, HUN-REN Centre for Economic and Regional Studies, CEPR and CESifo. E-mail: korenm@ceu.edu} \\
        Krisztina Orbán\thanks{Monash University.} \\
        Bálint Szilágyi\thanks{HUN-REN Centre for Economic and Regional Studies.} \\
        Álmos Telegdy\thanks{Corvinus University of Budapest.} \\
        András Vereckei\thanks{HUN-REN Centre for Economic and Regional Studies, Institute of Economics.}}

\date{\today}

\begin{document}

\maketitle

\begin{abstract}
We develop a model-based approach to measure CEO value in privately held businesses, holding fixed inputs chosen by owners while substituting out variable inputs optimized by managers. Using the universe of Hungarian firms and their CEO networks over three decades (1992-2022), we estimate skill differences that translate into measurable productivity impacts. Most importantly, we develop a novel placebo-controlled event study design that reveals [[percent noise in fixed effects]] percent of typical fixed-effect estimates reflect noise rather than true CEO effects. The naive comparison shows firms hiring better CEOs outperform those hiring worse CEOs by [[raw CEO effect]] percent, but placebo transitions---randomly assigned fake CEO changes excluding actual transition periods---generate a [[placebo effect]] percent spurious effect. The true causal impact is thus [[true CEO effect]] percent: statistically significant and economically meaningful, but only [[true effect as percent of raw]] percent of the raw correlation. This methodology addresses a fundamental identification challenge in the managerial effects literature and suggests new approaches for using CEO quality measures in empirical work.
\end{abstract}

\textbf{Keywords:} CEO value, private firms, productivity

\textbf{JEL Classification:} [To be added]

\newpage

\section{Introduction}

[To be written]

\section{Related Literature}

[To be written]

\section{Conceptual Framework}

We develop a framework to measure CEO value in privately held firms where owners retain control over strategic decisions while delegating operational choices to managers. This division of decision rights, common in private businesses, has important implications for identifying manager effects.

\subsection{Production and Decision Rights}

Firms produce output using a Cobb-Douglas production function with both fixed and variable inputs. The production function for firm $i$ with manager $m$ at time $t$ is:
\begin{equation}\label{eq:production}
Q_{imt} = A_i Z_{m} \varepsilon_{it} K_{it}^\alpha L_{imt}^{\beta} M_{imt}^{\gamma}
\end{equation}
where $A_i$ represents time-invariant organizational capital (location, brand value, customer relationships), $Z_m$ captures manager skill, $\varepsilon_{it}$ is residual productivity, $K_{it}$ is physical capital, $L_{imt}$ is labor input, and $M_{imt}$ is intermediate input usage. 

In standard production function estimation, these three components would be combined into a single measure: $\Omega_{it} = A_i Z_m \varepsilon_{it}$, commonly called total factor productivity (TFP). Our framework decomposes TFP into firm-specific advantages ($A_i$), manager-specific skill ($Z_m$), and residual productivity shocks ($\varepsilon_{it}$) to identify the distinct contribution of managers. 

The key institutional feature we model is the separation of decision rights. Owners control physical capital investment ($K_{it}$) and organizational assets ($A_i$), including location choices, brand development, and CEO hiring decisions. Managers control labor hiring ($L_{imt}$), input purchasing ($M_{imt}$), and day-to-day operations. This separation reflects the reality of private businesses where owners maintain direct involvement in strategic decisions while delegating operational management.

With $\chi := 1 - \beta - \gamma$, the production function exhibits decreasing returns to scale in variable inputs ($\beta + \gamma < 1$), which pins down optimal firm size even under perfect competition. Fixed inputs $A_i$ and $Z_m$ create firm-specific and manager-specific advantages that generate economic rents.

\subsection{Optimal Input Choices and Revenue}

Managers maximize profit by choosing variable inputs optimally given the owner's fixed choices. Under sector-specific output price $P_{st}$, wage rate $W_{st}$, and material price $\varrho_{st}$, the first-order conditions yield closed-form solutions for optimal input demands. Substituting these back into the revenue function gives:
\begin{equation}\label{eq:revenue}
R_{imst} = (P_{st}A_i Z_m \varepsilon_{it})^{1/\chi}
K_{it}^{\alpha/\chi}
W_{st}^{-\beta/\chi}
\varrho_{st}^{-\gamma/\chi}
(1-\chi)^{(1-\chi)/\chi}
\end{equation}

Revenue increases in manager skill $Z_m$, organizational capital $A_i$, and physical capital $K_{it}$, while decreasing in input prices. The elasticity of revenue with respect to manager skill is $1/\chi > 1$, reflecting the leverage effect: better managers can scale up operations by hiring more variable inputs, amplifying their productivity advantage.

\subsection{Surplus and Manager Value}

The surplus accruing to fixed factors—what owners and managers ultimately care about—equals revenue minus payments to variable inputs:
\begin{equation}\label{eq:surplus}
S_{imst} = R_{imst} - W_{st}L_{imt} - \varrho_{st}M_{imt} = \chi R_{imst}
\end{equation}

Under Cobb-Douglas technology, surplus is a constant fraction $\chi$ of revenue. This proportionality means we can work directly with revenue throughout our analysis, simplifying estimation while preserving all economic insights. Taking logarithms of the revenue equation:
\begin{equation}\label{eq:log_revenue}
r_{imst} = C+\frac{\alpha}{\chi} k_{it} + \frac{1}{\chi} z_{m} + \frac{1}{\chi} a_i + \frac{1}{\chi} p_{st} + \frac{1}{\chi}\epsilon_{it} 
- \frac{\beta}{\chi} w_{st} - \frac{\gamma}{\chi} \rho_{st}
\end{equation}
where lowercase letters denote logarithms (e.g., $\epsilon_{it} = \log \varepsilon_{it}$) and $C$ is a constant.

The value of replacing manager $m$ with manager $m'$ at the same firm is:
\begin{equation}\label{eq:manager_value}
r_{im'st} - r_{imst} = \frac{1}{\chi}(z_{m'} - z_{m})
\end{equation}

Manager value equals the skill difference scaled by $1/\chi$. This scaling reflects the leverage effect: a 1\% increase in manager skill generates a $(1/\chi)\%$ increase in revenue.

\subsection{Empirical Specification}

To estimate manager effects from observational data, we substitute unobserved prices and organizational capital with fixed effects:
\begin{equation}\label{eq:empirical}
r_{imst} = \frac{\alpha}{\chi} k_{it} + \frac{1}{\chi}z_m + \lambda_i + \mu_{st} + \tilde{\epsilon}_{it}
\end{equation}
where $\lambda_i = a_i/\chi$ captures time-invariant firm characteristics, $\mu_{st}$ absorbs sector-time variation in prices, and $\tilde{\epsilon}_{it} = \epsilon_{it}/\chi$ is rescaled residual productivity.

The key identifying assumption is that residual productivity $\tilde{\epsilon}_{it}$ is uncorrelated with manager assignment conditional on firm and sector-time fixed effects. This allows manager skills and physical capital to be arbitrarily correlated with firm quality and market conditions—better firms may hire better managers and invest more. We only require that the residual variation in productivity is orthogonal to manager assignment.

\subsection{Challenges in Estimating Manager Effects}

Three challenges complicate the estimation of manager effects from equation \eqref{eq:empirical}:

First, residual productivity $\tilde{\epsilon}_{it}$ may correlate with manager changes if firms replace CEOs in response to productivity shocks. This reverse causality would bias estimates of manager effects.

Second, manager fixed effects are only identified for the connected set of firms and managers linked through mobility. Effects are measured relative to a reference group within each connected component, limiting comparability across components.

Third, and most importantly, estimated manager effects $\hat{z}_m$ include both true skill and averaged residual productivity: $\hat{z}_m = z_m + \bar{\epsilon}_{im}$ where $\bar{\epsilon}_{im}$ is the average $\tilde{\epsilon}_{it}$ during manager $m$'s tenure at firm $i$. When managers have short tenures, this noise component dominates the signal, making raw fixed effects unreliable measures of true skill.

Our placebo-controlled approach addresses these challenges by explicitly measuring and removing the noise component from estimated manager effects.

\section{Corporate Data from Hungary}

Hungary provides an ideal setting for studying CEO effects in private firms. The country offers complete administrative data coverage for all incorporated businesses with mandatory CEO registration, spanning over 30 years from the transition economy of the 1990s through EU accession in 2004 to the present. This comprehensive coverage enables us to track CEO careers across firms and construct the mobility networks necessary for identification.

\subsection{Data Sources}

Our analysis combines two administrative datasets. The firm registry, maintained by Hungarian corporate courts, contains legally mandated records on all company representatives—individuals authorized to act on behalf of firms in legal and business matters. These records include CEOs and other executives with signatory rights, tracked through a temporal database where each entry reflects representation status over specific time intervals. Updates occur not only when positions change but also when personal identifiers are modified or reporting standards evolve. The registry provides names, addresses, dates of birth (from 2010), and mother's names (from 1999), though numerical identifiers only exist from 2013 onward.

The balance sheet dataset contains annual financial reports for essentially all Hungarian firms required to file statements. This includes sales revenue, export revenue, employment counts, tangible and intangible assets, raw material and intermediate input costs, personnel expenses, and indicators for state and foreign ownership. The two datasets together cover [[total firms]] firms over [[total years]] years, yielding [[total firm-years]] firm-year observations before sample restrictions.

\subsection{Entity Resolution and CEO Identification}

Constructing a panel of CEOs requires resolving two fundamental questions: what constitutes a firm and who qualifies as a CEO. For firms, we track legal entities through time using tax identifiers, which remain relatively stable despite occasional reuse (approximately [[tax ID reuse percent]]\% of cases). Mergers and acquisitions create new entities in our framework—we follow individual legal entities rather than economic conglomerates.

Identifying individual CEOs poses greater challenges. Before 2013, no numerical identifiers existed, requiring entity resolution based on names, addresses, mother's names, and birthdates. We link records across these dimensions to create unique person identifiers, enabling tracking across firms and over time. The quality of matching improves substantially after 1999 (mother's names) and 2010 (birthdates), though even the 1990s data achieves reasonable coverage through careful name and address matching.

CEO identification within firms requires additional heuristics since job titles are inconsistently recorded. When explicit "managing director" titles exist (approximately [[managing director percent]]\% of cases), we use them directly. For remaining cases, we assume all representatives are CEOs if three or fewer exist at the firm. When more than three representatives are present, we assign CEO status based on continuity with previously identified CEOs. Time spans between appointments are often unclosed or non-contiguous, requiring imputation based on sequential information and assuming representatives remain active if their tenure includes June 21 of each year.

\subsection{Sample Restrictions}

We apply several restrictions to create a sample suitable for productivity analysis. First, we exclude mining and finance sectors due to their specialized accounting frameworks and regulatory environments. Second, we drop firms ever having more than [[max simultaneous CEOs]] simultaneous CEOs (removing [[dropped for multiple CEOs]] observations) to avoid complex governance structures that complicate identification. Third, we exclude firms with more than [[max CEO changes]] CEO changes over the sample period ([[dropped for excess changes]] observations) to reduce noise from potentially misclassified transitions. Fourth, we remove all state-owned enterprises, as their objectives and constraints differ fundamentally from private businesses.

Most importantly, we restrict attention to firms that at some point employ at least [[min employees]] workers. This filter removes [[percent dropped for size]] of observations but eliminates shell companies, tax optimization vehicles, and self-employment arrangements masquerading as corporations. The remaining firms represent genuine businesses with meaningful economic activity where management quality plausibly affects performance.

\subsection{CEO and Firm Characteristics}

The resulting sample contains distinctive patterns of CEO demographics and tenure. Among CEOs in our analysis sample, [[percent Hungarian names]]\% have Hungarian names (verified algorithmically), with only [[percent foreign names]]\% foreign. Gender identification among Hungarian-named CEOs reveals [[percent male]]\% male representation. Remarkably, [[percent founders]]\% of CEOs were present at their firm's founding, highlighting the prevalence of founder-managers in private businesses.

CEO mobility creates networks essential for fixed effects identification. Approximately [[percent multiple firms]]\% of CEOs manage multiple firms during the sample period, generating connections between firms. The largest connected component of the bipartite firm-CEO network contains [[connected component size]] managers and a comparable number of firms—roughly [[percent in component]]\% of all managers but representing a substantial share of economic activity. This connected component enables comparison of CEO effects within a common reference frame, though raises questions about representativeness we address in robustness checks.

\subsection{CEO Turnover Patterns}

CEO tenure exhibits substantial variation across firms. While [[percent single CEO firms]]\% of firms retain the same CEO throughout their observed lifetime, the remaining firms experience leadership transitions that enable identification. Among firm-years with CEO information, [[percent single CEO firm-years]]\% have single CEOs, [[percent two CEO firm-years]]\% have two, and [[percent multiple CEO firm-years]]\% have more (before our sample restrictions). CEO spell lengths follow approximately an exponential distribution with a [[annual hazard rate]]\% annual hazard rate, implying typical tenures of [[typical tenure range]] years.

To validate our placebo methodology, we verify that synthetically generated CEO transitions match actual patterns. Taking firms with stable leadership (same CEO for [[min stable tenure]]+ years), we randomly assign placebo transitions using the empirical hazard function. The resulting spell length distribution closely mirrors actual CEO changes: [[percent one year spells]]\% last one year, [[percent two year spells]]\% two years, declining thereafter. This similarity confirms that our placebo transitions capture realistic turnover dynamics while maintaining the crucial property that no actual skill change occurs.

\section{Estimation}

Our estimation proceeds in four steps: measuring the surplus share, estimating the revenue function, recovering manager fixed effects, and validating causality through event studies. Each step builds toward separating true CEO effects from the substantial noise that contaminates raw estimates.

\subsection{Step 1: Measuring the Surplus Share}

The parameter $\chi$—the share of surplus in revenue—determines how manager skill translates into firm performance. Under Cobb-Douglas technology, this share equals one minus the combined revenue shares of labor and materials. Following \citet{Gandhi2020-nu}, we measure $\chi$ directly from the data as:
\begin{equation}
\hat{\chi}_s = 1 - \frac{\sum_{i \in s}(W_{st}L_{it} + \varrho_{st}M_{it})}{\sum_{i \in s} R_{it}}
\end{equation}
where the summation runs over firms in sector $s$.

This approach yields sector-specific estimates of $\chi$. The $1/\chi$ scaling in our framework creates a leverage effect where manager skill is amplified in its impact on firm performance.

\subsection{Step 2: Estimating the Revenue Function}

With $\hat{\chi}$ measured, we estimate the revenue function to recover the capital elasticity and control for observable factors. Using lowercase letters for logarithms, the estimating equation is:
\begin{equation}
r_{imst} = \frac{\alpha}{\chi} k_{it} + \frac{1}{\chi}z_m + \lambda_i + \mu_{st} + \tilde{\epsilon}_{it}
\end{equation}
where $r_{imst} = \log R_{imst}$ is log revenue, $k_{it} = \log K_{it}$ is log capital, $\lambda_i$ captures firm fixed effects, $\mu_{st}$ are sector-year fixed effects, and $\tilde{\epsilon}_{it} = \epsilon_{it}/\chi$ is rescaled residual productivity.

In practice, we allow firm productivity to vary across CEOs by including firm-CEO fixed effects $\lambda_{im}$ rather than separate firm and manager effects. This captures the joint contribution of organizational capital and manager skill for each firm-manager pair. We also include controls for firm age, intangible asset presence, and foreign ownership, though these barely affect the capital coefficient.

The key assumptions are: (1) all firms within a sector face the same prices, (2) residual productivity $\tilde{\epsilon}_{it}$ is uncorrelated with owner and manager choices, and (3) owner and manager choices can be arbitrarily correlated with each other. Assumption (2) is critical—we do not require random manager assignment, only that the residual productivity after controlling for observables is orthogonal to CEO changes.

We estimate using OLS with high-dimensional fixed effects via \texttt{reghdfe} \citep{reghdfe}. The coefficient on log capital is $\hat{\alpha}/\hat{\chi}$, which we multiply by $\hat{\chi}$ to recover $\hat{\alpha}$.

\subsection{Step 3: Recovering Manager Fixed Effects}

After estimating the revenue function, we compute log total factor productivity by removing the contribution of capital from revenue:
\begin{equation}
\omega_{imst} = \hat{\chi} r_{imst} - \hat{\alpha} k_{it} - \hat{\mu}_{st} = z_m + a_i + \epsilon_{it}
\end{equation}

This measure of log TFP contains manager skill, firm effects, and residual productivity. Recall that in standard production function estimation, this entire term would be treated as a single TFP measure. Our decomposition separates the manager contribution from other sources of productivity.

To isolate manager effects, we remove the firm fixed effect by subtracting the firm average:
\begin{equation}
\Delta\omega_{imt} = \Delta z_{m_{it}} + \Delta\epsilon_{it}
\end{equation}
where $\Delta x_{it} := x_{it} - \frac{1}{N_i}\sum_{\tau} x_{i\tau}$ denotes the within-firm deviation.

When a firm changes CEOs, the change in log TFP captures both the true skill difference and accumulated noise. The noise component—the average of residual productivity shocks during each manager's tenure—can dominate the signal when tenures are short.

\subsubsection{Two-Way Fixed Effects in the Connected Component}

For broader comparisons across firms, we estimate a two-way fixed effects model:
\begin{equation}
\omega_{imst} = a_i + z_m + \epsilon_{it}
\end{equation}

This system is only identified within connected components—groups of firms and managers linked through mobility. Two managers can be compared if they worked at the same firm or connect through a chain of shared employment relationships. Our largest connected component contains [[connected component size]] managers, enabling meaningful comparisons within this network. Following \citet{Abowd1999Econometrica}, we normalize one manager's effect to zero and estimate via within-transformation.

\subsubsection{Three Identification Challenges}

First, residual productivity trends may correlate with manager changes (reverse causality). Firms might replace CEOs when $\epsilon_{it}$ declines, creating spurious negative effects for incoming managers. We do not need random mobility—manager quality can correlate with firm quality—but we require that residual productivity shocks are orthogonal to CEO transitions.

Second, firm and manager effects are only interpretable within their connected component. Effects measure relative differences from a baseline group, limiting cross-component comparisons.

Third, and most critically, estimated fixed effects contain substantial small-sample noise. With short CEO tenures, the averaged residual productivity $\bar{\epsilon}_{im}$ often overwhelms true skill differences. Instrumental variable approaches that rely on exogenous CEO removal (death, hospitalization) address reverse causality but exacerbate the small-sample problem by further restricting the data.

\subsection{Step 4: Placebo-Controlled Event Studies}

Our solution to the noise problem leverages a simple insight: when CEOs do not change, we still observe variation in log TFP driven purely by noise:
\begin{equation}
\Delta\omega_{imt} = \Delta\epsilon_{it}
\end{equation}

By applying the same estimation procedure to "non-changes," we can measure and filter out the noise component.

\subsubsection{Constructing Placebo Transitions}

We construct placebo CEO transitions in three steps. First, we estimate the time-varying hazard of actual CEO changes. Second, we identify firms with long CEO tenures (7+ years) where no actual change occurs. Third, we randomly assign placebo transitions to these stable firms using the estimated hazard function.

For instance, a firm with the same CEO from 1992-2000 might receive a placebo transition in 1996, artificially splitting the tenure into two pseudo-CEOs. The timing follows the empirical hazard, ensuring the mechanical noise properties—averaging residual productivity over varying tenure lengths—match between actual and placebo groups.

\subsubsection{Event Study Design}

We implement the event study around CEO transitions at time $g$, comparing actual changes (treatment) to placebo changes (control):
\begin{equation}
\omega_{imt} = a_i + \gamma_{t-g} + \epsilon_{it}
\end{equation}

The coefficients $\gamma_{t-g}$ capture the evolution of log TFP in event time, where $t-g \in \{-4, -3, -2, -1, 0, 1, 2, 3\}$ and we normalize $\gamma_{-1} = 0$.

Because both groups receive a "treatment" (actual or placebo), standard difference-in-differences fails. We adapt the \citet{Callaway2021JoLE} estimator for two treatment types using the \texttt{xt2treatments} package \citep{Koren2024xt2treatments}. The key innovation is precisely aligning transitions in event time—both actual and placebo changes occur at $t=0$—enabling clean comparison of dynamics.

\subsubsection{Separating Good from Bad CEOs}

To examine heterogeneity in CEO quality, we split actual transitions based on the incoming CEO's estimated fixed effect $\hat{z}_m$ from Step 3. CEOs with above-median $\hat{z}_m$ are classified as "good," those below as "bad."

This classification is noisy—"good" CEOs are partly lucky (positive $\epsilon$ draws), "bad" CEOs partly unlucky (negative $\epsilon$ draws). But this is precisely the bias our placebo control corrects. When we randomly split stable tenures and classify the resulting pseudo-CEOs as good or bad based on their averaged $\epsilon_{it}$, we generate the same mechanical selection bias without any true quality difference.

\subsubsection{The Placebo Correction in Practice}

The placebo control reveals the extent of noise contamination in estimated manager effects. The decomposition shows that a substantial portion of apparent CEO effects are spurious, explaining why studies using manager fixed effects as explanatory variables often find weak or inconsistent results—they are measuring mostly noise. Our placebo method provides a practical solution for future research: always benchmark against placebo transitions to isolate true effects from mechanical bias.

\end{document}
