\documentclass[11pt,a4paper]{article}
\usepackage[utf8]{inputenc}
\usepackage[T1]{fontenc}
\usepackage{amsmath,amsfonts,amssymb}
\usepackage{natbib}
\usepackage{graphicx}
\usepackage{booktabs}
\usepackage{url}
\usepackage{hyperref}
\usepackage[margin=2.5cm]{geometry}
\usepackage{setspace}
\onehalfspacing

\title{Estimating the Value of CEOs in Privately Held Businesses}

\author{Miklós Koren\thanks{HUN-REN Centre for Economic and Regional Studies, Institute of Economics and Corvinus University of Budapest. Email: koren.miklos@krtk.hun-ren.hu. Project no. 144193 has been implemented with the support provided by the Ministry of Culture and Innovation of Hungary from the National Research, Development and Innovation Fund, financed under the KKP\_22 funding scheme. This project was funded by the European Research Council (ERC Advanced Grant agreement number 101097789). The views expressed in this research are those of the authors and do not necessary reflect the official view of the European Union or the European Research Council.} \\
        Krisztina Orbán\thanks{HUN-REN Centre for Economic and Regional Studies, Institute of Economics.} \\
        Bálint Szilágyi\thanks{HUN-REN Centre for Economic and Regional Studies, Institute of Economics.} \\
        Álmos Telegdy\thanks{HUN-REN Centre for Economic and Regional Studies, Institute of Economics.} \\
        András Vereckei\thanks{HUN-REN Centre for Economic and Regional Studies, Institute of Economics.}}

\date{\today}

\begin{document}

\maketitle

\begin{abstract}
[Abstract to be written]
\end{abstract}

\textbf{Keywords:} CEO value, private firms, productivity

\textbf{JEL Classification:} [To be added]

\newpage

\section{Introduction}

[Introduction to be written]

\section{Data}

\textbf{Main data sources.} Our analysis uses comprehensive administrative data on Hungarian firms during 1992-2022, created by merging balance sheet data with firm registry information. The balance sheet data come from the Mérleg LTS dataset, which contains financial information for essentially all Hungarian firms required to file annual reports. The firm registry data come from the Cégjegyzék LTS dataset, which includes information on firm registration, ownership structure, and executive appointments. Both datasets are distributed by HUN-REN KRTK and originally published by Opten Zrt \citep{merleg2024,cegjegyzek2024}.

The data are proprietary and cannot be made public. Interested researchers should contact opten.hu to obtain a license, with annual license fees in the order of 10,000 EUR and expected contract and data access within 1-2 months. For replication purposes or academic research projects, researchers may contact KRTK Adatbank at \url{https://adatbank.krtk.mta.hu/}.

The balance sheet data include all firms required to file financial statements with Hungarian authorities, covering essentially the entire formal business sector except for the smallest individual entrepreneurs. The dataset contains detailed financial information including sales revenue, export revenue, employment, tangible and intangible assets, raw materials costs, wage bills, personnel expenses, and ownership indicators for state and foreign control.

\textbf{Sample construction.} We construct our analytical sample through several filtering steps. Starting from the full balance sheet dataset covering 1980-2022, we restrict our analysis to 1992-2022 to focus on the post-transition Hungarian economy. This removes 136,141 observations from years prior to 1992, when the economic and institutional environment was fundamentally different.

We then apply a data quality filter by excluding observations with incomplete administrative records, specifically those where \texttt{frame\_id == "only\_originalid"}. This filter removes an additional 195,582 observations, ensuring that our analysis uses only complete firm records with full administrative information available.

After these exclusions, our sample contains 10,214,120 firm-year observations spanning 31 years. Table \ref{tab:sample} shows the temporal distribution of observations in our final sample. The sample exhibits steady growth from 98,780 observations in 1992 to 454,106 in 2022. This expansion reflects both the growth of the Hungarian economy following the transition to a market economy and improvements in administrative data coverage over time.

\begin{table}[htbp]
\centering
\caption{Sample Distribution by Year}
\label{tab:sample}
\begin{tabular}{rr|rr|rr}
\toprule
Year & Observations & Year & Observations & Year & Observations \\
\midrule
1992 & 98,780 & 2002 & 301,278 & 2012 & 397,131 \\
1993 & 122,677 & 2003 & 305,947 & 2013 & 437,692 \\
1994 & 153,639 & 2004 & 319,750 & 2014 & 427,494 \\
1995 & 171,759 & 2005 & 326,905 & 2015 & 433,371 \\
1996 & 198,558 & 2006 & 334,498 & 2016 & 431,041 \\
1997 & 219,751 & 2007 & 345,134 & 2017 & 424,184 \\
1998 & 246,660 & 2008 & 362,920 & 2018 & 425,601 \\
1999 & 256,992 & 2009 & 370,788 & 2019 & 419,883 \\
2000 & 280,386 & 2010 & 384,570 & 2020 & 424,501 \\
2001 & 302,894 & 2011 & 402,636 & 2021 & 432,594 \\
 &  &  &  & 2022 & 454,106 \\
\midrule
\multicolumn{6}{c}{Total: 10,214,120} \\
\bottomrule
\end{tabular}
\footnotesize
Notes: Sample distribution after applying time period restrictions (1992-2022) and data quality filters.
\end{table}

\textbf{Variable construction.} Missing values in financial variables are systematically recoded to zero, following standard practice in administrative data analysis where missing values typically indicate zero rather than unknown values. The extent of missing data varies considerably across variables, reflecting different reporting requirements and business activities. Export data has the highest rate of missing values, with 5,456,815 observations recoded, reflecting that many firms do not engage in export activities. Employment data required recoding for 1,138,791 observations, while sales revenue had relatively few missing values with only 486,197 observations recoded.

For employment, we make an additional adjustment by setting values below one to equal one. This transformation affects 3,655,899 observations and acknowledges that active firms filing administrative reports must have positive employment. Zero or negative employment values likely reflect administrative reporting inconsistencies rather than true zero employment.

We also address issues with wage bill and personnel expense variables, where 3,931,270 and 1,117,283 observations respectively are recoded from missing to zero. For asset variables, tangible assets required recoding for 1,014,331 observations while intangible assets had 4,299,589 missing values recoded, reflecting that many firms do not report significant intangible assets.

\section{Methodology}

[Methodology to be written]

\section{Results}

[Results to be written]

\section{Conclusion}

[Conclusion to be written]

\bibliographystyle{chicago}
\begin{thebibliography}{}

\bibitem[HUN-REN KRTK, 2024a]{cegjegyzek2024} HUN-REN KRTK (distributor). 2024. ``Cégjegyzék LTS [data set]'' Published by Opten Zrt, Budapest. Contributions by CEU MicroData.

\bibitem[HUN-REN KRTK, 2024b]{merleg2024} HUN-REN KRTK (distributor). 2024. ``Mérleg LTS [data set]'' Published by Opten Zrt, Budapest. Contributions by CEU MicroData.

\end{thebibliography}

\end{document}
