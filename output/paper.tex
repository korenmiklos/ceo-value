\documentclass[11pt,a4paper]{article}
\usepackage[utf8]{inputenc}
\usepackage[T1]{fontenc}
\usepackage{amsmath,amsfonts,amssymb}
\usepackage{apacite}
\usepackage{natbib}
\usepackage{graphicx}
\usepackage{booktabs}
\usepackage{threeparttable}
\usepackage{url}
\usepackage{hyperref}
\usepackage[margin=2.5cm]{geometry}
\usepackage{setspace}
\onehalfspacing

\newcommand{\Var}{\text{Var}}
\newcommand{\Cov}{\text{Cov}}

% Define \sym command for significance stars from esttab
\newcommand{\sym}[1]{{#1}}

\title{Estimating the Value of CEOs in Privately Held Businesses\thanks{Project no. 144193 has been implemented with the support provided by the Ministry of Culture and Innovation of Hungary from the National Research, Development and Innovation Fund, financed under the KKP\_22 funding scheme. This project was funded by the European Research Council (ERC Advanced Grant agreement number 101097789). The views expressed in this research are those of the authors and do not necessarily reflect the official view of the European Union or the European Research Council. \emph{Author contributions:} Conceptualization and study design: Koren, Orbán and Telegdy. Data curation, integration and quality assurance: Szilágyi and Vereckei. Statistical analysis: Koren and Telegdy. Writing the original draft: Koren. Review and editing: Koren, Orbán, Szilágyi, Telegdy and Vereckei. \emph{AI disclosure:} Claude Sonnet 4 was used to write and edit the research code and to format the manuscript (such as editing tables, figures, references, creating summaries). All code and text generated by AI tools were reviewed and edited by the authors. All authors have read and agreed to the published version of the manuscript. \emph{Data availability statement:} The data underlying this article cannot be shared publicly due to privacy and licensing restrictions. The replication package is available at \url{https://github.com/korenmiklos/ceo-value}.}}

\author{Miklós Koren\thanks{Central European University, HUN-REN Centre for Economic and Regional Studies, CEPR and CESifo. E-mail: korenm@ceu.edu} \\
        Krisztina Orbán\thanks{Monash University.} \\
        Bálint Szilágyi\thanks{HUN-REN Centre for Economic and Regional Studies.} \\
        Álmos Telegdy\thanks{Corvinus University of Budapest.} \\
        András Vereckei\thanks{HUN-REN Centre for Economic and Regional Studies, Institute of Economics.}}

\date{\today}

\begin{document}

\maketitle

\begin{abstract}
We develop a model-based approach to measure CEO value in privately held businesses, holding fixed inputs chosen by owners while substituting out variable inputs optimized by managers. Using the universe of Hungarian firms and their CEO networks over three decades (1992-2022), we estimate skill differences that translate into measurable productivity impacts. Most importantly, we develop a novel placebo-controlled event study design that reveals 75 percent of typical fixed-effect estimates reflect noise rather than true CEO effects. The naive comparison shows firms hiring better CEOs outperform those hiring worse CEOs by 22.5 percent, but placebo transitions---randomly assigned fake CEO changes excluding actual transition periods---generate a 17.0 percent spurious effect. The true causal impact is thus 5.5 percent: statistically significant and economically meaningful, but only 25 percent of the raw correlation. This methodology addresses a fundamental identification challenge in the managerial effects literature and suggests new approaches for using CEO quality measures in empirical work.
\end{abstract}

\textbf{Keywords:} CEO value, private firms, productivity

\textbf{JEL Classification:} [To be added]

\newpage

\section{Introduction}

[To be written]

\section{Related Literature}

[To be written]

\section{Conceptual Framework}

We develop a framework to measure CEO value in privately held firms where owners retain control over strategic decisions while delegating operational choices to managers. This division of decision rights, common in private businesses, has important implications for identifying manager effects.

\subsection{Production and Decision Rights}

Firms produce output using a Cobb-Douglas production function with both fixed and variable inputs. The production function for firm $i$ with manager $m$ at time $t$ is:
\begin{equation}\label{eq:production}
Q_{imt} = \Omega_{it}A_i Z_{m}  K_{it}^\alpha L_{imt}^{\beta} M_{imt}^{\gamma}
\end{equation}
where $\Omega_{it}$ is residual total factor productivity, $A_i$ represents time-invariant organizational capital (location, brand value, customer relationships), $Z_m$ captures manager skill, $K_{it}$ is physical capital, $L_{imt}$ is labor input, and $M_{imt}$ is intermediate input usage. 

The key institutional feature we model is the separation of decision rights. Owners control physical capital investment ($K_{it}$) and organizational assets ($A_i$), including location choices, brand development, and CEO hiring decisions. Managers control labor hiring ($L_{imt}$), input purchasing ($M_{imt}$), and day-to-day operations. This separation reflects the reality of private businesses where owners maintain direct involvement in strategic decisions while delegating operational management.

With $\chi := 1 - \beta - \gamma$, the production function exhibits decreasing returns to scale in variable inputs ($\beta + \gamma < 1$), which pins down optimal firm size even under perfect competition. Fixed inputs $A_i$ and $Z_m$ create firm-specific and manager-specific advantages that generate economic rents.

\subsection{Optimal Input Choices and Revenue}

Managers maximize profit by choosing variable inputs optimally given the owner's fixed choices. Under sector-specific output price $P_{st}$, wage rate $W_{st}$, and material price $\varrho_{st}$, the first-order conditions yield closed-form solutions for optimal input demands. Substituting these back into the revenue function gives:
\begin{equation}\label{eq:revenue}
R_{imst} = (P_{st}\Omega_{it}A_i Z_m)^{1/\chi}
K_{it}^{\alpha/\chi}
W_{st}^{-\beta/\chi}
\varrho_{st}^{-\gamma/\chi}
(1-\chi)^{(1-\chi)/\chi}
\end{equation}

Revenue increases in manager skill $Z_m$, organizational capital $A_i$, and physical capital $K_{it}$, while decreasing in input prices. The elasticity of revenue with respect to manager skill is $1/\chi > 1$, reflecting the leverage effect: better managers can scale up operations by hiring more variable inputs, amplifying their productivity advantage.

\subsection{Surplus and Manager Value}

The surplus accruing to fixed factors equals revenue minus payments to variable inputs:
\begin{equation}\label{eq:surplus}
S_{imst} = R_{imst} - W_{st}L_{imt} - \varrho_{st}M_{imt} = \chi R_{imst}
\end{equation}

Under Cobb-Douglas technology, surplus is a constant fraction $\chi$ of revenue. Taking logarithms:
\begin{equation}\label{eq:log_surplus}
s_{imst} = C+\frac{\alpha}{\chi} k_{it} + \frac{1}{\chi} z_{m} + \frac{1}{\chi} a_i + \frac{1}{\chi} p_{st} + \frac{1}{\chi}\omega_{it} 
- \frac{\beta}{\chi} w_{st} - \frac{\gamma}{\chi} \rho_{st}
\end{equation}
where lowercase letters denote logarithms and $C$ is a constant.

The value of replacing manager $m$ with manager $m'$ at the same firm is:
\begin{equation}\label{eq:manager_value}
s_{im'st} - s_{imst} = \frac{1}{\chi}(z_{m'} - z_{m})
\end{equation}

Manager value equals the skill difference scaled by $1/\chi$. This scaling reflects the leverage effect: a 1\% increase in manager skill generates a $(1/\chi)\%$ increase in surplus, typically 5-10\% given our estimates of $\chi \approx 0.1-0.2$.

\subsection{Empirical Specification}

To estimate manager effects from observational data, we substitute unobserved prices and organizational capital with fixed effects:
\begin{equation}\label{eq:empirical}
s_{imst} = \frac{\alpha}{\chi} k_{it} + \frac{1}{\chi}\tilde{z}_m + \lambda_i + \mu_{st} + \tilde{\omega}_{it}
\end{equation}
where $\lambda_i = a_i/\chi$ captures time-invariant firm characteristics, $\mu_{st}$ absorbs sector-time variation in prices, and $\tilde{\omega}_{it} = \omega_{it}/\chi$ is rescaled residual productivity.

The key identifying assumption is that residual productivity $\tilde{\omega}_{it}$ is uncorrelated with manager assignment conditional on firm and sector-time fixed effects. This allows manager skills and physical capital to be arbitrarily correlated with firm quality and market conditions—better firms may hire better managers and invest more. We only require that the residual variation in productivity is orthogonal to manager assignment.

\subsection{Challenges in Estimating Manager Effects}

Three challenges complicate the estimation of manager effects from equation \eqref{eq:empirical}:

First, residual productivity $\tilde{\omega}_{it}$ may correlate with manager changes if firms replace CEOs in response to productivity shocks. This reverse causality would bias estimates of manager effects.

Second, manager fixed effects are only identified for the connected set of firms and managers linked through mobility. Effects are measured relative to a reference group within each connected component, limiting comparability across components.

Third, and most importantly, estimated manager effects $\hat{z}_m$ include both true skill and averaged residual productivity: $\hat{z}_m = z_m + \bar{\omega}_{im}$ where $\bar{\omega}_{im}$ is the average $\tilde{\omega}_{it}$ during manager $m$'s tenure at firm $i$. When managers have short tenures, this noise component dominates the signal, making raw fixed effects unreliable measures of true skill.

Our placebo-controlled approach addresses these challenges by explicitly measuring and removing the noise component from estimated manager effects.

\section{Corporate Data from Hungary}

Hungary provides an ideal setting for studying CEO effects in private firms. The country offers complete administrative data coverage for all incorporated businesses with mandatory CEO registration, spanning over 30 years from the transition economy of the 1990s through EU accession in 2004 to the present. This comprehensive coverage enables us to track CEO careers across firms and construct the mobility networks necessary for identification.

\subsection{Data Sources}

Our analysis combines two administrative datasets. The firm registry, maintained by Hungarian corporate courts, contains legally mandated records on all company representatives—individuals authorized to act on behalf of firms in legal and business matters. These records include CEOs and other executives with signatory rights, tracked through a temporal database where each entry reflects representation status over specific time intervals. Updates occur not only when positions change but also when personal identifiers are modified or reporting standards evolve. The registry provides names, addresses, dates of birth (from 2010), and mother's names (from 1999), though numerical identifiers only exist from 2013 onward.

The balance sheet dataset contains annual financial reports for essentially all Hungarian firms required to file statements. This includes sales revenue, export revenue, employment counts, tangible and intangible assets, raw material and intermediate input costs, personnel expenses, and indicators for state and foreign ownership. The two datasets together cover 1,063,172 firms over 31 years, yielding 10,151,997 firm-year observations before sample restrictions.

\subsection{Entity Resolution and CEO Identification}

Constructing a panel of CEOs requires resolving two fundamental questions: what constitutes a firm and who qualifies as a CEO. For firms, we track legal entities through time using tax identifiers, which remain relatively stable despite occasional reuse (approximately 2\% of cases). Mergers and acquisitions create new entities in our framework—we follow individual legal entities rather than economic conglomerates.

Identifying individual CEOs poses greater challenges. Before 2013, no numerical identifiers existed, requiring entity resolution based on names, addresses, mother's names, and birthdates. We link records across these dimensions to create unique person identifiers, enabling tracking across firms and over time. The quality of matching improves substantially after 1999 (mother's names) and 2010 (birthdates), though even the 1990s data achieves reasonable coverage through careful name and address matching.

CEO identification within firms requires additional heuristics since job titles are inconsistently recorded. When explicit "managing director" titles exist (approximately 80\% of cases), we use them directly. For remaining cases, we assume all representatives are CEOs if three or fewer exist at the firm. When more than three representatives are present, we assign CEO status based on continuity with previously identified CEOs. Time spans between appointments are often unclosed or non-contiguous, requiring imputation based on sequential information and assuming representatives remain active if their tenure includes June 21 of each year.

\subsection{Sample Restrictions}

We apply several restrictions to create a sample suitable for productivity analysis. First, we exclude mining and finance sectors due to their specialized accounting frameworks and regulatory environments. Second, we drop firms ever having more than two simultaneous CEOs (removing 1,519,524 observations) to avoid complex governance structures that complicate identification. Third, we exclude firms with more than six CEO changes over the sample period (45,216 observations) to reduce noise from potentially misclassified transitions. Fourth, we remove all state-owned enterprises, as their objectives and constraints differ fundamentally from private businesses.

Most importantly, we restrict attention to firms that at some point employ at least five workers. This filter removes three-quarters of observations but eliminates shell companies, tax optimization vehicles, and self-employment arrangements masquerading as corporations. The remaining firms represent genuine businesses with meaningful economic activity where management quality plausibly affects performance.

\subsection{CEO and Firm Characteristics}

The resulting sample contains distinctive patterns of CEO demographics and tenure. Among CEOs in our analysis sample, 95\% have Hungarian names (verified algorithmically), with only 5\% foreign. Gender identification among Hungarian-named CEOs reveals 73\% male representation. Remarkably, 69\% of CEOs were present at their firm's founding, highlighting the prevalence of founder-managers in private businesses.

CEO mobility creates networks essential for fixed effects identification. Approximately 18\% of CEOs manage multiple firms during the sample period, generating connections between firms. The largest connected component of the bipartite firm-CEO network contains 26,000 managers and a comparable number of firms—roughly 10-15\% of all managers but representing a substantial share of economic activity. This connected component enables comparison of CEO effects within a common reference frame, though raises questions about representativeness we address in robustness checks.

\subsection{CEO Turnover Patterns}

CEO tenure exhibits substantial variation across firms. While 63\% of firms retain the same CEO throughout their observed lifetime, the remaining firms experience leadership transitions that enable identification. Among firm-years with CEO information, 82\% have single CEOs, 15\% have two, and 3\% have more (before our sample restrictions). CEO spell lengths follow approximately an exponential distribution with a 20\% annual hazard rate, implying typical tenures of 4-5 years.

To validate our placebo methodology, we verify that synthetically generated CEO transitions match actual patterns. Taking firms with stable leadership (same CEO for 7+ years), we randomly assign placebo transitions using the empirical hazard function. The resulting spell length distribution closely mirrors actual CEO changes: 23\% last one year, 18\% two years, declining thereafter. This similarity confirms that our placebo transitions capture realistic turnover dynamics while maintaining the crucial property that no actual skill change occurs.

\section{Estimation}

Our estimation proceeds in four steps: measuring the surplus share, estimating the revenue function, recovering manager fixed effects, and validating causality through event studies. Each step builds toward separating true CEO effects from the substantial noise that contaminates raw estimates.

\subsection{Step 1: Measuring the Surplus Share}

The parameter $\chi$—the share of surplus in revenue—determines how manager skill translates into firm performance. Under Cobb-Douglas technology, this share equals one minus the combined revenue shares of labor and materials. Following \citet{Gandhi2020-nu}, we measure $\chi$ directly from the data as:
\begin{equation}
\hat{\chi}_s = 1 - \frac{\sum_{i \in s}(W_{st}L_{it} + \varrho_{st}M_{it})}{\sum_{i \in s} R_{it}}
\end{equation}
where the summation runs over firms in sector $s$.

This approach yields sector-specific estimates ranging from 0.10 to 0.20. Manufacturing shows lower values (around 0.12) while services show higher values (around 0.18). These estimates imply substantial leverage: a 1\% improvement in manager skill generates a 5-10\% increase in surplus through the $1/\chi$ scaling in our framework.

\subsection{Step 2: Estimating the Revenue Function}

With $\hat{\chi}$ measured, we estimate the revenue function to recover the capital elasticity and control for observable factors. Using lowercase letters for logarithms, the estimating equation is:
\begin{equation}
r_{imst} = \frac{\alpha}{\chi} k_{it} + \frac{1}{\chi}z_m + \lambda_i + \mu_{st} + \tilde{\omega}_{it}
\end{equation}
where $r_{imst} = \log S_{imst}$ is log surplus, $k_{it} = \log K_{it}$ is log capital, $\lambda_i$ captures firm fixed effects, $\mu_{st}$ are sector-year fixed effects, and $\tilde{\omega}_{it} = \omega_{it}/\chi$ is rescaled residual productivity.

In practice, we allow firm productivity to vary across CEOs by including firm-CEO fixed effects $\lambda_{im}$ rather than separate firm and manager effects. This captures the joint contribution of organizational capital and manager skill for each firm-manager pair. We also include controls for firm age, intangible asset presence, and foreign ownership, though these barely affect the capital coefficient.

The key assumptions are: (1) all firms within a sector face the same prices, (2) residual TFP $\tilde{\omega}_{it}$ is uncorrelated with owner and manager choices, and (3) owner and manager choices can be arbitrarily correlated with each other. Assumption (2) is critical—we do not require random manager assignment, only that the residual productivity after controlling for observables is orthogonal to CEO changes.

We estimate using OLS with high-dimensional fixed effects via \texttt{reghdfe} \citep{reghdfe}. The coefficient on log capital is $\hat{\alpha}/\hat{\chi}$, which we multiply by $\hat{\chi}$ to recover $\hat{\alpha}$. Estimates of $\alpha$ range from 0.03 to 0.05 across industries.

\subsection{Step 3: Recovering Manager Fixed Effects}

After estimating the revenue function, we compute residualized surplus by removing the contributions of physical capital and other controls:
\begin{equation}
\tilde{s}_{imst} = s_{imst} - \hat{\alpha} k_{it} - \text{controls} - \hat{\mu}_{st}
\end{equation}

This residualized surplus contains three components: organizational capital ($a_i$), manager skill ($z_m$), and residual productivity ($\omega_{it}$). To separate manager effects from firm effects, we implement three complementary approaches.

\subsubsection{Within-Firm CEO Changes}

For firms experiencing CEO transitions, we can identify relative skill differences between successive CEOs. Normalizing the first CEO's skill to zero, subsequent CEOs' effects measure productivity changes relative to this baseline. If a firm has $n$ CEOs over time, we can estimate $n-1$ relative effects. This within-firm approach differences out time-invariant organizational capital, providing clean identification under our assumption that residual productivity shocks are uncorrelated with CEO changes.

The challenge is that these within-firm estimates only identify relative, not absolute, skill levels. We cannot compare CEOs across different firms without additional structure. Moreover, firms with CEO changes may differ systematically from those with stable leadership, limiting external validity.

\subsubsection{Largest Connected Component}

To enable broader comparisons, we estimate a two-way fixed effects model in the spirit of \citet{Abowd1999Econometrica}:
\begin{equation}
\tilde{s}_{imst} = \theta_i + \psi_m + \nu_{imst}
\end{equation}
where $\theta_i$ are firm fixed effects and $\psi_m$ are manager fixed effects.

This system is only identified for firms and managers in the same connected component—sets linked through manager mobility. Two CEOs' skills can be compared if they worked at the same firm or connect through a chain of firms and CEOs. Think of it as a network where firms are squares, managers are circles, and edges represent employment relationships. Within a connected component, all manager effects are comparable because they share a common reference point.

Our largest connected component contains roughly 26,000 managers and comparable numbers of firms. While this represents only 10-15\% of all managers, it captures a substantial share of economic activity since connected firms tend to be larger and more successful. Managers outside this component—typically founders who never leave their firms—cannot be compared to those within it.

The practical estimation proceeds by selecting the largest connected component, normalizing one manager's effect to zero as the reference point, and estimating the system via within-transformation. The resulting manager fixed effects $\hat{\psi}_m$ represent skill differences relative to the reference manager.

\subsubsection{The Noise Problem}

Both approaches yield manager effect estimates, but these estimates combine true skill with noise. Specifically, the estimated effect for manager $m$ at firm $i$ equals:
\begin{equation}
\hat{\psi}_m = \psi_m + \frac{1}{N_{im}} \sum_{t \in T_{im}} \omega_{it}
\end{equation}
where $N_{im}$ is the number of years manager $m$ works at firm $i$, and $T_{im}$ denotes their tenure period.

This averaging of residual productivity creates a severe small-sample problem. Consider a CEO managing a firm for just three years—a typical tenure in our data. Their estimated fixed effect includes the average of three random productivity shocks. If these shocks happen to be positive, we incorrectly classify them as a "good" CEO; if negative, as "bad." With short tenures dominating private firm data, most of the variation in estimated fixed effects reflects noise rather than true skill differences.

\subsection{Step 4: Placebo-Controlled Event Studies}

To separate signal from noise, we implement a novel placebo-controlled event study design. The key insight is that randomly splitting a long CEO tenure creates the same mechanical averaging problem as actual CEO changes, but without any true skill difference. By comparing actual transitions to these placebo transitions, we can filter out the noise component.

\subsubsection{Constructing Placebo Transitions}

We begin by identifying firms with stable leadership—those where the same CEO remains for at least seven consecutive years. These firms provide a clean environment where we know no actual skill change occurs. Within each stable firm, we randomly assign a placebo CEO change using the empirical hazard function from actual transitions (approximately 20\% annual probability).

For example, consider a firm with the same CEO from 1992 to 2000. We might randomly assign a placebo transition in 1996, splitting the tenure into two pseudo-CEOs: one from 1992-1995 and another from 1996-2000. Crucially, we ensure placebo transitions never overlap with actual CEO changes in the same firm, preserving the interpretation that no real skill change occurs.

The timing of placebo transitions follows the same distribution as actual changes. Just as real CEOs face a 20\% annual replacement probability, our placebo CEOs "exit" with the same hazard. This ensures the mechanical noise properties—the averaging of residual productivity over different tenure lengths—match between treatment and placebo groups.

\subsubsection{Event Study Design}

We implement the event study over a seven-year window: four years before the CEO change ($t \in \{-4, -3, -2, -1\}$) and three years after ($t \in \{0, 1, 2, 3\}$), where $t=0$ denotes the first year of the new CEO. The pre-period allows us to check for anticipation effects or reverse causality, while the post-period captures the new CEO's impact.

For both actual and placebo transitions, we estimate:
\begin{equation}
s_{it} = \sum_{\tau=-4}^{3} \beta_\tau D_{it}^\tau + \gamma_i + \delta_t + \varepsilon_{it}
\end{equation}
where $D_{it}^\tau$ indicates event time $\tau$, $\gamma_i$ are firm fixed effects, and $\delta_t$ are calendar year effects. We normalize $\beta_{-1} = 0$, making the year before transition our reference point.

The challenge is that our "control" group (placebo firms) also receives a treatment, just a fake one. Standard difference-in-differences estimators assume untreated controls. We therefore adapt the approach, similar to \citet{Callaway2021JoLE}, to account for two treatment types. The key innovation is aligning placebo transitions with actual transitions in event time—both groups experience their respective "changes" at $t=0$, enabling precise comparison of dynamics.

\subsubsection{Identifying Variation in CEO Quality}

Simply comparing average outcomes between actual and placebo transitions would only reveal whether CEO changes matter at all, not whether better CEOs outperform worse ones. To examine quality differences, we split actual transitions based on the incoming CEO's estimated fixed effect from Step 3.

Despite the noise in these estimates, they contain signal about true quality. CEOs with above-median estimated effects are classified as "good," those below as "bad." We then run separate event studies for each group, always comparing against the same placebo transitions.

This approach creates a specific bias: since estimated effects include noise, "good" CEOs are partly good by luck (positive residual productivity draws), while "bad" CEOs are partly bad by misfortune. The placebo control precisely corrects this bias by measuring how much of the apparent quality difference stems from this mechanical selection on noise.

\subsubsection{Interpreting the Placebo Correction}

The event study yields three key pieces of evidence. First, examining pre-trends reveals whether firms change CEOs in response to performance changes. Second, comparing actual to placebo transitions identifies the average causal effect of CEO replacement. Third, contrasting good versus bad CEOs (both relative to placebos) estimates the variance in CEO quality.

The placebo correction is substantial. Without it, we would conclude that good CEOs improve firm performance by 22.5\% relative to bad CEOs. The placebo transitions, however, generate a 17.0\% "effect" despite no actual change occurring—pure noise from the mechanical averaging problem. The true causal difference is thus only 5.5\% (22.5\% - 17.0\%), meaning three-quarters of the apparent CEO effect is spurious.

This finding has profound implications for empirical work using manager fixed effects. Studies that treat estimated manager effects as explanatory variables likely capture mostly noise, especially when based on short panels. Our placebo method provides a practical solution: always benchmark against placebo transitions to separate signal from noise.

\end{document}
