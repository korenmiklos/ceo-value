\documentclass[11pt,a4paper]{article}
\usepackage[utf8]{inputenc}
\usepackage[T1]{fontenc}
\usepackage{amsmath,amsfonts,amssymb}
\usepackage{apacite}
\usepackage{natbib}
\usepackage{graphicx}
\usepackage{booktabs}
\usepackage{threeparttable}
\usepackage{url}
\usepackage{hyperref}
\usepackage[margin=2.5cm]{geometry}
\usepackage{setspace}
\onehalfspacing

\newcommand{\Var}{\text{Var}}
\newcommand{\Cov}{\text{Cov}}

% Define \sym command for significance stars from esttab
\newcommand{\sym}[1]{{#1}}

\title{Placebo-Debiasing CEO Fixed Effects: A Measurement-Error Correction for Event-Study and Fixed-Effect Analyses\thanks{Project no. 144193 has been implemented with the support provided by the Ministry of Culture and Innovation of Hungary from the National Research, Development and Innovation Fund, financed under the KKP\_22 funding scheme. This project was funded by the European Research Council (ERC Advanced Grant agreement number 101097789). The views expressed in this research are those of the authors and do not necessarily reflect the official view of the European Union or the European Research Council. \emph{Author contributions:} Conceptualization and study design: Koren, Orbán and Telegdy. Data curation, integration and quality assurance: Szilágyi and Vereckei. Statistical analysis: Koren and Telegdy. Writing the original draft: Koren. Review and editing: Koren, Orbán, Szilágyi, Telegdy and Vereckei. \emph{AI disclosure:} Claude Sonnet 4 was used to write and edit the research code and to format the manuscript (such as editing tables, figures, references, creating summaries). All code and text generated by AI tools were reviewed and edited by the authors. All authors have read and agreed to the published version of the manuscript. \emph{Data availability statement:} The data underlying this article cannot be shared publicly due to privacy and licensing restrictions. The replication package is available at \url{https://github.com/korenmiklos/ceo-value}.}}

\author{Miklós Koren\thanks{Central European University, HUN-REN Centre for Economic and Regional Studies, CEPR and CESifo. E-mail: korenm@ceu.edu} \\
        Krisztina Orbán\thanks{Monash University.} \\
        Bálint Szilágyi\thanks{HUN-REN Centre for Economic and Regional Studies.} \\
        Álmos Telegdy\thanks{Corvinus University of Budapest.} \\
        András Vereckei\thanks{HUN-REN Centre for Economic and Regional Studies, Institute of Economics.}}

\date{\today}

\begin{document}

\maketitle

\begin{abstract}
Empirical work often treats estimated CEO fixed effects as regressors or summary statistics of managerial quality. Because these effects are measured with error, their use induces systematic biases in means, covariances, variances, and regression slopes. We develop a placebo-controlled design that identifies and removes these biases without high-dimensional modeling of the shock process. The key decomposition shows that the covariance between outcomes and estimated effects and the variance of the estimated effects are inflated by terms that depend only on the spell design and the autocovariance structure of shocks. By constructing placebo CEO spells at firms that do not change managers but mimic the design of treated spells, we nonparametrically estimate these bias components and form debiased estimators of first and second moments and of regression coefficients. The approach corrects spurious pre-trends in event studies and the attenuation/inflation mix in regressions. In an application to the universe of Hungarian firms (1992--2022), the placebo-corrected event-study slope is 5.5\%, one quarter of the raw 22.5\% contrast, implying that roughly three quarters of the apparent effect is noise. We provide implementation guidance and Monte Carlo logic clarifying when biases vanish (i.i.d. shocks) and when they are large (persistent shocks, short spells).
\end{abstract}

\textbf{Keywords:} CEO fixed effects, measurement error, placebo design, event study

\textbf{JEL Classification:} [To be added]

\newpage

\section{Introduction}
Fixed-effect measures of CEO quality are pervasive in empirical corporate and labor research. They summarize persistent differences across managers and underpin analyses that relate managerial quality to firm outcomes and policies. Yet estimated fixed effects are noisy. Using them naively---as regressors, in event studies, or to compute dispersion---systematically biases inference. In settings with persistent shocks and short spells, the biases can be first order.

We propose a simple solution: a placebo-controlled design that directly measures and removes noise-induced biases using firms that do not actually change CEOs. Placebo spells replicate the empirical spell design of treated firms while excluding periods around real transitions. Because no manager change occurs for placebo firms, any ``effect'' observed there is purely mechanical and reveals the bias terms. Subtracting these terms from the corresponding treated moments yields debiased estimates of means, variances, covariances, and slopes.

Our contributions are threefold. First, we provide a transparent bias decomposition for moments involving estimated CEO effects that holds under arbitrary shock autocovariance and general spell designs. Second, we show how placebo spells identify the bias components needed to debias both event-study contrasts and regressions of outcomes on estimated effects. Third, we offer a practical implementation with clear diagnostics and apply it to comprehensive administrative data from Hungary, where the placebo-corrected contrast is 5.5\% versus a naive 22.5\%.

Methodologically, our approach complements two-way fixed-effect frameworks \citep{Abowd1999Econometrica,Card2018JoLE} and modern event-study estimators \citep{Callaway2021JoLE}. Substantively, it reconciles large correlational estimates with more modest quasi-experimental findings on managerial impacts \citep{bennedsen2020ceos}.

\subsection*{Why this matters for applied work}
Fixed-effect measures of CEO quality are widely used as inputs into applied analyses. But most private firms operate under owner control, managers mainly adjust variable inputs, and estimated manager effects blend true skill with firm shocks and selection. In our companion study on Hungarian firms (1992--2022), naive event-study contrasts suggest a 22.5\% gap between firms hiring better versus worse CEOs, while placebo transitions---constructed to mimic spell design but excluding actual changes---produce 17.0\%. The remaining 5.5\% is the causal impact. For policy, corporate governance, and compensation, getting from 22.5\% to 5.5\% matters: it changes predicted gains from CEO replacement and where to target performance incentives.

\subsection*{A simple worked example: ATET$(g,t)$ and the spell design}
We use a concrete toy design that we revisit throughout. Consider an event window $t\in\{-2,-1,0,+1,+2\}$ with baseline $t=-1$. Let $g\in\{-2,\ldots,+2\}$ index the \emph{lead/lag at which we evaluate the contrast} relative to the transition (e.g., $g=-2$ refers to two years before the change; $g=+1$ one year after). Define the average treatment effect on the treated at $(g,t)$ for firms that replace a worse CEO with a better CEO versus those that replace with a worse CEO:
\begin{equation*}
\text{ATET}(g,t) := \mathbb E\big[(y_{i,t}-y_{i,-1})\,\big|\,\text{better},\,G=g\big]
\;-
\mathbb E\big[(y_{i,t}-y_{i,-1})\,\big|\,\text{worse},\,G=g\big],
\end{equation*}
where $y_{i,t}$ is the outcome (e.g., residualized surplus) of firm $i$ at event time $t$, and $G=g$ indicates the evaluation lead/lag. The placebo-corrected contrast subtracts the same statistic computed on placebo spells:
\begin{equation*}
\text{ATET}^{\,db}(g,t) = \text{ATET}^{\,act}(g,t) - \text{ATET}^{\,pl}(g,t).
\end{equation*}
In our data, pooling across $g$ in the window $[-2,+2]$ yields 22.5\% (actual), 17.0\% (placebo), and 5.5\% (debiased).

\paragraph{Design matrices for 2 and 3 spells.} Let $\mathbf D$ map the five event-time rows into spell indicators. For a two-spell case (CEO A from $t=-2\!:\!-1$, CEO B from $t=0\!:\!+2$):
\[
\mathbf D^{(2)} =
\begin{bmatrix}
1 & 0\\
1 & 0\\
0 & 1\\
0 & 1\\
0 & 1
\end{bmatrix},\quad
\mathbf T^{(2)}=\mathrm{diag}(2,3),\quad
\mathbf P^{(2)}=\mathbf D^{(2)}{\mathbf T^{(2)}}^{-1}{\mathbf D^{(2)}}'.
\]
For a three-spell case (e.g., CEO A at $t=-2\!:\!-1$, a short CEO B at $t=0$, CEO C at $t=+1\!:\!+2$):
\[
\mathbf D^{(3)} =
\begin{bmatrix}
1 & 0 & 0\\
1 & 0 & 0\\
0 & 1 & 0\\
0 & 0 & 1\\
0 & 0 & 1
\end{bmatrix},\quad
\mathbf T^{(3)}=\mathrm{diag}(2,1,2),\quad
\mathbf P^{(3)}=\mathbf D^{(3)}{\mathbf T^{(3)}}^{-1}{\mathbf D^{(3)}}'.
\]
With weights $\mathbf w$ that implement $y_{i,t}-y_{i,-1}$, $\hat z_i=\mathbf w'\mathbf P\,\mathbf y_i$ is a spell-mean of outcomes combined per $\mathbf w$. Short middle spells (as in $\mathbf D^{(3)}$) raise noise, making debiasing more important.

\section{Model Setup}
Consider firm $i$ observed for $T$ periods with outcome vector $\mathbf y_i \in \mathbb R^T$ and a (possibly piecewise-constant) latent manager-effect path $\mathbf z_i$. Let shocks be $\mathbf e_i$ with $\mathbb E[\mathbf e_i]=0$ and $\mathbb E[\mathbf e_i\mathbf e_i']=\boldsymbol\Sigma_g$, where $g$ indexes a design group (e.g., spell-length pattern). The outcome decomposes as
\begin{equation}
\mathbf y_i = \mathbf z_i + \mathbf e_i,\qquad \mathbb E[\mathbf z_i\mathbf e_i']=0.
\end{equation}
For a linear functional with weights $\mathbf w\in\mathbb R^T$, define the scalar outcome $y_i=\mathbf w'\mathbf y_i$ and its components $z_i=\mathbf w'\mathbf z_i$ and $\varepsilon_i=\mathbf w'\mathbf e_i$.

\paragraph{Estimating CEO effects.} Within a group $g$, let $\mathbf D_g$ denote the $T\times S$ design matrix that maps calendar time into the $S$ manager spells, and let $\mathbf T_g$ be diagonal with spell lengths. The canonical within-spell mean estimator of the path is
\begin{equation}
\hat{\mathbf z}_i = \mathbf P_g\,\mathbf y_i = \mathbf D_g\,\mathbf T_g^{-1}\mathbf D_g'\,\mathbf y_i = \mathbf z_i + \mathbf P_g\,\mathbf e_i,
\end{equation}
where $\mathbf P_g$ is a group-specific projection operator. The estimated scalar effect is $\hat z_i = \mathbf w'\hat{\mathbf z}_i = z_i + \eta_i$ with $\eta_i=\mathbf w'\mathbf P_g\mathbf e_i$.

\section{Bias Decomposition}\label{sec:bias}
\paragraph{Interpreting through the worked example.} In the two-spell design $\mathbf D^{(2)}$, shocks that are positively autocorrelated inflate both $\Cov(y,\hat z)$ and $\Var(\hat z)$ by similar magnitudes when spells are balanced. In the three-spell design $\mathbf D^{(3)}$ with a short middle spell, $\Var(\hat z)$ is disproportionately inflated (because one spell mean uses a single observation), while $\Cov(y,\hat z)$ may inflate less or even shift sign depending on which side of $t=-1$ the short spell sits. Hence $A_g\neq B_g$ becomes pronounced, creating spurious pre-trends and slope bias unless corrected. This is exactly what placebo spells recover nonparametrically.
Consider the population regression $y_i=\beta z_i+u_i$ with $\mathbb E[z_i u_i]=0$ and $u_i=\varepsilon_i$. Two bias components pin down the discrepancy between moments involving $\hat z_i$ and those involving $z_i$:
\begin{align}
A_g &:= \Cov(\varepsilon_i,\eta_i) = \mathbf w'\boldsymbol\Sigma_g\,\mathbf P_g\,\mathbf w,\\
B_g &:= \Var(\eta_i) = \mathbf w'\mathbf P_g\,\boldsymbol\Sigma_g\,\mathbf P_g\,\mathbf w.
\end{align}
Then
\begin{align}
\Cov(y,\hat z) &= \beta\Var(z) + A_g,\\
\Var(\hat z) &= \Var(z) + B_g,
\end{align}
so the naive slope from regressing $y$ on $\hat z$ is
\begin{equation}
\tilde\beta_g = \frac{\beta\Var(z)+A_g}{\Var(z)+B_g} = \beta + \frac{A_g-\beta B_g}{\Var(z)+B_g}.
\end{equation}
Special cases clarify intuition. With i.i.d. shocks (balanced panel), $A_g=B_g$, so $\tilde\beta_g=\beta$ (no slope bias) even though $\Var(\hat z)$ is inflated by $B_g$. With persistent shocks, $A_g\neq B_g$, generating slope bias and spurious pre-trends in event studies.

\section{Placebo Design and Identification}
\paragraph{Worked example continued.} For the $\mathbf D^{(2)}$ toy design, construct placebo spells by selecting firms that never change CEO over the window but have five consecutive observations, then assign the breakpoint between the first two and last three years and compute the same event-time contrasts. Any nonzero ATET on these placebo spells is mechanical bias ($\hat A_g,\hat B_g$). For $\mathbf D^{(3)}$, select placebo firms with a five-year run, place a single-year ``short spell'' at $t=0$, and treat $t=+1\!:\!+2$ as the third spell. The larger placebo moments in this design illustrate how short spells magnify bias; subtracting them yields stable debiased contrasts across designs.
Construct placebo spells among firms that do not change CEOs, replicating the distribution of $\mathbf D_g$ while excluding windows around any actual transitions. For placebo firms, $\Var(z)=0$, so
\begin{equation}
\Cov(y,\hat z) = A_g,\qquad \Var(\hat z)=B_g.
\end{equation}
Hence, placebo moments identify the bias components:
\begin{equation}
\hat A_g = \widehat{\Cov}^{\,\text{pl}}(y,\hat z),\qquad \hat B_g = \widehat{\Var}^{\,\text{pl}}(\hat z).
\end{equation}
For treated spells, form debiased estimators of the moments and the slope:
\begin{align}
\widehat{\Cov}^{\,\text{db}}(y,\hat z) &= \widehat{\Cov}^{\,\text{tr}}(y,\hat z) - \hat A_g,\\
\widehat{\Var}^{\,\text{db}}(\hat z) &= \widehat{\Var}^{\,\text{tr}}(\hat z) - \hat B_g,\\
\hat\beta^{\text{db}}_g &= \frac{\widehat{\Cov}^{\,\text{tr}}(y,\hat z)-\hat A_g}{\widehat{\Var}^{\,\text{tr}}(\hat z)-\hat B_g}.
\end{align}
Aggregating across groups with weights $w_g$ yields a pooled estimator
\begin{equation}
\hat\beta^{\text{db}}_{\text{pool}} = \frac{\sum_g w_g\,\big(S^{(g)}_{xy}-\hat A_g\big)}{\sum_g w_g\,\big(S^{(g)}_{zz}-\hat B_g\big)},
\end{equation}
where $S^{(g)}_{xy}$ and $S^{(g)}_{zz}$ are treated covariances and variances.

\paragraph{Inference.} Standard errors follow from a delta method applied to the ratio or from a nonparametric bootstrap that resamples firms within groups, preserving spell design.

\section{Implementation Guide}
\paragraph{Concrete recipe for the example.}
- Event window and weights: set $t\in[-2,+2]$; implement $y_{i,t}-y_{i,-1}$ via weights $\mathbf w$ that subtract the baseline row.
- Design groups: define $g\in\{\mathbf D^{(2)},\mathbf D^{(3)}\}$ by whether the observed transition sequence matches two- or three-spell patterns.
- Treated moments: within each $g$, compute $S^{(g)}_{xy}$ and $S^{(g)}_{zz}$ on actual transitions, contrasting better vs. worse CEO replacements.
- Placebo moments: on placebo firms matched to $g$, compute $\hat A_g$ and $\hat B_g$ using the same $\mathbf D$ and $\mathbf w$.
- Debias and pool: form $\hat\beta^{db}_g$ per Section~\ref{sec:bias}; pool by group sizes or precision weights.
- Diagnostics to report: overlap of treated/placebo spell-length distributions; magnitude of $\hat A_g,\hat B_g$; residual pre-trends at $t<0$ after debiasing.
\begin{enumerate}
  \item Define the statistic $y_i=\mathbf w'\mathbf y_i$ and the weights $\mathbf w$ appropriate for the event-study window (e.g., baseline at $-1$, window $[-2,+2]$) or the target linear functional.
  \item Within each design group $g$ (e.g., by spell-length pattern), compute $\hat z_i=\mathbf w'\mathbf P_g\,\mathbf y_i$ using the within-spell mean operator $\mathbf P_g=\mathbf D_g\mathbf T_g^{-1}\mathbf D_g'$.
  \item Estimate treated moments $S^{(g)}_{xy}$ and $S^{(g)}_{zz}$ and placebo moments $\hat A_g,\hat B_g$ on matched placebo spells that replicate $\mathbf D_g$ and exclude windows around actual transitions.
  \item Form debiased moments and slopes using the formulas above; pool across $g$ with weights (e.g., group sizes or precision weights).
  \item Report diagnostics: (i) similarity of placebo and treated spell-length distributions; (ii) placebo pre-trends; (iii) magnitude of $\hat A_g$ and $\hat B_g$ relative to treated moments.
\end{enumerate}

\section{Monte Carlo Logic}
Four design lessons guide applied work:
\begin{enumerate}
  \item \textbf{i.i.d. shocks, balanced panels:} $A_g=B_g$; regression slopes are unbiased though variances inflate by a known amount.
  \item \textbf{Persistent shocks (AR(1)):} $A_g\neq B_g$; slopes are biased by $(A_g-\beta B_g)/(\Var(z)+B_g)$; spurious pre-trends emerge.
  \item \textbf{Unbalanced spells:} Short spells amplify noise; the gap $|A_g-B_g|$ typically grows with persistence and shorter $T$.
  \item \textbf{Groupwise volatility scales:} If $\boldsymbol\Sigma_g=s_g\boldsymbol\Sigma_0$, then $A_g=s_gA_0$ and $B_g=s_gB_0$; placebo debiasing still applies.
\end{enumerate}

\section{Empirical Illustration}
\paragraph{Reading the main results through the example.}
- Naive ATET: pooling across $t\in[-2,+2]$ and across two- and three-spell designs yields 22.5\%.
- Placebo ATET: replicating the same designs on placebo firms yields 17.0\%.
- Debiased ATET: 5.5\% remains, our causal estimate.
- Pattern by design: placebo moments are modest under $\mathbf D^{(2)}$ but larger under $\mathbf D^{(3)}$, consistent with short spells amplifying noise; debiased effects align across designs.
- Dynamics: after debiasing, pre-trends flatten near $t<0$ while post-$t=0$ gains persist, matching the $1/\chi$ interpretation from the production framework.
We implement the design on Hungarian administrative data (1992--2022), following the sample and event-study construction in the companion analysis. The naive contrast between firms hiring higher- versus lower-skill CEOs is 22.5\% (ATET). Placebo transitions—randomly assigned fake CEO changes that replicate empirical spell design while excluding actual windows—produce a 17.0\% effect despite no manager change. The placebo-corrected effect is therefore 5.5\%, which we interpret as the causal impact of CEO-skill differences within the event window.

This magnitude is consistent with quasi-experimental evidence \citep{bennedsen2020ceos} and with a broader view in which much of the cross-sectional dispersion attributed to CEOs reflects persistent shocks, mean reversion, and endogenous transition timing. The placebo framework also corrects spurious pre-trends, aligning dynamic profiles with credible identifying assumptions \citep{Callaway2021JoLE}.

\section{Practical Guidance}
\paragraph{Applied takeaways using the example.}
- Use debiasing whenever short spells are prevalent or shocks are persistent; check placebo moments by design.
- Report design matrices (like $\mathbf D^{(2)}$ and $\mathbf D^{(3)}$) and baseline weights to make the contrast transparent.
- Prefer left-hand-side uses of estimated CEO quality; if used on the right-hand side, report $\hat A_g$ and $\hat B_g$ and provide debiased slopes.
- In event studies, show both naive and debiased profiles; explain the gap using the placebo example so readers see which part is noise.
\begin{itemize}
  \item \textbf{When to debias:} Always when estimated effects are on the right-hand side or when event-study contrasts use noisy estimated regressors; particularly crucial with persistent shocks or short spells.
  \item \textbf{Design groups:} Define $g$ by spell-length patterns and windows. Report results pooled and by $g$.
  \item \textbf{Reporting:} Present naive and debiased moments side by side; include placebo diagnostics; report $\hat A_g,\hat B_g$ magnitudes.
  \item \textbf{Usage of CEO quality:} Prefer using CEO quality on the left-hand side where classical error is less problematic; avoid naive regressions with noisy right-hand-side fixed effects.
\end{itemize}

\section{Conclusion}
Estimated CEO fixed effects are informative but noisy. We show that their naive use induces predictable biases governed by the spell design and the shock autocovariance. A placebo-controlled design identifies and removes these biases with minimal structure, delivering debiased moments and slopes and correcting spurious pre-trends. In comprehensive administrative data, this correction reduces a 22.5\% raw contrast to 5.5\%, reconciling correlational and quasi-experimental evidence on managerial impacts. The approach is easy to implement, transparent to diagnose, and broadly applicable wherever estimated fixed effects are used as regressors or in dynamic contrasts.

\bibliographystyle{apacite}
\bibliography{references}

\end{document}
