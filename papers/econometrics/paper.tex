\documentclass[11pt,a4paper]{article}
\usepackage[utf8]{inputenc}
\usepackage[T1]{fontenc}
\usepackage{amsmath,amsfonts,amssymb}
\usepackage{apacite}
\usepackage{natbib}
\usepackage{graphicx}
\usepackage{booktabs}
\usepackage{threeparttable}
\usepackage{url}
\usepackage{hyperref}
\usepackage[margin=2.5cm]{geometry}
\usepackage{setspace}
\usepackage{comment}
\usepackage{outlines}
\onehalfspacing

\newcommand{\Var}{\text{Var}}
\newcommand{\Cov}{\text{Cov}}

% Define \sym command for significance stars from esttab
\newcommand{\sym}[1]{{#1}}

\title{Debiasing Second Moments of Estimated CEO Effects: A Placebo-Controlled Event-Study Approach\thanks{Project no. 144193 has been implemented with the support provided by the Ministry of Culture and Innovation of Hungary from the National Research, Development and Innovation Fund, financed under the KKP\_22 funding scheme. This project was funded by the European Research Council (ERC Advanced Grant agreement number 101097789). Telegdy received support from the Hungarian Scientific Research Fund – OTKA, contract number 143346. The views expressed in this research are those of the authors and do not necessarily reflect the official view of the European Union or the European Research Council. \emph{Author contributions:} Conceptualization and study design: Koren, Orbán and Telegdy. Statistical analysis: Koren, Orbán and Telegdy. Writing the original draft: Koren. Review and editing: Koren, Orbán and Telegdy.}}

\author{Miklós Koren\thanks{Central European University, ELTE Centre for Economic and Regional Studies, CEPR and CESifo. E-mail: korenm@ceu.edu} \\
        Krisztina Orbán\thanks{Monash University.} \\
        Álmos Telegdy\thanks{Corvinus University of Budapest.}}

\date{September 2025}

\begin{document}

\maketitle
\thispagestyle{empty}

\begin{abstract}
Understanding the link between estimated manager effects and firm outcomes is central to research on CEOs and firm performance. While fixed-effect estimators of mean CEO effects are unbiased under random mobility, most applied analyses use second moments of these estimates—variances, covariances, correlations, and event-study dynamics—that are systematically biased by small-sample noise. We develop a placebo-controlled event-study design that identifies and removes these biases without modeling the full error process. The method constructs placebo transitions that replicate the spell-length design of actual transitions while excluding periods around real changes. Moments computed on placebo spells recover the bias components that inflate variances and covariances; subtracting them yields debiased moments and regression slopes. A Monte Carlo study with six scenarios—baseline, long panel, persistent errors, unbalanced panels, excess variance, and all complications—illustrates when the correction matters most. We outline an application to the universe of Hungarian firms (1992--2022) that examines pre-trends and dynamic effects after CEO arrival, a common use case in the literature.
\end{abstract}

\textbf{Keywords:} CEO value, private firms, productivity

\textbf{JEL Classification:} D24, G34

\clearpage
\setcounter{page}{1}

%%%%%%%%%%%%%%%%%%%%%%
\section{Introduction}
%%%%%%%%%%%%%%%%%%%%%%

CEO effects are prevalent in empirical corporate and labor research. Under random mobility (strict exogeneity), first moments of estimated fixed effects are unbiased—an insight reinforced by designs that exploit exogenous mobility such as CEO death or illness \citep[e.g.,][]{bennedsen2020ceos}. But most applied work is concerned with second moments: variances (how large is CEO heterogeneity?), covariances with other variables (what do fixed effects capture?), and dynamic covariances used in event studies (are there pre-trends and post-arrival dynamics?). These second moments are known to be biased in short panels, as emphasized by the worker–firm fixed-effects literature \citep{andrews2008high,gaure2014correlation,Bonhomme2023-dx}.

%what we do
We develop a placebo-controlled design to debias second moments of estimated CEO effects without modeling the full variance–covariance of errors. The approach constructs placebo transitions that replicate the spell-length design of actual transitions while excluding windows around real changes. Computing the same statistic with the same weights in the treated and placebo samples, and subtracting the placebo moment, yields debiased variances, covariances, and event-study profiles. Our assumptions are transparent: (i) strict exogeneity (random mobility) for unbiased first moments; (ii) the autocovariance of shocks is the same up to a scalar across treated and placebo groups; and (iii) design matrices and group weights match by construction. We illustrate performance in six Monte Carlo scenarios (baseline, long panel, persistent errors, unbalanced panels, excess variance, and all complications), and we outline a typical application—dynamic effects of CEO arrival on TFP in Hungarian firms—where researchers care about pre-trends and post-arrival dynamics.

%Why is our approach unique - data
Papers use three approaches when analyzing the effect of CEOs. One group studied the effects of a distinct, measurable variable, such as managerial practices used in the firm \citep{bloom2012organization} or a CEO attribute (age, gender, tenure, education or country of origin \citep{anderson2018pathways, henderson2006quickly, Koren2023expat}). These studies are useful to show the effects of the studied variables, but they arguably capture little of the whole effect of CEOs over the firms they manage. Another group uses an external event to assess the effect of CEO turnover (e.g., \citet{bennedsen2020ceos}). While this approach brought about credible causal estimates, the heavy data needs render this approach impossible to use in most situations. 

The third group employed firm and CEO fixed effects to separate the quality of firm from that of the CEOs \citep{Bertrand2003-io, crossland2011differences, quigley2015has}. While these studies employ a framework that can be used in many settings, the generalizability of their results is challenged by the data used because almost all study public companies only.\footnote{\citet{quigley2022ceo} is a notable exception studying CEO effects in private businesses, but the private sample comprises of large firms.} Our data has information on the universe of Hungarian businesses and all their CEOs for the period between 1992 and 2022, tracking over 1 million firms and a similar number of CEOs.\footnote{Mandatory registration of all company directors ensures complete coverage of CEOs in the whole corporate sector.} The overwhelming majority of the firms in our sample are small and medium-sized enterprises (SMEs). This adds a new dimension to the analysis that was largely missing from the literature and allows drawing conclusions about CEOs' effects on the whole economy. Our analysis is also distinctive in that earlier studies focused on developed countries, while our data come from Hungary. Indeed, the institutional context matters: \citet{crossland2011differences} show that informal and formal institutions affect CEO discretion.

%Why is our approach unique - estimation
We measure firm productivity using total factor productivity (TFP) estimated from a standard production function. Firm productivity is the dependent variable in our baseline regression, where we correlate productivity with firm and CEO fixed effects. We define the value of each CEO as the average firm productivity measured over all firm-years when the CEO was in office.

While the estimation of CEO effects is unbiased under standard identification assumptions of fixed effects regressions, bias may arise when regressions are conducted on subsamples segmented by CEO quality. (Estimating the effect of CEO quality on firm performance is the ultimate goal of such studies.) This is caused by the short time span of many CEOs and also by the fact that we observe few CEO replacements for a firm.\footnote{This is the limited mobility bias, which contaminates the second moment of wage regressions in the presence of worker and firm fixed effects \cite{kline2024firm}.} CEO fixed effects are estimated from few changes and CEO and firm fixed effects will partly overlap. Firms that experience a positive shock will be more likely classified as having good CEOs, and those that draw a negative shock as having bad CEOs. To address this bias, we introduce a placebo-controlled event study design.

Most applied work is concerned with second moments of estimated fixed effects. We highlight three use cases. First, variance decompositions that ask how much CEO fixed effects contribute to dispersion; these are upward biased because the estimator packs in small-sample noise from short spells (limited mobility). Second, covariances with other variables used to interpret what the fixed effects capture; for example, whether firms led by better CEOs export more. If the outcome covaries with the estimation error in the fixed effect (e.g., unobserved productivity that also predicts exporting), the covariance with the estimated effect is biased; correlations are ratios of second moments and are therefore highly non-linear and biased unless their components are debiased first. Third, dynamic covariances used in event studies to assess pre-trends and post-transition dynamics. Without correction, persistent shocks and short spells may generate spurious pre-trends even when the causal pre-trend is zero.

Our solution is placebo-controlled. We create placebo job spells in firms with no actual CEO change by splitting long no-change windows into the same pre/post spell lengths as the treated sample and aligning them in event time. Matching on covariates such as firm age is good practice because volatility is age dependent. Computing the same second moment with the same weights in treated and placebo samples, and subtracting the placebo quantity yields a debiased estimate. The key intuition is that covariance is bilinear and variance is additive under strict exogeneity. We also allow the shock variance to differ by a scalar between treated and placebo groups and estimate this scalar at the design-group level.

Assumptions and intuition. We make three assumptions to support identification and simple implementation: (i) strict exogeneity (random mobility) so that first moments are unbiased; (ii) the error autocovariance matrix (Sigma) is the same across treated and placebo groups up to a scalar; and (iii) the design matrices and any group weights are the same by construction because the placebo sample replicates the treated spell-length distribution and event-time alignment. Because covariance is bilinear, the bias term in any covariance with the estimated effect can be “precomputed” in the placebo sample; we subtract it from the treated covariance. Because variance is additive under strict exogeneity, the variance of the estimated effect splits into the variance of the true effect plus the variance of the small-sample error; the latter is measured in the placebo and subtracted. For non-linear transformations such as correlations and regression coefficients (beta), we first debias all underlying covariances and variances and only then form the ratio. [[comment: add a brief reference to pre-trend testing guidance, e.g., Roth, and clarify baseline normalization in event time]]

The main idea of this correction is the creation of pseudo CEO transitions in the data\footnote{This step in the spirit of \cite{jarosiewicz2023revisiting}, who compare the replicated results of  \cite{Bertrand2003-io} with placebo regressions.} We take firms with CEO transitions, record their event window. The event window is the period encompassing the years under the CEO prior to the CEO change and the period under the new CEO post the change. We match these firms to firms born in the same year, belonging to the same sector, that however do not experience a CEO transition during the years of the event window. Then, we assign to the control firms a placebo CEO transition happening in the same year as the treated firm experiences the true CEO transition.\footnote{In addition to addressing the limited mobility bias, our approach has the advantage of creating a control group that is more similar to the treated firms than the universe of firms}. On the matched sample we implement two-way fixed effects regressions.\footnote{The estimation is similar to the two way fixed effects regression developed by \cite{Callaway2021JoLE}, but we exploit the fact that the control group also has a time of pseudo CEO change. We compare firms with a true CEO change with firms with a fake CEO change along the same event-time years.} If estimated `effects' for these pseudo-CEOs diverge substantially, this reveals the noise problem in fixed effects estimation. By comparing actual CEO transitions to these placebo transitions, we can correct the fixed effects estimates for noise, isolating the true CEO contribution.

% What we do in brief
In brief, we: (i) estimate log TFP; (ii) recover firm and CEO fixed effects; (iii) construct matched placebo CEO transitions that replicate each treated spell’s window and spell-length pattern while excluding any real changes; (iv) estimate event-study dynamics comparing actual to placebo transitions in aligned event time; (v) use placebo moments to remove noise-induced bias from means, variances, covariances, and regression slopes; and (vi) quantify CEO contributions via an age-adjusted variance decomposition.

%results
Our results indicate that the average CEO transition does not have substantial effects on firm productivity. This result is not counterintuitive: sorting by quality implies that CEO replacements tend to be matched with successors of comparable quality. Nonetheless, sorting is not perfect and we find large swings in productivity when we run the regression for transitions that improve or decline CEO quality.\footnote{In these regression, we also select the control sample by improving and declining productivity along the pseudo CEO transition.} Our placebo-controlled regressions estimate a 3\% increase in productivity when the CEO value increases, and a 2\% decline when it decreases. The low magnitude is due to our correction mechanism: in the placebo estimates, the measured ``effect'' is on the order of 20\%.  

To validate our model, we run the same regressions on the main inputs of the firm. Contrary to our assumption about capital not being affected by CEOs, we find that the value of tangible assets declines before, and increases after the CEO change, but the proportion of firms possessing intangible capital does not change. However, the change in tangible assets is both smaller and slower than the massive change which are measured for both short term inputs, the cost of materials and labor. While the change in tangible assets is 5-10\%, variable inputs change by about 20\% in an order of magnitude larger.

To quantify the effect of CEOs on firm performance in our data, we perform a variance decomposition, which computes what proportion of the total dispersion of productivity change is explained by CEO transitions. As demonstrated by countless papers in labor economics, the variance is also contaminated by the limited mobility bias and we make a correction of it with the  placebo method. With our carefully chosen control group we also take into account the mechanical effect of firm age on productivity dispersion: as time passes, the firm gathers shocks and so the dispersion will increase. 

We find that CEO transitions explain about 30\% of the productivity change in our data. The placebo correction is also important here: without it, this share would be about twice as large.

%literature
Our work connects to the broader literature on management practices and firm productivity. Randomized controlled trials demonstrate that management training and consulting improve firm performance \citep{bloom2013does, mckenzie2021small}, but these interventions change practices rather than people. Whether replacing managers generates similar gains remains contentious. Evidence from public sector organizations suggests modest manager effects \citep{fenizia2022managers, janke2024role}, while studies of family firms find larger impacts when professional managers replace family members \citep{bennedsen2007inside, sraer2007performance}. Our results for private firms fall between these extremes: CEOs matter, but less than raw correlations suggest.

Methodologically, our paper builds on the two-way fixed effects literature in labor economics that decomposes wages into worker and firm components \citep{Abowd1999Econometrica, Card2018JoLE}. These studies face similar challenges from limited mobility creating small-sample bias \citep{andrews2008high} and have developed bias-correction methods \citep{Bonhomme2023-dx, gaure2014correlation}. We adapt this framework to the CEO-firm setting but add placebo controls to separate signal from noise. This approach is valuable when studying managers who, unlike workers, have few observations per individual, making traditional bias-correction methods less effective. Recent work has documented apparently increasing CEO effects over time \citep{quigley2015has}, but these studies do not account for the mechanical noise we identify. \citet{lippi2014corporate} find that concentrated ownership in Italian firms distorts executive selection and reduces productivity by 10\%, providing motivation for our framework separating owner and CEO decisions.

%%%%%%%%%%%%%%%%%%%%
\section{The Econometric Problem}
%%%%%%%%%%%%%%%%%%%%

Let firm $i$ in year $t$ have outcome
\begin{equation}\label{eq:model1}
  y_{it} = z_{m(i,t)} + e_{it},\qquad \mathbb E[e_{it}|z_{is}]=0,
\end{equation}
where $z_{m(i,t)}$ collects the CEO effect at time $t$ and $e_{it}$ is a shock. We assume that $z_{m(i,t)}$ is piecewise constant, changing only when the CEO changes and that $e_{it}$ is mean independent of the CEO path for all $s$ (``strict exogeneity'' or ``random mobility'').

Such fixed effect models are often estimated in applied work. For example, \citet{Bertrand2003-io} ETC ETC

Under the strict exogeneity assumption, the ordinary least squares (OLS) estimator of $z_m$ is unbiased. Equation \eqref{eq:model1} can be rewritten as a dummy-variable regression 
$$
y_{it} = \sum_{n} z_n D_{m(i,t)=n}  + e_{it},
$$
and the first-order condition for the OLS estimator $\hat z_n$ is
$$
0 = \sum_{i,t:m(i,t)=n} (y_{it} - \hat z_n),
$$
or 
$$
\hat z_n = \frac{1}{T_n} \sum_{i,t:m(i,t)=n} y_{it} = z_n + \frac{1}{T_n} \sum_{i,t:m(i,t)=n} e_{it},
$$
where $T_n$ is the number of observations with CEO $n$. The estimator is unbiased because $\mathbb E[\hat z_n|z_n] = z_n$. The estimator, however, is only consistent as $T_n\to\infty$, that is, as the number of observations per CEO grows large. In practice, many CEOs have short tenures (REFs), so $\hat z_n$ contains substantial noise. 

Econometric applications often require the second moments of $\hat z_n$, for example, to estimate the variance of CEO effects or to run regressions of an outcome $y_{it}$ on $\hat z_n$. The former could inform us about the importance of CEO heterogeneity (REFs), while the latter could inform us about the effect of CEO quality on firm performance and firm policies (REFs). 

The small-sample error, however, contaminates \emph{all second moments} of $\hat z_n$. The variance of $\hat z_n$ is inflated relative to the variance of true CEO effects $z_n$ (REFs), and the covariance of $\hat z_n$ with other variables that are affected by $e_{it}$ is also contaminated. A well-known example of this in the labor literature is ``limited mobility bias'' \citep{andrews2008high}, which states that in a two-way fixed effects model of worker and firm effects, the variance of both effects is biased upwards, and their correlation is biased downwards.

Existing methods of bias correction for second moments take two broad forms. One is ad hoc, such as leave-one-out adjustments that hope to remove own-observation noise. These can be impractical in CEO settings with a single manager per firm-year and lack formal guarantees. The other is structural: estimate the bias from the variance–covariance of shocks and the design matrix and then apply Empirical Bayes–type shrinkage (e.g., \citealt{andrews2008high}, \citealt{Bonhomme2023-dx}). While powerful, this requires estimating many components and re-deriving formulas for each target second moment. Our approach is different: we construct placebo transitions that replicate the treated design matrix and use the placebo moments to read off the bias term directly, then subtract it. This delivers debiased variances, covariances, and slopes without modeling the full shock process.


\newenvironment{example}{\par\noindent\textbf{Example.}\ }{\par}

Consider the following example to fix ideas. Our method will be more general, but this example captures the key elements.
\begin{example}
Event window $t\in\{-2,-1,0,+1,+2\}$ with baseline $t=-1$. Pre-spell length $T_1=2$ (years $-2,-1$ under CEO A). Post-spell length $T_2=3$ (years $0, +1,+2$ under CEO B). The contrast of interest is the change from the baseline year $t=-1$ to three years after the transition $t=+3$:
\[
y_{i,+3} - y_{i,-1}.
\]
\end{example}

Stack outcomes as $\mathbf y_i=[y_{i,-2},\,y_{i,-1},\,y_{i,0},\,y_{i,+1},\,y_{i,+2}]'$. Define the piecewise-constant CEO path $\mathbf z_i$ so that $z_{i,-2}=z_{i,-1}$ (CEO A), $z_{i,0} = z_{i,+1}=z_{i,+2}$ (CEO B). The linear contrast for the example is implemented by the weight vector
\[
  \mathbf w = [\,0,\,-1,\,0,\,0,\,+1\,]',
\]
so that 
$$
y_{i,+3} - y_{i,-1} = \mathbf w' \mathbf y_i.
$$

\paragraph{Model setup.} We are now ready to present the general setup in matrix notation. Let firm $i$ be observed for $T$ periods with outcome path $\mathbf y_i\in\mathbb R^T$ that decomposes into a (piecewise-constant) manager-effect path $\mathbf z_i$ and shocks $\mathbf e_i$:
\begin{equation}
\mathbf y_i = \mathbf z_i + \mathbf e_i,
\qquad \mathbb E[\mathbf z_i\mathbf z_i']=??,
\qquad \mathbb E[\mathbf z_i\mathbf e_i']=0,
\qquad \mathbb E[\mathbf e_i\mathbf e_i']=\Sigma.
\end{equation}
The first assumption about the error terms is the standard strict exogeneity assumption, often stated as ``random mobility.'' Every period's shock is mean independent of the entire CEO path. The second assumption allows for arbitrary autocorrelation in shocks over time (e.g., persistent shocks).

For any linear contrast with weights $\mathbf w\in\mathbb R^T$, define $y_i=\mathbf w'\mathbf y_i$, $z_i=\mathbf w'\mathbf z_i$, and $\varepsilon_i=\mathbf w'\mathbf e_i$, resulting in
\begin{equation}
y_i = z_i + \varepsilon_i,
\qquad \mathbb E[z_i\varepsilon_i]=0,
\qquad \mathbb E[z_i^2] = \lambda.
\end{equation}
The last assumption defines the variance of the change in the CEO fixed effect $z_i = z_{i,+3} - z_{i,-1}$ and is typically the object of interest (REFs).

\paragraph{Estimation.} Introduce a design matrix $\mathbf D$ to map rows to CEO spells as
\[
  \mathbf D = \begin{bmatrix}
    1 & 0\\
    1 & 0\\
    0 & 1\\
    0 & 1\\
    0 & 1
  \end{bmatrix},
\]
and the diagonal matrix of spell lengths $\mathbf T$ as
$$
  \mathbf T = \operatorname{diag}(T_1,T_2)=\operatorname{diag}(2,3).
$$
We introduce the notation
$$
  \mathbf P = \mathbf D\,\mathbf T^{-1}\,\mathbf D'
$$
for the projection matrix which converts outcomes to within-spell means. In the example,
\[
  \mathbf P = \begin{bmatrix}
    1/2 & 1/2 & 0 & 0 & 0\\
    1/2 & 1/2 & 0 & 0 & 0\\
    0 & 0 & 1/3 & 1/3 & 1/3\\
    0 & 0 & 1/3 & 1/3 & 1/3\\
    0 & 0 & 1/3 & 1/3 & 1/3
  \end{bmatrix}\quad\Rightarrow\quad
  \mathbf {Py}_i = 
  \begin{pmatrix}
  \frac12 \sum_{t=-2}^{-1} y_{it} \\
  \frac12 \sum_{t=-2}^{-1} y_{it} \\
  \frac13 \sum_{t=0}^{+2} y_{it}\\
  \frac13 \sum_{t=0}^{+2} y_{it}\\
  \frac13 \sum_{t=0}^{+2} y_{it}
  \end{pmatrix}.
\]

Concretely for the example, $\hat z_i$ is a weighted combination of the two spell means that implements $y_{i,+3}-y_{i,-1}$.

Note that $\mathbf P$ is idempotent ($\mathbf P^2=\mathbf P$) and symmetric ($\mathbf P'=\mathbf P$). This means that computing the within-spell mean twice is the same as doing it once.


\paragraph{Naive OLS using estimated effects.} Define the estimated effect for the contrast by
\[
  \hat z_i = \mathbf w'\hat{\mathbf z}_i = \mathbf w'\mathbf P\,\mathbf y_i = z_i + \eta_i,\qquad \eta_i:=\mathbf w'\mathbf P\,\mathbf e_i.
\]
Also define the contrasted outcome
\[
  y_i = \mathbf w'\mathbf y_i = z_i + \varepsilon_i,\qquad \varepsilon_i:=\mathbf w'\mathbf e_i.
\]
Consider the (conceptual) population regression of $y_i$ on $z_i$, $y_i=\beta z_i+u_i$ with $u_i=\varepsilon_i$ and $\mathbb E[z_i u_i]=0$. In practice we do not observe $z_i$ and instead regress $y_i$ on $\hat z_i$. The OLS slope is
\[
  \tilde\beta \,=\, \frac{\Cov(y_i,\hat z_i)}{\Var(\hat z_i)} \,=\, \frac{\beta\Var(z_i)+\Cov(\varepsilon_i,\eta_i)}{\Var(z_i)+\Var(\eta_i)}.
\]
 
\textit{Covariance term.} $\Cov(\varepsilon_i,\eta_i)$ is typically positive because both $\varepsilon_i$ and $\eta_i$ are formed from the same underlying shocks averaged over overlapping windows (short spells and persistent shocks make this more pronounced). This inflates the numerator relative to $\beta\Var(z_i)$.

\textit{Variance term (classical measurement error).} $\Var(\eta_i)>0$ inflates the denominator relative to $\Var(z_i)$, the standard attenuation channel.

Together, the slope $\tilde\beta$ can be biased up or down depending on the balance between the inflated covariance (numerator) and the inflated variance (denominator). With i.i.d. shocks and long, balanced spells, $\Cov(\varepsilon_i,\eta_i)\approx 0$ and $\Var(\eta_i)$ is small, so the attenuation channel dominates but vanishes as sample size grows.

\paragraph{Small-sample nature.} Both components are small-sample phenomena driven by short $T_1$ and $T_2$. As $T_1,T_2\to\infty$, spell means average out shocks so $\Var(\eta_i)\to 0$ and $\Cov(\varepsilon_i,\eta_i)\to 0$, implying $\tilde\beta\to\beta$.

\paragraph{Placebo identification.} We construct placebo spells among firms that do not change CEOs but replicate the observed spell-length patterns and event windows of treated firms while excluding any windows around actual transitions. For placebo spells, $\Var(z)=0$, so
\begin{equation}
\widehat{\Cov}^{\,\text{pl}}(y,\hat z)=\hat A_g,\qquad \widehat{\Var}^{\,\text{pl}}(\hat z)=\hat B_g.
\end{equation}
These identify the bias components nonparametrically. Debiased treated moments and slopes are then
\begin{align}
\widehat{\Cov}^{\,\text{db}}(y,\hat z) &= \widehat{\Cov}^{\,\text{tr}}(y,\hat z) - \hat A_g,\\
\widehat{\Var}^{\,\text{db}}(\hat z) &= \widehat{\Var}^{\,\text{tr}}(\hat z) - \hat B_g,\\
\hat\beta^{\text{db}}_g &= \frac{\widehat{\Cov}^{\,\text{tr}}(y,\hat z)-\hat A_g}{\widehat{\Var}^{\,\text{tr}}(\hat z)-\hat B_g},
\end{align}
pooled across groups with weights if needed. Inference follows from a delta method for ratios or a firm-level bootstrap that resamples within design groups, preserving spell design.

%%%%%%%%%%%%%%%%%%%%
\section{Estimation}\label{sec:est}
%%%%%%%%%%%%%%%%%%%%

Our estimation proceeds in three steps: estimate total factor productivity, recover manager fixed effects through two-way fixed effects regression, and implement placebo-controlled event studies to separate true CEO effects from noise. Each step builds toward separating true CEO effects from the noise that contaminates raw estimates.

Having estimated the placebo-corrected CEO effect, we decompose the variance of TFP into a general component and one depending on the CEO. The share of the CEO dependent component in the total variance provides a quantification of the CEO effect on firm productivity.

\paragraph{Step 1: Estimating Total Factor Productivity.} We estimate total factor productivity (TFP) following standard production function approaches. We estimate a Cobb-Douglas production function with capital, labor, and materials as inputs, controlling for sector-year fixed effects. The residual from this estimation represents log TFP, which we decompose into firm effects ($\lambda_i$), manager effects ($z_m$), and residual productivity shocks ($\epsilon_{it}$):
\begin{equation}
\omega_{it} = \lambda_i + z_m + \epsilon_{it}
\end{equation}
This measure of log TFP contains manager skill, firm effects, and residual productivity. In standard production function estimation, this entire term would be treated as a single TFP measure. Our decomposition separates the manager contribution from other sources of productivity.

\paragraph{Step 2: Recovering Manager Fixed Effects.}

We estimate a two-way fixed effects model with firm and manager fixed effects on log TFP. Our identification relies on the strict exogeneity assumption where the conditional expectation of the error term $\epsilon_{it}$ is zero for a given firm-manager pair, for every t: $E[\epsilon_{it}|\lambda_i, z_m]=0$ for $t=1...T$ This assumption allows for various forms of endogenous mobility but rules out systematic patterns where managers consistently join improving or declining firms. The estimated CEO fixed effects are unbiased, but inconsistent due to the fact that with the sample size $N$ increasing, the length of tenure of CEO-s at a given firm does not increase.

The event study provides a diagnostic test for this identification assumption. Pre-trends in productivity before CEO transitions would suggest that the strict exogeneity assumption is violated. If productivity systematically rises before good CEOs arrive, we worry that the positive trend continues post-transition, violating $E[\epsilon_{it}|z_m, \lambda_i] = 0$ for $\forall t$. Conversely, the absence of pre-trends makes it harder to construct plausible endogeneity stories. While we cannot rule out contemporaneous shocks that coincide exactly with CEO changes (e.g., owners simultaneously firing the CEO and adopting new technology), such precise timing is less plausible than gradual changes that would manifest as pre-trends. Our event studies show no significant pre-trends, supporting but not proving our identification assumption.\footnote{The absence of strong pre-trends in our data contrasts with evidence from \citet{cornelli2013monitoring} showing boards actively monitor and replace CEOs when performance deteriorates in public firms, suggesting our private firm transitions may be less performance-driven. While \citet{jenter2021performance} find 38-55\% of turnovers are performance-induced in U.S. public firms, our private firm setting likely features more random CEO transitions given the absence of market pressures and board oversight.}

The system of fixed effects is identified only within connected components: groups of firms and managers linked through mobility. Two managers can be compared if they worked at the same firm or are connected through a chain of shared CEO-firm relationships. We can estimate $\hat z_m$ for every manager, but only up to a constant term that may vary across connected components. Our largest connected component contains 22,001 managers, enabling comparisons within this network. We normalize the manager effect to be mean zero in the largest connected component.
In TWFE models researchers are typically limited to analysis within the largest connected component of their network. We are not limited by the largest connected component, only by the set of firms which experience a CEO transition. 
As ours is a within firm analysis, the two CEO-s we compare at the event of a CEO transition always belong to the same connected component, and hence their estimated fixed effects are comparable.
\paragraph{Step 3: Placebo-Controlled Event Studies.} Even when $\epsilon$ is orthogonal to $z$, estimated fixed effects contain substantial small-sample noise when manager transitions are infrequent and manager tenures are short.\footnote{Worker-firm fixed effect studies face similar challenges called ``limited mobility bias'' \citep{andrews2008high}. This literature has developed bias-correction methods for the variance estimates of fixed effects \citep{Bonhomme2023-dx, gaure2014correlation}.} A consequence of this small sample bias is that naively estimating the effect of CEO transitions on TFP will likely overestimate the true effect.

To understand the sources of small-sample bias and how we address it, we remove the firm fixed effect from TFP by subtracting the firm average:
\begin{equation}
\Delta\omega_{imt} = \Delta z_{m_{it}} + \Delta\epsilon_{it},
\end{equation}
where $\Delta x_{it} := x_{it} - \frac{1}{N_i}\sum_{\tau} x_{i\tau}$ denotes the deviation of a variable from its within-firm mean. When a firm changes CEOs, the change in log TFP captures both the true skill difference and accumulated noise. The noise component---the average of residual productivity shocks during each manager's tenure---dominates the signal when tenures are short. 

Our solution leverages a simple insight: when CEOs do not change, we still observe variation in log TFP driven purely by noise:
\begin{equation}
\Delta\omega_{imt} = \Delta\epsilon_{it}.
\end{equation}
By applying the estimation procedure to measure the effect of CEO transitions on productivity not only to real CEO transitions but also to placebo CEO transitions, we can measure and filter out the noise component.

We construct placebo CEO transitions to use as control transitions to actual CEO transitions in three steps. First, we identify the set of clean CEO transitions by looking at firms with one CEO per year, where firms have at least two consecutive CEO spells. From around 3.5 Million firm-year observations, we identify 59,954 CEO changes, involving 42,902 firms. On this sample of clean CEO changes we estimate the time-varying hazard of actual CEO changes. Second, we identify all firms with long CEO tenures during which no actual CEO change occurs. Third, we find possible control firms for every CEO transition experienced by a firm by matching on the firm's birth cohort, sector, year of CEO transition, and we also require that the control firms have no CEO change during the [-4,3] interval around the actual CEO transition. Since there is a very large number of possible controls, we take a random sample of all possible controls for any given actual transition. Then, we assign the calendar year of the actual CEO transition to the set of control firms for any given actual CEO transition. The assigned timing of the fake
CEO change follows the empirical hazard, ensuring the mechanical noise properties—averaging
residual productivity over varying tenure lengths—match between actual and placebo groups. This is evidenced in Panel B of Table 2, which shows that the spell-length distribution in the placebo transitions mirrors the distribution of spell-lengths among the actual CEO changes. We have ultimately 676,370 placebo changes of CEO-s which serve as controls for the 59,954 actual changes of CEOs.

As an example, consider a firm born in 1995 which experienced a CEO change in 2000, the CEO stayed until 2005. The event window for the transition is [1995, 2005]. To find the set of placebo CEO changes, we identify the set of all firms which were born in 1995, belong to the same sector as our firm of interest, and had no CEO change between 1995 and 2005. Among all the possible control firms we take a random sample and assign to this random sample a placebo CEO transition for the year 2000, creating two artificial CEO-spells, one from 1995 to 2000 and one from 2001 to 2005.

To obtain the dynamics of TFP within firms that experience a CEO transition, we implement a modified differences-in-differences framework. In particular, we estimate the event study coefficients for every event time $g$ on the set of actual CEO transitions, and estimate analogous event study parameters on the set of placebo CEO transitions. We subtract the estimated coefficients of the placebo from the coefficients on the real transition event time by event time, and plot this difference. Our modified differences-in-differences estimator adapts the \citet{Callaway2021JoLE} estimator for two treatment types. We implement it using the \texttt{xt2treatments} package \citep{Koren2024xt2treatments}. The key innovation is precisely aligning transitions in event time: both actual and placebo changes occur at $t=0$, enabling a clean comparison of dynamics between treated and control firms. 
Formally, we implement the event study around CEO transitions at time $g$, comparing actual changes (treatment) to placebo changes (control):
\begin{equation}
\omega_{imt} = a_i + \gamma_{t-g} + \epsilon_{it}
\end{equation}
The coefficients $\gamma_{t-g}$ capture the evolution of log TFP in event time for actual CEO transitions relative to placebo CEO transitions, where $t-g \in [-4, 3]$ and we normalize $\gamma_{-2} = 0$. [[comment: confirm exact baseline normalization in the estimator scripts; some versions use $t=-1$ as baseline.]]


\paragraph{Variance Decomposition.} To quantify the contribution of CEO transitions to long-term performance differences, we carry out a variance decomposition. The limited mobility bias will affect the firm component of the variance of $\omega_{it}$ \citep{kline2024firm}, and we correct our estimate with the placebo procedure discussed above. In addition to this problem, we face another difficulty: due to cumulative shocks, the variance of productivity mechanically increases as the firm ages. Our method of estimating an unbiased variance is the following.

We take the birth year $b$ of each firm,\footnote{In the estimation, we do not take the birth year but a latter year to get a full year of existence as baseline.} and compute the change in productivity, $\Delta_0\omega_{it}$ between year $t$ and $b$ -- note that with this method we remove firm fixed effects from the dependent variable. For our control firms with a pseudo CEO change this will be equal to the following: 
\begin{equation}
    \Delta_{0}\omega_{it} = \sum_{a=1}^{t-b} \Delta \omega_{i,b+a} 
= \sum_{a=1}^{t-b} \Delta \epsilon_{i,b+a},
\end{equation}
while for treated firms (those with an actual CEO change):
\begin{equation}
    \Delta_{0}\omega_{it} = \Delta_0 z_{mi} + \sum_{a=1}^{t-b} \Delta \omega_{i,b+a} 
= \Delta_0 z_{mi} + \sum_{a=1}^{t-b} \Delta \epsilon_{i,b+a},
\end{equation}
The variance of productivity change for the control and treated groups is
\begin{equation}
\operatorname{Var}(\Delta_{0}\omega_{it} \mid D_{it} = 0) = (t - b)\sigma_{0}^{2}
\end{equation}
\begin{equation}
\operatorname{Var}(\Delta_{0}\omega_{it} \mid D_{it} = 1) = \operatorname{Var}(\Delta_{0}z) + (t - b)\sigma_{1}^{2}
\end{equation}
For the latter formula, we assumed that $\Delta_0z$ is orthogonal to $\Delta_0\epsilon$. 

The key assumption for an unbiased estimator of $\Var(\Delta_0z)$ is that the variance of error terms depends only on firm age and treatment status, but not on the timing of treatment:
\begin{equation}
    \operatorname{Var}(\Delta_{0}\epsilon_{it} \mid D_{it} = 1,\, t = b_{i} + a) = \operatorname{Var}(\Delta_{0}\epsilon_{it} \mid D_{it} = 0,\, t = b_{i} + a)
\end{equation}
The estimation of the unbiased variance is as follows. We first compute the variance (relative to year $b$) as $(\Delta_0\omega - \overline{\Delta_0\omega})^2$, where the latter term is the mean of productivity within firm. We regress the variance on firm age - treatment status interactions and compute the age-adjusted variance by subtracting the estimated age - treatment effects from the actual variance. Finally, we run a simple difference-in-differences regression on the age-adjusted variance, which provides an unbiased estimate of the difference of $\operatorname{Var}(\Delta z)$.
%When it is done, need to place in the multi-CEO version

%%%%%%%%%%%%%%%%%%%%%%%%%%%%%%%%%%%%%
\section{Monte Carlo Design and Interpretation}

We use a simple Monte Carlo to show why second moments are biased and how the placebo correction behaves under empirically relevant complications. The guiding intuition from our design applies: (i) when shocks are i.i.d. and spells are balanced, naive and debiased profiles line up; (ii) with persistent shocks or short/unbalanced spells, spurious pre-trends may appear even if the causal pre-trend is zero; (iii) allowing for a scalar volatility difference across groups preserves identification via placebo moments.

Our starting point is a window with two spells per firm. Placebo firms never actually change CEOs; we split long no-change stretches into the same pre/post spell lengths as in the treated sample and align them in event time. We compute the same statistics in both samples with the same weights and then subtract the placebo quantity. Because covariance is bilinear and variance is additive under strict exogeneity, the subtraction removes the bias term without modeling the full autocovariance structure. If treated firms are more volatile, we estimate a groupwise scalar and multiply the placebo term before subtraction.

The scenarios and figure panels follow the code in src/montecarlo and src/figuremc.do:

\begin{table}[t]
\centering
\caption{Monte Carlo parameters by scenario}
\label{tab:mc_params}
\begin{threeparttable}
\begin{tabular}{l*{6}{c}}
\toprule
 & Baseline & Long Panel & Persistent Errors & Unbalanced Panel & Excess Variance & All Complications \\
\midrule
 Common parameters & \multicolumn{6}{c}{$N_{\text{changes}}=10{,}000$; Control:treated = 9:1; $\sigma_z=0.10$} \\
\addlinespace
 $\rho$ & 0.00 & 0.00 & 0.90 & 0.90 & 0.00 & 0.90 \\
 $\sigma_{\epsilon0}$ & 0.05 & 0.05 & 0.05 & 0.05 & 0.05 & 0.05 \\
 $\sigma_{\epsilon1}$ & 0.05 & 0.05 & 0.05 & 0.05 & 0.075 & 0.075 \\
 Hazard & 0.00 & 0.00 & 0.00 & 0.20 & 0.00 & 0.20 \\
 $T_{\max}$ & 5 & 20 & 5 & 5 & 5 & 5 \\
\bottomrule
\end{tabular}
\begin{tablenotes}[flushleft]\footnotesize
Notes: Baseline parameters from src/montecarlo/params.do; scenario overrides from src/montecarlo/*.do. $\sigma_{\epsilon0}$ is the innovation s.d. for placebo; $\sigma_{\epsilon1}$ for treated. Hazard governs spell lengths (discretized exponential) when positive. $T_{\max}$ truncates spell lengths.
\end{tablenotes}
\end{threeparttable}
\end{table}

- Baseline (params.do): rho = 0 (i.i.d. shocks), sigma_epsilon0 = 0.05, sigma_epsilon1 = 0.05, hazard = 0 (balanced spells), T_max = 5.
- Long Panel (longpanel.do): T_max = 20.
- Persistent Errors (persistent.do): rho = 0.9.
- Unbalanced Panel (unbalanced.do): rho = 0.9 and hazard = 0.2 (spell lengths from a discretized exponential, truncated at T_max).
- Excess Variance (excessvariance.do): sigma_epsilon1 = 0.075 in the treated group (placebo remains at 0.05).
- All Complications (all.do): rho = 0.9, hazard = 0.2, sigma_epsilon1 = 0.075.
}
The figure script imports the scenario-specific CSVs, clips confidence bands for three plotted series (beta0, beta1, dbeta) to [-0.75, 1.75] for legibility, draws per-panel graphs, and combines them into a 2x3 layout.

\begin{figure}[htbp]
\centering
\includegraphics[width=0.95\textwidth]{figure/figuremc.pdf}
\caption{Monte Carlo event studies under six scenarios. Panels: Baseline, Long Panel, Persistent Errors, Unbalanced Panel, Excess Variance, All Complications. Bands are truncated to [-0.75, 1.75] for readability. [[comment: add precise definitions for beta0, beta1, dbeta from the estimator]]}
\label{fig:mc}
\end{figure}

We expect to see exactly zero before and exactly one after the arrival of the CEO in TFP units when the true effect is normalized so that a one-unit increase in the fixed effect raises TFP by one. Other outcomes may have different scales, but the logic is the same. With persistence, short spells, or excess variance in treated firms, the naive profiles will typically display pre-trends and inflated post effects; subtracting the placebo series removes these artifacts.

\section{Corporate Data from Hungary}
%%%%%%%%%%%%%%%%%%%%%%%%%%%%%%%%%%%%%

Hungary provides an ideal setting for studying CEO effects in the universe of private firms. The country offers complete administrative data coverage for all incorporated businesses, spanning 30 years from the transition economy of the 1990s through EU accession in 2004 to the present. CEO registration is mandatory for every registered firm. The coverage enables us to track CEO careers across firms and construct mobility networks necessary for identification.

Our analysis combines two administrative datasets. The firm registry, maintained by Hungarian corporate courts, contains records on all company representatives — individuals authorized to act on behalf of firms. These records include CEOs and other executives with signatory rights, tracked through a temporal database where each entry reflects representation status over specific time intervals. Updates occur when positions change, personal identifiers are modified, or reporting standards evolve. The registry provides names, addresses, dates of birth (from 2010), and mother's names (from 1999), though numerical identifiers exist only from 2013 onward.

The balance sheet dataset contains annual financial reports for all Hungarian firms with double-entry bookkeeping. The data has information on financial variables and some additional information, such as sector of activity, employment, and ownership (state, domestic, foreign). The two datasets cover 1,063,172 firms over 31 years, yielding 9,627,484 firm-year observations before sample restrictions.

Constructing CEO time series across firms poses some challenges. Before 2013, no numerical identifiers existed, requiring entity resolution based on names, addresses, mother's names, and birthdates. We link records across these dimensions to create unique person identifiers, enabling tracking across firms and over time. Matching quality improves after 1999 (when mother's names reporting begins) and 2010 (when birthdates reporting starts), though the 1990s data achieves reasonable coverage through name and address matching. 

Finding CEOs among all individuals from the data requires heuristics since job titles are inconsistently recorded. When explicit ``managing director'' titles exist, we use them directly. For remaining cases, we assume all representatives are CEOs if three or fewer exist at the firm. When more than three representatives are present, we assign CEO status based on continuity with previously identified CEOs. Time spans between appointments are often unclosed or non-contiguous, requiring imputation based on sequential information, assuming representatives remain active if their tenure includes June 21 of each year.

We apply several restrictions to create a sample suitable for productivity analysis. First, we exclude mining and finance sectors due to specialized accounting frameworks and regulatory environments. Second, we drop firms ever having more than 2 simultaneous CEOs to avoid complex governance structures that complicate identification. Third, we exclude firms with more than 10 CEO changes over the sample period (removing observations for unstable firms) to reduce noise from misclassified transitions. Fourth, we remove all state-owned enterprises, as their objectives and constraints differ from private businesses \citep{orban2019inception, shleifer1994politicians}.

We restrict attention to firms that employ at least 5 workers at some point in the firm lifecycle. This filter removes a substantial portion of observations but eliminates shell companies, tax optimization vehicles, and self-employment arrangements masquerading as corporations. The remaining firms represent genuine businesses with meaningful economic activity where management quality affects performance.

We exclude public firms and joint-stock companies from our analysis. The few companies traded on the Budapest Stock Exchange operate under different governance structures, compensation schemes, and disclosure requirements than private businesses. We also exclude cooperatives and other non-standard corporate forms where multiple managing directors share executive authority, as these organizational structures complicate identification of individual CEO effects.

Firms with multiple CEO changes are handled by splitting their history into adjacent transition pairs. After collapsing firm-CEO spells, intermediate spells are duplicated so that a firm with spells 1/2/3 contributes two transitions, 1$\to$2 and 2$\to$3. Each transition defines a pseudo-firm, indexed by a constructed identifier that groups the true firm with the relevant spell window. Fixed effects are estimated separately for these pseudo-firms, so outcomes around different transitions are allowed to have distinct firm intercepts. These pseudo-firms are not treated as independent statistical units. We retain the original firm identifier and cluster standard errors by the true firm. 



\begin{comment}
Jó lenne a mintára csinálni, és nem a populációra
Kellene néhány céges jellemző (emp, TFP) az A panelbe
Kellene az átlag (átlagosan hány CEOja van egy cégnek, és egy CEO átlagosan mennyit van a cégnél)
\end{comment}

Appendix Table \ref{tab:sample} presents the number of firms in our sample relative to the total number of firms in the data for several year along the time series. The number of Hungarian firms increases abruptly during the nineties and also in the first 15 years of 2000s, albeit at a smaller pace.  The result of dropping the small firms results in using about one quarter of all firms. In the first year of study (1992) we have 26 thousand firms which increases to 115 thousand towards the end of the period. Overall, we have over one million firms and almost 10 million firm-years in the sample. The table also shows the number of CEOs each year, which presents a similar pattern; in 1992 we observe 32 thousand CEOs and in 2022 122 thousand. The total number of distinct CEOs is 340 thousand. Appendix Table \ref{tab:industry_stats} shows the industrial distribution of firms and CEOs. The largest sectors are Wholesale-Retail, Telecom and Business Services and Nontradable Services. 

Table \ref{tab:CEO_desc} shows the proportion of firms (and firm-years) by the number of CEOs observed in the firm during the period of observation. Not surprisingly, a majority of firms have only one CEO: 63\% of firms and 80\% of firm-years belong to this category. One quarter of firms had two CEOs while 13\% at least three CEOs. This latter category comprises only 3\% of firm-years. CEO spell lengths follow an exponential distribution with a 20 percent annual hazard rate, implying typical tenures of 3-7 years.


%%%%%%%%%%%%%%%%%
\section{Results}
%%%%%%%%%%%%%%%%% 


%elég 3 tizedesjegy, nem vagyok mink atom fizikusok
%a nobs nem jó, mert három különböző kell a három sorba

\paragraph{CEO Transitions and Productivity Change.} Table \ref{tab:placeholder} shows the average effect of CEO transition on productivity as estimated by the naive OLS regression, the placebo effect and the corrected regression estimate.\footnote{The naive and the pseudo outcome regression are estimated on the treated and placebo samples, respectively. The controls are the not-yet-treated firm years. The corrected regressions are estimated on the merged samples and we use the two-way treatment method discussed in Section \ref{sec:est}.} To start with the effect on the whole sample, TFP increases by 0.5\% around the CEO change. The placebo regression produces precisely measured zeros.\footnote{This is reasonable because here we do not have a reason to believe the estimated effect by simple OLS regression is biased as we do not select on unobservables correlated with productivity.} That is, even without a CEO change, TFP changes somewhat. The placebo controlled estimate equals 0.3\%. 

As most of our firms are family-owned, we look into an interesting CEO transition: when founders relinquish control. For comparison, we also run a separate regression on other CEO changes. When founders are the departing CEO, the estimated effect is 1\%. For the other CEO changes we do not find any change in TFP. This is also a test of our method: when similar CEOs replace each other, the effect is zero, at least on average. On the contrary, when we measure the TFP change between two distinctly different CEOs, we do find an economically and statistically significant effect.\footnote{The reason for a positive effect of founder replacement may be the timing of the change. Founders usually give up their central role in the firm when they find they no longer have enough energy to run the firm.} 



Finally, we analyze perhaps the most interesting question: how the productivity reacts to the arrival of a better or worse CEO than the incumbent? For this, we split the sample into two types of CEO transitions: when the incoming CEO has a higher or lower quality (as measured by the associated fixed effect) than the incumbent CEO. We also split the placebo sample into firms which experience an increase/decline of their TFP around the pseudo CEO change. The estimated coefficients are presented in the last two columns of Table \ref{tab:placeholder} and demonstrate the importance of CEOs and also show how the placebo correction works. Better CEOs than the incumbent increase TFP by 6\% while worse CEOs decrease it by 5\%. This large effect, however, is upward biased, as the placebo regressions demonstrate. The pseudo CEO change also produces a sizable effect of $\pm$2.5\%. Thus, the corrected effect is much smaller: better CEOs increase TFP by 2.7\% while worse CEOs decrease it by 1.8\%. Thus, the difference between a good and a bad CEO is 4.5\%. 

Figure \ref{fig:event_study_main} visualizes the evolution of TFP around the CEO change for four samples: all changes, when the incumbent is the founder of the firm, all other incumbents, and for the period of mature market economy (2004 -- 2022).\footnote{The latter sample is used to make our results more comparable to mature market economies. In the 1990s the economy was changing rapidly as it underwent rapid economic liberalization, transition to market economy, foreign direct investment inflows and large scale privatization. We conduct this robustness check by restricting the sample to post-2004 data following Hungary's EU accession.} On each figure we present the event time regression estimates for the whole sample (black line) and for better and worse CEO replacements (blue and red lines). The estimations leave little pre-trend and we find large swings in TFP. If the incoming CEO is better than the incumbent, TFP increases by 3 -- 4\% while worse CEOs decrease it by about 2\%. New CEOs have an effect immediately on the firm and this proves to be quite stable, as the estimated coefficients do not change much.  



\paragraph{Differential Effects on Inputs.} Figure \ref{fig:event_study_input} presents event studies examining how CEO transitions affect different firm inputs.

Good CEOs have immediate and substantial effects on operational inputs. Material costs increase by 30\% and the wage bill by 13 percent (all effects highly significant). Firms under bad managers experience the opposite: a decline in material and employment costs by about 20\%. These effects appear immediately in the year of CEO transition and persist throughout the post-period, consistent with new CEOs quickly adjusting operational scale.

In contrast, strategic variables show different patterns. Fixed assets exhibit a gradual change under good CEOs and little or no change under bad CEOs. The proportion of firms with intangible assets does not change at all around the CEO transition.

\paragraph{Contribution of CEOs to Firm Productivity.} To assess how important CEOs are for firm performance, we conduct the variance decomposition. As we describe at the end of Section \ref{sec:est}, we face two challenges. The variance is biased and it also depends on firm age. The outcome of our method is presented in Figure \ref{fig:variance_decomp} for TFP in the top row and log revenue in the bottom row. The blue line shows the within-firm change in variance relative to the second year after firm foundation. The blue line on panels A and C show the evolution of this variance around the CEO transition. Before the CEO transition, the variance grows continuously. The transition increases the variance abruptly; in later years it stays relatively stable for TFP and continues growing for revenue. Part of this growth, however, is due to limited mobility bias, and part arises mechanically: as time passes, the chance of receiving shocks increases. To control for these biases, we estimate the same variance on the pseudo transition sample. As the red line shows, this also increases in time. Panels B and D of the figure show the evolution of the estimated variance in the two samples by firm age. 

Our estimate of the contribution of CEO change to the total variance of TFP is the difference between the two estimates, and we quantify it in Appendix Table \ref{tab:var_share}. The table presents the total variance, the unadjusted contribution of CEO change and the adjusted contribution. We analyze only one CEO change, and the contribution depends on the length of the period measured (the longer the period, the more the variance increases for reasons unrelated to CEO change). In the first 10 years of existence, the uncorrelated share of CEO change in the variance is 62\%. Our correction decreases this number to 29\%.




%%%%%%%%%%%%%%%%%%%%
\section{Conclusion}
%%%%%%%%%%%%%%%%%%%%

This paper estimates the contribution of CEOs to firm productivity by exploiting a unique administrative dataset covering the entire population of Hungarian private firms and their CEOs from 1992 to 2022. The novelty of the data lies in its unprecedented scope and completeness, allowing the study of CEO effects not only in large firms but crucially in small and medium-sized enterprises that dominate every economy. The combination of the dataset with a theoretically grounded model that distinguishes owner-controlled capital decisions from CEO-controlled operational inputs allows for a more precise attribution of productivity effects. 

The paper develops a new placebo-controlled event study design that overcomes limited mobility bias, which contaminates studies using two fixed effects.  By creating matched placebo CEO transitions in firms without actual leadership changes, the method effectively separates true CEO skill effects on productivity from mechanical noise. Empirically, the findings reveal that the true causal effect of CEO quality on firm productivity is economically meaningful but notably smaller than raw correlations suggest, explaining about 7–8 percent of productivity variation. 

An alternative way to deal with the noise problem is to use observable manager characteristics as measures of skill. Observable characteristics such as education and work experience \citep{DePirro2025}, foreign name as a proxy for international exposure \citep{Koren2023expat}, and the selectiveness of entry cohorts \citep{koren2024managers} offer more reliable, though narrower, measures of specific dimensions of managerial quality. These observables capture only partial aspects of CEO ability but avoid the mechanical noise that contaminates fixed effects estimates.

\clearpage
\bibliographystyle{apalike}
\bibliography{../../lib/references}

\clearpage
\appendix
\section{Online Appendix: Additional Tables and Figures}
\renewcommand{\thefigure}{A\arabic{figure}}
\renewcommand{\thetable}{A\arabic{table}}
\setcounter{figure}{0}
\setcounter{table}{0}




%Benne maradhat a populáció, de kellene a minta is. Esetleg kidobhatjuk a firms oszlopot, és készítsük el a táblát a populációra és a mi mintánkra (Population: firm-year CEO, Sample: firm-year CEO, Surplus Share)


%\begin{table}[htbp]\centering
\def\sym#1{\ifmmode^{#1}\else\(^{#1}\)\fi}
\caption{Surplus Function Estimation Results}
\begin{tabular}{l*{6}{c}}
\toprule
                    &\multicolumn{1}{c}{(1)}&\multicolumn{1}{c}{(2)}&\multicolumn{1}{c}{(3)}&\multicolumn{1}{c}{(4)}&\multicolumn{1}{c}{(5)}&\multicolumn{1}{c}{(6)}\\
                    &\multicolumn{1}{c}{Revenue}&\multicolumn{1}{c}{EBITDA}&\multicolumn{1}{c}{Wagebill}&\multicolumn{1}{c}{Materials}&\multicolumn{1}{c}{Revenue}&\multicolumn{1}{c}{Revenue}\\
\midrule
Fixed assets (log)  &       0.323\sym{***}&       0.325\sym{***}&       0.287\sym{***}&       0.375\sym{***}&       0.290\sym{***}&       0.295\sym{***}\\
                    &     (0.001)         &     (0.001)         &     (0.001)         &     (0.002)         &     (0.001)         &     (0.004)         \\
\addlinespace
Has intangible assets&       0.268\sym{***}&       0.163\sym{***}&       0.267\sym{***}&       0.317\sym{***}&       0.205\sym{***}&       0.255\sym{***}\\
                    &     (0.003)         &     (0.003)         &     (0.003)         &     (0.004)         &     (0.003)         &     (0.010)         \\
\addlinespace
Founding owner      &      -0.079\sym{***}&      -0.051\sym{***}&      -0.017\sym{***}&      -0.099\sym{***}&      -0.010\sym{***}&      -0.015\sym{*}  \\
                    &     (0.005)         &     (0.005)         &     (0.005)         &     (0.006)         &     (0.003)         &     (0.008)         \\
\addlinespace
Non-founding owner  &      -0.001         &      -0.000         &      -0.003         &      -0.006         &      -0.011\sym{***}&      -0.017\sym{*}  \\
                    &     (0.007)         &     (0.006)         &     (0.007)         &     (0.009)         &     (0.004)         &     (0.009)         \\
\addlinespace
Foreign owned       &       0.025\sym{**} &       0.018         &       0.064\sym{***}&       0.010         &       0.022\sym{*}  &       0.033         \\
                    &     (0.012)         &     (0.012)         &     (0.012)         &     (0.015)         &     (0.011)         &     (0.025)         \\
\addlinespace
State owned         &       0.148\sym{***}&       0.047\sym{*}  &       0.448\sym{***}&       0.146\sym{***}&       0.062\sym{**} &       0.050         \\
                    &     (0.029)         &     (0.028)         &     (0.029)         &     (0.032)         &     (0.026)         &     (0.059)         \\
\midrule
Observations        &     3004184         &     2326192         &     2949024         &     3059662         &     2967233         &      374084         \\
\bottomrule
\multicolumn{7}{l}{\footnotesize Standard errors in parentheses}\\
\multicolumn{7}{l}{\footnotesize All models include firm fixed effects, industry-year fixed effects, and a step function for firm age.}\\
\multicolumn{7}{l}{\footnotesize Outcome variables are log-transformed. Models (5) and (6) include quadratic controls for CEO age and tenure.}\\
\multicolumn{7}{l}{\footnotesize Model (6) restricts to largest connected component of CEO-firm network.}\\
\multicolumn{7}{l}{\footnotesize \sym{*} \(p<0.10\), \sym{**} \(p<0.05\), \sym{***} \(p<0.01\)}\\
\end{tabular}
\end{table}




\end{document}








\appendix
\section{Online Appendix: Additional Tables and Figures}
\renewcommand{\thefigure}{A\arabic{figure}}
\renewcommand{\thetable}{A\arabic{table}}
\setcounter{figure}{0}
\setcounter{table}{0}

\subsection{Production Manager Autonomy in Family Firms}

%\input{table/tableA0}

\subsection{Industry Breakdown}

%

%\begin{table}[htbp]\centering
\def\sym#1{\ifmmode^{#1}\else\(^{#1}\)\fi}
\caption{The revenue function by sector}
\begin{tabular}{l*{4}{c}}
\hline\hline
                    &\multicolumn{1}{c}{(1)}&\multicolumn{1}{c}{(2)}&\multicolumn{1}{c}{(3)}&\multicolumn{1}{c}{(4)}\\
                    &\multicolumn{1}{c}{Manufacturing}&\multicolumn{1}{c}{Wholesale, Retail, Transportation}&\multicolumn{1}{c}{Telecom and Business Services}&\multicolumn{1}{c}{Nontradable services}\\
\hline
Tangible and intangible assets (log)&       0.302\sym{***}&       0.262\sym{***}&       0.242\sym{***}&       0.212\sym{***}\\
                    &     (0.003)         &     (0.002)         &     (0.002)         &     (0.002)         \\
[1em]
Intangible assets share&       0.011         &      -0.010         &      -0.065\sym{***}&      -0.027\sym{*}  \\
                    &     (0.025)         &     (0.014)         &     (0.012)         &     (0.015)         \\
[1em]
Foreign owned       &       0.046\sym{*}  &       0.011         &       0.088\sym{***}&      -0.014         \\
                    &     (0.023)         &     (0.015)         &     (0.022)         &     (0.014)         \\
[1em]
State owned         &       0.083\sym{*}  &      -0.029         &      -0.133\sym{*}  &       0.006         \\
                    &     (0.048)         &     (0.042)         &     (0.068)         &     (0.040)         \\
\hline
Observations        &      783394         &     1978400         &     1280504         &     1804551         \\
\hline\hline
\multicolumn{5}{l}{\footnotesize Controls: firm-CEO-spell fixed effects; industry-year fixed effects.}\\
\end{tabular}
\end{table}



\subsection{Manager Skill Distributions}

\begin{figure}[htbp]
\centering
\begin{minipage}{0.48\textwidth}
\centering
\includegraphics[width=\textwidth]{figure/manager_skill_within.pdf}
\end{minipage}
\hfill
\begin{minipage}{0.48\textwidth}
\centering
\includegraphics[width=\textwidth]{figure/manager_skill_connected.pdf}
\end{minipage}
\caption{Manager Skill Distributions}
\label{fig:manager_skills_appendix}
\footnotesize
Notes: Panel A shows the distribution of within-firm manager skill variation for firms with multiple CEOs. Panel B shows the distribution of manager skills in the largest connected component of managers. Both distributions show manager skills in log points after normalization and scaling.
\end{figure}


% Additional tables and figures to be inserted here as needed

\subsection{Variance-Based Evidence for CEO Heterogeneity}



Figure \ref{fig:variance_event_study} presents complementary evidence using the variance of log TFP around CEO transitions. Under our framework, if CEO changes introduce real heterogeneity in managerial quality, the cross-sectional variance of outcomes should increase at the transition point—some firms get better CEOs, others get worse ones. In contrast, pure noise or firm-specific trends would not systematically alter variance.

The variance analysis shows that actual CEO transitions are associated with increased dispersion in outcomes, while placebo transitions show no such effect. This provides model-free evidence that CEO transitions introduce real heterogeneity in firm performance, supporting our identification strategy.


\subsection{Treatment Effects on Owner vs Manager Variables}

{
\def\sym#1{\ifmmode^{#1}\else\(^{#1}\)\fi}
\begin{tabular}{l*{3}{c}}
\hline\hline
                    &\multicolumn{1}{c}{(1)}&\multicolumn{1}{c}{(2)}&\multicolumn{1}{c}{(3)}\\
                    &\multicolumn{1}{c}{Fixed assets (log)}&\multicolumn{1}{c}{Has intangible assets}&\multicolumn{1}{c}{Foreign owned}\\
\hline
Better CEO          &       0.171         &       0.063\sym{*}  &       0.017         \\
                    &     (0.131)         &     (0.035)         &     (0.021)         \\
\hline
Observations        &        7801         &        7801         &        7801         \\
\hline\hline
\end{tabular}
}


{
\def\sym#1{\ifmmode^{#1}\else\(^{#1}\)\fi}
\begin{tabular}{l*{2}{c}}
\hline\hline
                    &\multicolumn{1}{c}{(1)}&\multicolumn{1}{c}{(2)}\\
                    &\multicolumn{1}{c}{Wagebill (log)}&\multicolumn{1}{c}{Materials (log)}\\
\hline
Better CEO          &       0.251\sym{***}&       0.335\sym{***}\\
                    &     (0.045)         &     (0.047)         \\
\hline
Observations        &       58025         &       58025         \\
\hline\hline
\end{tabular}
}

