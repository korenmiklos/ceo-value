\documentclass[11pt,a4paper]{article}
\usepackage[utf8]{inputenc}
\usepackage[T1]{fontenc}
\usepackage{amsmath,amsfonts,amssymb}
\usepackage{apacite}
\usepackage{natbib}
\usepackage{graphicx}
\usepackage{booktabs}
\usepackage{threeparttable}
\usepackage{array}
\usepackage{url}
\usepackage{hyperref}
\usepackage[margin=2.5cm]{geometry}
\usepackage{setspace}
\usepackage{comment}
\usepackage{outlines}
\onehalfspacing

\newcommand{\Var}{\text{Var}}
\newcommand{\Cov}{\text{Cov}}

% Define \sym command for significance stars from esttab
\newcommand{\sym}[1]{{#1}}

\title{How Much Do CEOs Matter? Correcting for Small-Sample Bias in Fixed-Effect Estimates\thanks{Project no. 144193 has been implemented with the support provided by the Ministry of Culture and Innovation of Hungary from the National Research, Development and Innovation Fund, financed under the KKP\_22 funding scheme. This project was funded by the European Research Council (ERC Advanced Grant agreement number 101097789). Telegdy received support from the Hungarian Scientific Research Fund – OTKA, contract number 143346. The views expressed in this research are those of the authors and do not necessarily reflect the official view of the European Union or the European Research Council.}}

\author{Miklós Koren\thanks{Central European University, ELTE Centre for Economic and Regional Studies, CEPR and CESifo. E-mail: korenm@ceu.edu} \\
        Krisztina Orbán\thanks{Monash University. E-mail: krisztina.orban@monash.edu} \\
        Álmos Telegdy\thanks{Corvinus University of Budapest. E-mail: almos.telegdy@uni-corvinus.hu}}

\date{October 2025}

\begin{document}

\maketitle
\thispagestyle{empty}

\begin{abstract}
How much do individual CEOs contribute to firm performance? Studies estimating CEO fixed effects systematically overstate their importance due to small-sample bias. Even when mean effects are unbiased, short tenures make second moments---variances and covariances---severely biased. We develop a placebo-controlled method that constructs matched control transitions in non-changing firms, recovering bias terms without modeling the full error process. The method provides difference-in-differences debiasing for second moments in event studies. Monte Carlo experiments confirm the method removes bias under persistent shocks, short tenures, and unbalanced panels. Applying it to 60,000 CEO transitions in Hungarian firms (1992--2022), we find naive estimates overstate CEO contributions twofold: corrected estimates explain 20--30 percent of revenue variance versus 40--60 percent naively. Debiased estimates reveal immediate, persistent effects of CEO arrival on revenue and exporting with no spurious pre-trends.
\end{abstract}

\textbf{Keywords:} CEO value, CEO-Firm Fixed Effects, Bias Correction

\textbf{JEL Classification:} C23, D24, G34

\clearpage
\setcounter{page}{1}

%%%%%%%%%%%%%%%%%%%%%%
\section{Introduction}
%%%%%%%%%%%%%%%%%%%%%%

The impact of individual CEOs on business practices and firm performance is a widely studied question in economics, corporate finance, and management. In absence of direct measures of CEO characteristics relevant for firm outcomes, studies often include CEO fixed effects to capture variations in latent CEO skills \citep{Bertrand2003-io, crossland2011differences, quigley2015has}. 

Most applied work uses second moments of the estimated fixed effects to understand the importance of CEOs in shaping firm behavior. Variance decompositions reveal how much CEOs contribute to the variance of firm outcomes. Firm attributes, such as profitability \citep{mackey2008effect} and risk \citep{schoar2024effect}, are often regressed on the estimated CEO effect, where the estimated coefficient contains the covariance between the dependent variable and the fixed effect. Finally, event studies are used to assess how outcomes evolve before and after CEO transitions to test for pre-trends and to understand post-transition dynamics \citep{schoar2024effect}. However, both the variance and the covariance of estimated fixed effects are biased upwards by small-sample noise even under credible identification assumptions
\citep{andrews2008high,gaure2014correlation,Bonhomme2023-dx}, which leads to biased second moments and distorts inferences about CEO effects. The bias arises because fixed effects are only T-consistent and not N-consistent: adding more firms does not eliminate noise in individual CEO estimates when each CEO has short tenure at few firms. The bias also introduces spurious pre-trends and distorts post-transition dynamics in event studies when errors are autocorrelated. This spurious pre-trend problem is particularly concerning for applied researchers who use pre-trend tests to validate identification strategies. When mechanical bias generates pre-trends, researchers may mistakenly reject valid designs. Our placebo-controlled method helps distinguish mechanical bias from true violations of exogeneity.

Correcting for small-sample bias is challenging. Existing methods require parametric models for the error covariance structure \citep{andrews2008high,Bonhomme2019-xi,Bonhomme2023-dx,kline2024firm} or leave-one-out adjustments \citep{kline2020leave}. These approaches are impractical in CEO markets where most managers work at only one firm and mobility is limited. Parametric methods require correctly specifying and estimating complex autocorrelation patterns, while leave-one-out fails when individual managers lack sufficient alternative comparison paths.\footnote{Another difference in our application is that the major focus in the labor literature is on understanding the estimated firm effects, whereas we are primarily interested in CEO effects. We also include firm fixed effects, but remove it by differencing or demeaning, so we cannot comment on the correlation between firm and worker (CEO) effects.}

%% i feel like we say too much about leave-one-out, maybe shorten

We propose a placebo-controlled correction method.\footnote{This step in the spirit of \cite{fitza2014use} and \cite{jarosiewicz2023revisiting}.} We match CEO-changing firms with non-changing firms from the same sector and cohort, then assign fake CEO transitions at the same calendar year. Because placebo firms never change CEOs, any variance or covariance in their estimated CEO effects is pure bias. Under reasonable assumptions, the placebo group measures the exact bias in variance, covariance, and event-study dynamics. Differencing treated and placebo second moments yields debiased estimates without modeling the full error structure. This approach sidesteps both the parametric modeling requirements and the limited-mobility problem that plague existing methods.

Our method relies on two key assumptions on shocks. First we assume strict exogeneity: error terms are mean independent of the CEO path for all time periods. This assumption ensures that the estimated CEO effects are unbiased. Second, we assume that the autocovariance of errors is the same between treated and control firms, up to a scalar multiplier. We allow for treated firms to be more volatile than control firms, but we assume that the temporal correlation structure of shocks is identical in the treated and control groups. We call this assumption ``proportional autocovariance.'' This assumption is similar in spirit to the parallel trends assumption in difference-in-differences designs. Our method can be viewed as difference-in-differences for second moments: we difference out bias by comparing treated and control firms' variances and covariances. Other than these two assumptions, we make no further restrictions on the error process, the number and length of CEO spells, or the heterogeneity of CEO effects. 

We illustrate our method in Monte Carlo experiments across three broad scenarios: a baseline with balanced panels and independent shocks, extension to persistent shocks, and allowing for a difference up to a scalar of the error structure between the treated and control groups. In the baseline, naive estimates of variance and covariance are inflated by a constant factor. Under the other two scenarios, spurious pre-trends emerge (despite that the true pre-trend is zero by construction). When we allow for persistent shocks and a proportional difference between the variance-covariance matrix of the treated and control groups, the naive variance of the CEO effect is upward biased by a factor of 2.5 and pretrends become severe. In all scenarios, our placebo-controlled debiasing recovers the true variance, covariance, and event-study dynamics.

We apply the method to the universe of Hungarian private firms from 1992 to 2022, examining revenue dynamics around 60,000 CEO transitions. Naive estimates suggest that CEO effects explain 40--58\% of revenue variance, which decreases to 20--35\% after debiasing the variance. The debiased event studies reveal no pre-trends in revenue, confirming that pre-trends in naive estimates are entirely spurious. These spurious pre-trends would lead researchers to question their identification strategy, but our method reveals they are purely mechanical artifacts. We also find that export behavior is correlated with CEO quality, with a 10 log points better CEO implying about 0.6 percentage points increase in the probability of exporting.

%literature

Our work connects to the broader literature on CEOs and firm performance. Evidence from public sector organizations suggests modest manager effects \citep{fenizia2022managers, janke2024role}, while studies of family firms find larger impacts when professional managers replace family members \citep{bennedsen2007inside, sraer2007performance}. Our results for private firms fall between these extremes: CEOs matter, but less than raw correlations suggest.

Methodologically, our paper builds on the two-way fixed effects literature in labor economics that decomposes wages into worker and firm components \citep{Abowd1999Econometrica, Card2018JoLE}. These studies face similar challenges from limited mobility creating small-sample bias \citep{andrews2008high} and have developed bias-correction methods \citep{Bonhomme2023-dx, gaure2014correlation}. We adapt this framework to the CEO-firm setting but add placebo controls to separate signal from noise. This approach is valuable when studying managers who, unlike workers, have few observations per individual, making traditional bias-correction methods less effective. In addition to debiasing variance and covariance levels studied in the labor literature, we extend the analysis to event-study dynamics. We show that bias not only inflates second moments but also generates spurious pre-trends that can mislead applied researchers about identification validity. Recent work has documented apparently increasing CEO effects over time \citep{quigley2015has}, but these studies do not account for the mechanical noise we identify. \citet{lippi2014corporate} find that concentrated ownership in Italian firms distorts executive selection and reduces productivity by 10\%, providing motivation for our framework separating owner and CEO decisions.

%%%%%%%%%%%%%%%%%%%%
\section{The Econometric Problem}
%%%%%%%%%%%%%%%%%%%%

Let firm $i$ in year $t$ have outcome
\begin{equation}\label{eq:model1}
  y_{it} = \alpha_i + z_{m(i,t)} + e_{it},\qquad \mathbb E[e_{it}|\{z_{is}\}_{s=1}^T]=0,
\end{equation}
where $\alpha_i$ is a firm fixed effect, $z_{m(i,t)}$ collects the CEO effect at time $t$ and $e_{it}$ is a shock. We assume that $z_{m(i,t)}$ is piecewise constant, changing only when the CEO changes and that $e_{it}$ is mean independent of the CEO path for all $s$ (``strict exogeneity'').

This two-way fixed effects model can be estimated by least squares, or, if we are uninterested in firm fixed effects, by differences in differences. Under the strict exogeneity assumption, both the least-squares dummy variable (LSDV) estimator and the difference-in-differences estimator of CEO effects are unbiased. Equation \eqref{eq:model1} can be rewritten as a dummy-variable regression 
$$
y_{it} = \sum_{j}\alpha_j D_{i=j} + \sum_{n} z_n D_{m(i,t)=n}  + e_{it},
$$
and the first-order condition for the OLS estimator $\hat z_n$ is
$$
\sum_{i,t:m(i,t)=n} (y_{it} - \hat\alpha_i - \hat z_n) =0,
$$
or 
$$
\hat z_n = \frac{1}{T_n} \sum_{i,t:m(i,t)=n} (y_{it} - \hat\alpha_i) = z_n + \frac{1}{T_n} \sum_{i,t:m(i,t)=n} e_{it},
$$
where $T_n$ is the number of observations with CEO $n$. The estimator is unbiased because $\mathbb E[\hat z_n|z_n] = z_n$. The estimator, however, is only consistent as $T_n\to\infty$, that is, as the number of observations per CEO grows large. In practice, many CEOs have short tenures, so $\hat z_n$ contains substantial noise.\footnote{The mean CEO tenure (standard deviation) is 13.8 (1.7) years in US small and medium-sized companies \citep{simsek2007ceo}, 8.1 (5.8) years in US public companies \citep{brookman2009ceo}. The large standard deviations demonstrate that many CEOs have only a few years of tenure.} 

The small-sample error contaminates all second moments of $\hat z_n$, biasing estimates of variances, covariances, regression slopes, and correlation coefficients. Because almost every applied question uses second moments (How much do CEOs contribute to firm variance? Are better CEOs associated with higher investment? Do outcomes improve after CEO arrival?), the small-sample noise in first moments propagates into bias in the quantities researchers care about.\footnote{A well-known example is ``limited mobility bias'' \citep{andrews2008high}, which states that in a two-way fixed effects model of worker and firm effects, the variance of both effects is biased upwards and their correlation is biased downwards.}

Consider three common applied use cases:

\textit{Variance decompositions.} Researchers estimate $\Var(\hat z_m)$ to quantify the importance of CEO heterogeneity. But $\Var(\hat z_m) = \Var(z_m) + \Var(\text{noise})$, systematically overstating the role of CEOs.

\textit{Covariances with other variables.} Researchers regress firm outcomes or policies on $\hat z_m$ to understand mechanisms. But $\Cov(\hat z_m, y_{it})$ includes spurious correlation between the noise component of $\hat z_m$ and $y_{it}$.

\textit{Event studies.} Researchers examine outcome dynamics around CEO transitions by contrasting post-transition and pre-transition averages. These contrasts embed noise that can generate spurious pre-trends and distorted post-transition dynamics, even when the true causal effect has no pre-trend.

\paragraph{Bias terms.} We formalize the bias for a typical event-study contrast. Consider firms with a CEO transition, where we observe $T_1$ periods before the transition (under the old CEO) and $T_2$ periods after (under the new CEO). The contrast of interest compares outcomes from the post period to outcomes from the pre period, for example, the change from the year before transition ($t=-1$) to three years after ($t=+3$). 

For a general linear contrast, let $\Delta \hat z_i = \sum_s w_s \hat z_{is}$ denote the estimated change in CEO effects for transition $i$, and $\Delta y_i = \sum_s w_s y_{is}$ the corresponding change in outcomes. We assume that the contrast weights sum to zero , so that under strict exogeneity, the mean change in outcomes would equal the mean change in CEO effects: $\mathbb E[\Delta y_i|\Delta z_i] = \Delta z_i$. This holds for difference-in-differences estimators ($w_s = 1/T_{i+}$ when $s$ is in the post period and $w_s = -1/T_{i-}$ when $s$ is in the pre period), event-study estimators (such as $w_{+3}=1$, $w_{-1}=-1$), and LSDV estimators ($w_s = 1-T_{i+}/T_i$ for $s$ in the post period and $w_s = -T_{i+}/T_i$ for $s$ in the pre period).

In applications with heterogeneous spell lengths and event windows, firms can be grouped into design groups $g=1,\ldots,G$, each with $N_g$ transitions. The sample covariance and variance are
\begin{align}
\widehat{\Cov}(\Delta y_i,\Delta \hat z_i) &= \sum_{g=1}^G \frac{N_g}{N} \widehat{\Cov}_g(\Delta y_i,\Delta \hat z_i),\\
\widehat{\Var}(\Delta \hat z_i) &= \sum_{g=1}^G \frac{N_g}{N} \widehat{\Var}_g(\Delta \hat z_i).
\end{align}
Under the model assumptions, these moments have the expectation
\begin{align}
\mathbb E\widehat{\Cov}(\Delta y_i,\Delta \hat z_i) &= 
  \sum_{g=1}^G \frac{N_g}{N} (\lambda_g + A_g) = \Bar\lambda + A,\\
\mathbb E\widehat{\Var}(\Delta \hat z_i) &= 
\sum_{g=1}^G \frac{N_g}{N} (\lambda_g + B_g) = \Bar\lambda + B,
\end{align}
where $\lambda_g = \Var_g(\Delta z_i)$ is the variance of true CEO effect changes, $A_g$ is the covariance bias, and $B_g$ is the variance bias in group $g$. Averaging over groups, we get $\Bar\lambda$ as the average variance of true CEO effect changes across groups (the object of interest), and $A$ and $B$ are \emph{bias terms} that depend on the spell-length distributions, event-window designs, the autocovariance structure of shocks, and the relative size of groups. Appendix A derives these bias terms formally.

The bias terms have four key properties. First, $A_g$ and $B_g$ are weighted sums over variances and covariances of shocks within each spell, with weights determined by the estimation design. Second, when shocks are independent and identically distributed and spells are long, $A_g\approx 0$ and $B_g\approx 0$. In this ideal case, the bias vanishes asymptotically. Third, when spells are short or shocks are persistent, $A_g$ and $B_g$ can be large, severely distorting both variances and covariances. This is the empirically relevant case in many CEO studies, where tenures are often brief and performance shocks are serially correlated. Fourth, and most important for our method, the bias terms are the same up to a scalar multiplier for any two samples that have identical spell-length distributions and event-window designs, even if one sample has no true CEO effects. This last property is the foundation of our placebo-controlled debiasing.

The key insight is that the bias depends on the \emph{design}: the spell lengths (how long each CEO is observed) and the autocorrelation structure of shocks. If we can find a sample with the same design but no true CEO effects, then the variance and covariance in that sample identify the bias terms. 

\section{Placebo-Controlled Debiasing} 

Our approach constructs placebo CEO transitions to directly identify and remove the bias terms $A$ and $B$ without modeling the full autocovariance structure of shocks.

\paragraph{Placebo construction.} For each treated firm that experiences a CEO transition, we match control firms that do \emph{not} change CEOs but have similar characteristics (birth cohort, sector, etc.). We assign these control firms a fake CEO transition at the same calendar year as the treated transition, creating artificial pre and post spells of the same length as the treated transition.

For placebo transitions, there is no true CEO effect change: $\Delta z_i = 0$ for all placebo firms. Yet the placebo sample has the same spell-length distribution and event-window design as the treated sample. Therefore, the estimated covariances and variances in the placebo group recover the bias terms under the \emph{proportional autocovariance} assumption.

\paragraph{Proportional autocovariance assumption.} The placebo and treated groups may differ in volatility. We assume that the autocovariance structure of shocks is the same across groups up to a scalar: 
\begin{equation}
  \mathbb{E}^{\text{tr}}[e_{it} e_{is}] = c \cdot \mathbb{E}^{\text{pl}}[e_{it} e_{is}] \quad \text{for all } t,s,
\end{equation}
for some constant $c>0$. Under this assumption, which we call \emph{proportional autocovariance}, the covariance and variance in the placebo group can be written as 
\begin{equation}
\widehat{\Cov}_g^{\,\text{pl}}(\Delta y_i,\Delta \hat z_i) \xrightarrow{p} \frac{1}{c} A_g,\qquad \widehat{\Var}_g^{\,\text{pl}}(\Delta \hat z_i) \xrightarrow{p} \frac{1}{c} B_g.
\end{equation}
After obtaining an estimate for the \emph{excess variance factor} $c$, we can scale the placebo moments accordingly,  constructing the bias estimates.

\paragraph{Debiased moments.} Subtracting the rescaled placebo moments from the treated moments yields debiased estimates:
\begin{align}
\widehat{\Cov}^{\,\text{db}}(\Delta y_i,\Delta \hat z_i) &= \widehat{\Cov}^{\,\text{tr}}(\Delta y_i,\Delta \hat z_i) - \hat c\cdot\widehat{\Cov}^{\,\text{pl}}(\Delta y_i,\Delta \hat z_i),\\
\widehat{\Var}^{\,\text{db}}(\Delta \hat z_i) &= \widehat{\Var}^{\,\text{tr}}(\Delta \hat z_i) - \hat c\cdot\widehat{\Var}^{\,\text{pl}}(\Delta \hat z_i).
\end{align}
For nonlinear transformations such as regression slopes or correlations, we first debias the underlying variances and covariances, then form the ratio:
\begin{equation}
\hat\beta^{\text{db}} = \frac{\widehat{\Cov}^{\,\text{db}}(\Delta y_i,\Delta \hat z_i)}{\widehat{\Var}^{\,\text{db}}(\Delta \hat z_i)}.
\end{equation}
The placebo-control approach has several advantages. First, it requires no parametric assumptions about the shock process beyond proportional autocovariance. Second, it handles heterogeneous spell lengths and event windows automatically by matching the design between treated and placebo groups. Third, it extends naturally to dynamic event studies by debiasing period-by-period coefficients. Inference follows from standard methods: clustered standard errors for moments, delta method for ratios, or firm-level bootstrap that resamples within design groups.

The excess variance term $c$ can be estimated by comparing the overall variances of outcomes in treated and control groups in an event window where no CEO change occurs. In practice, we use a minimum distance estimator that matches various second moments before the arrival of a new CEO with the restriction that treated moments are $c$ times placebo moments. 

%%%%%%%%%%%%%%%%%%%%%%%%%%%%%%%%%%%%%%%%%%%%%%%
\section{Monte Carlo Design and Interpretation}
%%%%%%%%%%%%%%%%%%%%%%%%%%%%%%%%%%%%%%%%%%%%%%%

We use a simple Monte Carlo to show why second moments are biased and how the placebo correction behaves under empirically relevant complications. Our starting point is a window with two spells per firm. Placebo firms never actually change CEOs; we split long no-change stretches into the same pre/post spell lengths as in the treated sample and align them in event time. We compute the same statistics in both samples with the same weights and then subtract the placebo statistic from the treated statistic.

Before turning to specific parameter choices, we motivate the six scenarios from econometric first principles. Bias in second moments depends on two pillars: (i) the shock process (the autocovariance matrix \(\Sigma\)) and (ii) the spell design encoded by the event-time design matrix. Persistent shocks (AR(1) with \(\rho>0\)) can create spurious pre-trends and distort dynamic slopes because estimated effects embed noise that is serially correlated. Short or uneven (unbalanced) spells amplify small-sample noise because within-spell means are computed over few observations; this inflates \(\Var(\hat z)\) and, with persistence, also inflates the covariance component. Our scenarios therefore vary only along these two theoretically salient dimensions (persistence and spell length/design), plus a simple proportional variance scaling (excess variance in treated groups). We intentionally keep the rest of the setup simple so that each complication can be read as a clean stress test of the placebo debiasing.

Table~\ref{tab:mc_params} reports six Monte Carlo calibrations with various parameters. The baseline is very simple: balanced panels and i.i.d. (uncorrelated) error terms. Across scenarios we focus on five statistics that are widely used in applications: (i) $\sigma^2(\Delta \hat z)$: the variance of the change in the CEO fixed effect when the firm switches from the first to the second CEO; (ii) $ \mathrm{Cov}(\Delta y_2, \Delta \hat z)$: the covariance of the revenue change at two years after the CEO change with the change in the manager fixed effect; (iii) $ R^2$: the fraction of revenue variance explained by the CEO effect change; (iv) $\hat \beta_2$: an event-study slope two full years after the CEO change; and (v)  $\hat \beta_{-2}$: a pre-trend coefficient two years before the change.

In the baseline, the naive variance is upward biased, while the debiased variance of the CEO effect is very close to the true value of 1.00. The covariance of the revenue change at year $+2$ with the CEO effect is also upward biased in the naive estimate, but by the same proportion as the variance. This proportional inflation occurs because when shocks are independent and identically distributed, the bias in both the numerator and denominator of the regression slope scales identically. As a consequence, the naive regression coefficient $\hat\beta_2$ is unbiased even though both its components are biased. This is a knife-edge case that holds only when $\rho$ is exactly zero. The debiased estimates recover the true covariance and the true slope of 1.00. Even in this favorable scenario, however, the $R^2$ remains upward biased because it is a nonlinear transformation of second moments. The long-panel scenario (spells of length 20) shows that all biases essentially disappear; this is a small-sample phenomenon in short panels.

With persistent errors ($\rho=0.9$), the naive variance is severely upward biased (by about 100\% in our calibration), and spurious pre-trends emerge: before the manager arrives, the naive estimate suggests a negative effect. The true pre-trend is zero; the bias is entirely mechanical. The covariance is also upward biased, but by a smaller factor (about 80\%) than the variance. This differential inflation breaks the proportionality that held under i.i.d. shocks: the bias now depends on the specific autocorrelation structure, not just the spell design. As a result, the naive post-arrival slope $\hat\beta_2$ at $+2$ is below one (about 0.90), consistent with a gradual, biased buildup. Debiasing removes the pre-trend, recovers the true covariance and variance, and delivers a post-arrival slope indistinguishable from the true value of 1.00.

The unbalanced-panel scenario (with a 20\% annual hazard of CEO change) compounds persistence and varying spell lengths. We again see a severe upward bias in the variance and evidence of pre-trends in the naive estimates. Debiasing removes these artifacts. Post-arrival dynamics in the naive estimates look less gradual than under persistence alone, plausibly because typical spells are shorter in unbalanced panels, leaving less time for any gradual buildup to manifest.

In the excess-variance scenario, the treated group has higher shock variance than controls. Relative to the baseline, this produces more dispersion and an even stronger upward bias in both variance and covariance; the debiased estimates correct both. Finally, the ``all complications'' scenario combines short and unbalanced spells, persistent errors, and excess variance. Here all problems are most severe: very large variance bias, pronounced pre-trends before the CEO arrives, and distorted dynamics. Yet the placebo correction handles them jointly: the debiased variance matches 1.00, the post-arrival slope is 1.00, and the pre-trend is 0.00 across all six scenarios. 

Figure~\ref{fig:mc} displays variance and covariance dynamics across three Monte Carlo scenarios. Panels A--B examine the baseline with serially uncorrelated errors, panels C--D introduce persistent shocks, and panels E--F combine all complications including excess variance in the treated group. Our debiased estimator is the vertical distance between treated (red) and counterfactual (blue) lines.

Panels A and B show the baseline scenario with i.i.d. errors. Pre-arrival variance equals one in both treated and control groups. Post-arrival variance in the treated group jumps to two, reflecting the added contribution of CEO heterogeneity (true variance equals one). Covariance displays a parallel level shift. The red line lies exactly one unit above the blue line in both panels, confirming that the debiased estimate recovers the true effect. No spurious pre-trends emerge because shocks are serially uncorrelated. 

Panels C and D introduce persistent shocks ($\rho=0.9$). The control group now exhibits spurious pre-trends despite zero true CEO effects. Variance declines sharply before the pseudo-transition: because revenue changes are measured relative to the transition date, periods farther in the past mechanically accumulate more shocks, creating a negative pre-trend. Covariance displays a milder upward pre-trend. Persistent shocks induce correlation between past outcomes and future estimated CEO effects even when no true CEO change occurs, because high future outcomes signal a growth trajectory that began before the pseudo-transition. These spurious pre-trends are entirely mechanical artifacts of persistence combined with the event-study design. Post-transition, both moments continue rising in the control group, but the vertical gap between the red and blue line remains constant at one, demonstrating that our debiased estimator correctly removes the spurious dynamics and recovers the true CEO effect.

Panels E and F examine the scenario with all complications: persistent shocks, unbalanced panels, and excess variance in the treated group. We present three lines on the figures: treated (red), control (black), and excess-variance-corrected counterfactual (blue). The treated group exhibits higher volatility than controls by construction. The excess variance correction scales control group moments by the variance ratio estimated before CEO arrival, constructing the counterfactual second moments the treated group would exhibit absent the CEO change. The blue line removes both the spurious pre-trend and the scale difference. The vertical distance between red and blue again recovers the true effect of one in both variance and covariance, confirming that the method handles proportional volatility differences without parametric assumptions about the shock process.

\section{Application: CEO Arrivals and Revenue in Hungarian Firms}

We apply the placebo-controlled debiasing to the universe of Hungarian firms, focusing on CEO arrivals and firm revenue outcomes. The Hungarian data offer an ideal setting, as they encompass the entire population of businesses over three decades, unlike many other studies that focus exclusively on large corporations \citep{Bertrand2003-io, crossland2011differences, quigley2015has}. This scope yields a large number of CEO  transitions across firms, allowing us to study the arrival of new CEOs and the associated within-firm dynamics. We document the data construction, sample restrictions, and additional statistics in our companion paper \citep{ceo_value}. Here we summarize the minimal features needed for this methodological application and outline the design we will estimate.

We consider all CEO changes between 1992 and 2022 (31 years). We restrict the sample to firms that have ever reached at least 5 employees in their lifetime, to exclude non-employer businesses that rarely change CEOs and would inflate the sample numerically without adding identifying transitions. After this restriction, there are 59,039 CEO changes. Many transitions are from the first to the second CEO and from founders to non-founders; these categories are similar but not identical because founders can return later and some firms begin with outsider CEOs.

Identification relies on random mobility (strict exogeneity): past and future shocks to outcomes are orthogonal to CEO mobility timing. Because applied work commonly examines pre-trends to assess plausibility, we present event-study profiles grounded in this assumption.

We use firm revenue as the outcome variable for clarity of exposition. We remove industry–year means so that the outcome is measured as a deviation from the sectoral environment.

\subsection*{Estimating CEO Spell Effects}
For each transition from CEO A to CEO B, let the firm have two consecutive spells covering the entire tenures of A and B (not just the event window). We estimate a spell-level CEO effect as the within-spell mean of demeaned revenue. Concretely, let \(r_{it}\) denote firm revenue and let \(\tilde r_{it}=r_{it}-\bar r_{st}\) be revenue demeaned by industry–year \((s,t)\) cells. Let \(1\{\text{spell}=s\}\) indicate the A or B spell. The spell effect is
\begin{equation}
\hat z_{is} = \frac{1}{T_{is}}\sum_{t\in s} \tilde r_{it},\qquad s\in\{A,B\}.
\end{equation}
Define the change in the CEO effect for transition \(i\) by
\begin{equation}
\Delta \hat z_i = \hat z_{iB} - \hat z_{iA}.
\end{equation}

We estimate CEO spell effects but do not recover the firm fixed effects. Because all contrasts are computed relative to the year before CEO arrival ($t=-1$), any firm-level intercept would be differenced away.\footnote{If a firm has more than two consecutive spells, we treat adjacent spell pairs as separate transitions; multiple transitions per firm are allowed but clustered at the firm level.}

\subsection*{Event-Study Specification}
We then study revenue dynamics around CEO arrivals in event time. For each transition with event date \(g\), define dummies for event-time leads/lags relative to \(g\) over the window \([-4,+3]\). We estimate
\begin{equation}
\tilde r_{it} = \alpha_i + \sum_{\ell\in\{-4,-3,-2,0,1,2,3\}} \beta_{\ell}\, 1\{t-g=\ell\} + \varepsilon_{it},
\end{equation}
with the baseline period normalized to zero at \(t=-1\). The coefficients \(\beta_{\ell}\) trace out the revenue response before and after CEO arrival. Under random mobility, we expect no pre-trends: \(\beta_{-4},\beta_{-3},\beta_{-2}\approx 0\). Post-arrival dynamics are a question for the data: effects may be immediate or build gradually during the new CEO’s tenure. We will report both naive and placebo-corrected profiles; the latter subtract the corresponding placebo moments computed on matched non-changers with replicated spell designs.

\subsection*{Placebo Construction and Debiasing}
Placebo transitions assign a fake CEO change at the same calendar year as an actual transition to control firms that remain under the same CEO and have long no-change runs covering the event window. We split these control runs into the same pre/post spell lengths as the treated transition and align them in event time. Because the design matrices and weights are the same by construction, and because we assume the error autocovariance (\(\Sigma\)) is the same up to a scalar between treated and placebo groups, the covariance and variance bias components can be ``precomputed'' in the placebo sample and, after correcting for the excess variance factor, subtracted from the treated estimates. This delivers debiased variance of \(\Delta \hat z\) and debiased event-study coefficients \(\beta_{\ell}\).


%%%%%%%%%%%%%%%%%
\subsection*{Results}
%%%%%%%%%%%%%%%%% 

We present the naive (red line) and the debiased (blue line) event studies examining revenue dynamics around CEO transitions (Figure \ref{fig:application}). 

\paragraph{Variance of Revenue Changes (Panel A).} Panel A displays the variance of the change in log revenue by event time. The naive estimate is severely upward biased, particularly in the post-transition period. There is also a pre-trend, which is naturally arising in the data: at times farther away from the baseline year $t=-1$, the variance of revenue changes mechanically increases due to accumulating shocks. The debiased estimate  removes this mechanical noise. The debiased variance remains constant before the CEO arrival, and jumps to about 1 after arrival. There is a partial adjustment in the first year, but this is expected since some CEOs arrive mid-year. 

\paragraph{Covariance of Revenue with CEO Effects (Panel B).} Panel B shows the covariance of the revenue change with the change in CEO fixed effects, a key component in event-study regressions. The naive estimate displays strong spurious pre-trends: before the CEO arrives, past revenue changes appear correlated with future CEO quality. The debiased estimate demonstrates that the pre-trend is entirely spurious: the blue line is completely flat and close to zero. After the CEO arrives, the covariance becomes positive. As we saw with the variance event time, the first year shows partial adjustment, but covariance stabilizes thereafter.

\paragraph{Variance Decomposition (Panel C).} Panel C translates the variance and covariance estimates into an $R^2$: the fraction of revenue variance explained by CEO fixed effects. This statistic should be zero before the CEO arrives, since future CEO identity cannot explain past outcomes. The debiased estimate confirms this. After the CEO arrives, the naive estimate suggests CEO effects explain 40--60\% of revenue variance. The debiased estimate shows an $R^2$ of 20--30\% -- still substantial, but only half the naive estimate. This decomposition demonstrates that ignoring small-sample bias dramatically overstates the role of CEOs in firm performance.

\paragraph{Variance Decomposition by Firm Age (Panel D).} Using the estimated variance of CEO fixed effects, we can compute how much CEOs contribute to revenue variance by firm age. Assuming that each CEO change adds the same variance to log revenue, we can back out the variance of CEO effects by age group. As firms age, they accumulate revenue shocks (red line), but they also tend to change CEOs more often (blue). Over time, about a tenth of revenue variance is explained by CEO effects. This is smaller than the $R^2$ reported in Panel C because older firms have more accumulated shocks, diluting the relative contribution of CEOs.

\paragraph{Regression Coefficients (Panel E).} Panel E rescales the covariance and the variance of the CEO fixed effect into regression slopes, measuring the elasticity of revenue with respect to CEO quality. This is a nonlinear transformation, so we compute standard errors via the delta method. The naive estimate shows pronounced pre-trends and a gradual post-arrival buildup. The debiased estimate eliminates the pre-trend and reveals almost immediate, persistent effects. 

\paragraph{Regression of Exporter Status on CEO Quality (Panel F).} Panel F extends the analysis to a binary outcome: whether the firm exports. We estimate a linear probability model of exporter status on CEO fixed effects in event time. The naive estimate shows slow adjustment after CEO arrival, while the debiased estimate reveals an immediate jump in export probability. CEOs appear to have a substantial impact on firms' international engagement, with a 10\% better CEO increasing export likelihood by 0.4--0.6 percentage points.


%%%%%%%%%%%%%%%%%%%%
\section{Conclusion}
%%%%%%%%%%%%%%%%%%%%

This paper estimates the contribution of CEOs to firm revenue by exploiting a unique administrative dataset covering the entire population of Hungarian private firms and their CEOs from 1992 to 2022. The novelty of the data lies in its unprecedented scope and completeness, allowing the study of CEO effects not only in large firms but crucially in small and medium-sized enterprises that dominate every economy. The combination of the dataset with a theoretically grounded placebo-controlled method allows for a more precise attribution of CEO effects. 

The paper develops a new placebo-controlled event study design that overcomes limited mobility bias, which contaminates studies using two fixed effects.  By creating matched placebo CEO transitions in firms without actual leadership changes, the method effectively separates true CEO skill effects on revenue from mechanical noise. Empirically, the findings reveal that the true causal effect of CEO quality on firm revenue is economically meaningful but notably smaller than raw correlations suggest, explaining about 29 percent of revenue variation after correction. 

An alternative way to deal with the noise problem is to use observable manager characteristics as measures of skill. Observable characteristics such as education and work experience \citep{DePirro2025}, foreign name as a proxy for international exposure \citep{Koren2023expat}, and the selectiveness of entry cohorts \citep{koren2024managers} offer more reliable, though narrower, measures of specific dimensions of managerial quality. These observables capture only partial aspects of CEO ability but avoid the mechanical noise that contaminates fixed effects estimates.

\clearpage
\appendix
\renewcommand{\thefigure}{A\arabic{figure}}
\renewcommand{\thetable}{A\arabic{table}}
\setcounter{figure}{0}
\setcounter{table}{0}


\section{Appendix: Derivation of Bias Terms}
This appendix provides the formal matrix algebra derivation of the bias terms in second moments of estimated CEO effects.

\subsection{Matrix Notation Setup}

\paragraph{Model setup.} Let firm $i$ be observed for $T$ periods with outcome path $\mathbf y_i\in\mathbb R^T$ that decomposes into a (piecewise-constant) manager-effect path $\mathbf z_i$ and shocks $\mathbf e_i$. Suppose manager 1 serves in the first $T_1$ periods and manager 2 in the last $T_2=T-T_1$ periods. (Other transition paths can be modeled similarly.) The model is
\begin{equation}
\mathbf y_i = \mathbf \alpha_i + \mathbf z_i + \mathbf e_i
\end{equation}
with 
\begin{equation}
\mathbb E[\mathbf z_i\mathbf z_i']= \mathbf \Lambda_i=
\begin{bmatrix}
  \lambda_{i11}\otimes \mathbf{11}' & \lambda_{i12}\otimes \mathbf{11}'\\
  \lambda_{i12}\otimes \mathbf{11}' & \lambda_{i22}\otimes \mathbf{11}'
\end{bmatrix},
\qquad \mathbb E[\mathbf z_i\mathbf e_i']=0,
\qquad \mathbb E[\mathbf e_i\mathbf e_i']=\sigma_1\mathbf\Sigma
\end{equation}
for changing firms, and 
\begin{equation}
\mathbb E[\mathbf z_i\mathbf z_i']= \mathbf \Lambda_i=
  \lambda_{i11}\otimes \mathbf{11}',
\qquad \mathbb E[\mathbf z_i\mathbf e_i']=0,
\qquad \mathbb E[\mathbf e_i\mathbf e_i']=\sigma_0\mathbf\Sigma
\end{equation}
for non-changing firms. The first equation is notation for the variance and covariance of manager fixed effects at firm $i$. The second equation states strict exogeneity (random mobility): every period's shock is mean independent of the entire CEO path. The third assumption allows for arbitrary autocorrelation in shocks over time but assumes that this autocorrelation is the same for changing and non-changing firms, up to a scalar multiplier $\sigma_1/\sigma_0$ (proportional autocovariance).

\paragraph{Projection Matrix and LSDV Estimator.} Introduce a design matrix $\mathbf D$ to map rows to CEO spells. For the example with $T_1=2$ pre-spell periods and $T_2=3$ post-spell periods:
\[
  \mathbf D = \begin{bmatrix}
    1 & 0\\
    1 & 0\\
    0 & 1\\
    0 & 1\\
    0 & 1
  \end{bmatrix},
\]
and the diagonal matrix of spell lengths $\mathbf T$ as
$$
  \mathbf T = \operatorname{diag}(T_1,T_2)=\operatorname{diag}(2,3).
$$
The projection matrix that converts outcomes to within-spell means is
$$
  \mathbf P = \mathbf D\,\mathbf T^{-1}\,\mathbf D'.
$$
In the example,
\[
  \mathbf P = \begin{bmatrix}
    1/2 & 1/2 & 0 & 0 & 0\\
    1/2 & 1/2 & 0 & 0 & 0\\
    0 & 0 & 1/3 & 1/3 & 1/3\\
    0 & 0 & 1/3 & 1/3 & 1/3\\
    0 & 0 & 1/3 & 1/3 & 1/3
  \end{bmatrix}\quad\Rightarrow\quad
  \mathbf {Py}_i = 
  \mathbf\alpha_i +
  \begin{pmatrix}
  \frac12 \sum_{t=-2}^{-1} y_{it} \\
  \frac12 \sum_{t=-2}^{-1} y_{it} \\
  \frac13 \sum_{t=0}^{+2} y_{it}\\
  \frac13 \sum_{t=0}^{+2} y_{it}\\
  \frac13 \sum_{t=0}^{+2} y_{it}
  \end{pmatrix}.
\]
Note that $\mathbf P$ is idempotent ($\mathbf P^2=\mathbf P$) and symmetric ($\mathbf P'=\mathbf P$). We also have $\mathbf P\mathbf z_i = \mathbf z_i$ because $\mathbf z_i$ is piecewise constant within spells.

The least-squares dummy variable (LSDV) estimator of the CEO effects is
\begin{equation}\label{eq:lsdv_appendix}
  \hat{\mathbf z}_i = \mathbf P\mathbf y_i - \hat{\mathbf \alpha}_i = \mathbf z_i + \mathbf P\mathbf e_i.
\end{equation}  
The estimated CEO effects are the true effects plus the within-spell mean of shocks. (For our method, it does not matter how firm fixed effects $\hat{\mathbf \alpha}_i$ are estimated, as we difference out firm means in all contrasts.)

\subsection{Population Moments and Bias Terms}

The estimator is unbiased but is only consistent as spell lengths grow large. The small-sample noise contaminates all second moments of $\hat{\mathbf z}_i$. The relevant moments are
\begin{align}
  \mathbb E[\hat{\mathbf z_i}] &= \mathbf z_i,\\
  \mathbb E[\hat{\mathbf z_i}\hat{\mathbf z_i}'] &= 
\mathbb E[(\mathbf z_i + \mathbf P\mathbf e_i)(\mathbf z_i + \mathbf P\mathbf e_i)' ] =
    \mathbf \Lambda_i + \sigma\mathbf P\mathbf\Sigma\mathbf P,\\ 
  \mathbb E[\hat{\mathbf z_i}\mathbf e_i'] &= 
\mathbb E[(\mathbf z_i + \mathbf P\mathbf e_i)\mathbf e_i' ] = \sigma\mathbf P\mathbf\Sigma,\\
  \mathbb E[\hat{\mathbf z_i}\mathbf y_i'] &= 
\mathbb E[(\mathbf z_i + \mathbf P\mathbf e_i)(\mathbf z_i + \mathbf e_i)' ] = \mathbf \Lambda_i + \sigma\mathbf P\mathbf\Sigma.
\end{align}
The first equation restates unbiasedness. The second equation shows that the variance-covariance of estimated CEO effects is inflated by the term $\sigma\mathbf P\mathbf\Sigma\mathbf P$ relative to the true variance-covariance $\mathbf \Lambda_i$. The third equation shows that the covariance of estimated CEO effects with shocks is contaminated. The fourth equation shows that the covariance of outcomes with estimated CEO effects is also inflated.

\subsection{Bias in Regression Slopes}

For any linear contrast with weights $\mathbf w\in\mathbb R^T$ and $\mathbf w'\mathbf 1=0$, define the estimated effect for the contrast by
$$
\Delta\hat z_i = \mathbf w'\hat{\mathbf z}_i = \mathbf w'\mathbf P\,\mathbf y_i = \Delta z_i + \eta_i,\qquad \eta_i:=\mathbf w'\mathbf P\,\mathbf e_i,
$$
and recall that $\Delta y_i = \Delta z_i + \varepsilon_i$ where $\varepsilon_i=\mathbf w'\mathbf e_i$. 

The naive regression slope is
\begin{equation}
\hat\beta = 
  \frac {\Cov(\Delta \hat z_i, \Delta y_i)}
        {\Var(\Delta \hat z_i)} =
  \frac {\mathbf w' \mathbb E[\hat{\mathbf z_i}\mathbf y_i ']\mathbf w}
        {\mathbf w' \mathbb E[\hat{\mathbf z_i}\hat{\mathbf z_i}']\mathbf w} =
  \frac {\mathbf w' [\mathbf \Lambda_i + \sigma \mathbf P\mathbf\Sigma]\mathbf w}
        {\mathbf w' [\mathbf \Lambda_i + \sigma \mathbf P\mathbf\Sigma\mathbf P]\mathbf w} =
  \frac {\lambda_i + \sigma \mathbf w' \mathbf P\mathbf\Sigma \mathbf w}
        {\lambda_i + \sigma \mathbf w' \mathbf P\mathbf\Sigma \mathbf P \mathbf w}. 
\end{equation}  

\textit{Covariance bias term.} $\Cov(\varepsilon_i,\eta_i)$ is typically positive because both $\varepsilon_i$ and $\eta_i$ are formed from the same underlying shocks averaged over overlapping windows (short spells and persistent shocks make this more pronounced). This inflates the numerator.

\textit{Variance bias term (classical measurement error).} $\Var(\eta_i)>0$ inflates the denominator, the standard attenuation channel.

Together, the slope can be biased up or down depending on the balance between these terms. With i.i.d. shocks and long, balanced spells, $\Cov(\varepsilon_i,\eta_i)\approx 0$ and $\Var(\eta_i)$ is small. Both components are small-sample phenomena driven by short $T_1$ and $T_2$. As $T_1,T_2\to\infty$, spell means average out shocks so both bias terms vanish. 

\subsection{Sample Moments Across Heterogeneous Groups}

In practice, we estimate the population moments using sample analogs. There are $G$ groups with different spell lengths and contrasts (e.g., different event windows). Within each group $g$, we observe $N_g$ firms with outcomes $\{\Delta y_i, \Delta \hat z_i\}_{i=1}^{N_g}$. The sample moments are
\begin{align}
\widehat{\Cov}(\Delta y_i,\Delta \hat z_i) &= \sum_{g=1}^G \frac{N_g}{N} \widehat{\Cov}_g(\Delta y_i,\Delta \hat z_i)\\
\widehat{\Var}(\Delta \hat z_i) &= \sum_{g=1}^G \frac{N_g}{N} \widehat{\Var}_g(\Delta \hat z_i)
\end{align}

The expected values are
\begin{align}
\mathbb E\widehat{\Cov}(\Delta y_i,\Delta \hat z_i) &= \sum_{g=1}^G \frac{N_g}{N} \left[ 
\Bar\lambda_g
+ \sigma_g \mathbf w_g' \mathbf P_g\mathbf\Sigma \mathbf w_g\right] = 
\Bar\lambda +\sum_{g=1}^G \frac{N_g}{N} \sigma_g \mathbf w_g' \mathbf P_g\mathbf\Sigma \mathbf w_g\\
\mathbb E\widehat{\Var}(\Delta \hat z_i) &= \sum_{g=1}^G \frac{N_g}{N} \left[ 
\Bar\lambda_g
+ \sigma_g \mathbf w_g' \mathbf P_g\mathbf\Sigma \mathbf P_g \mathbf w_g\right] = 
\Bar\lambda +\sum_{g=1}^G \frac{N_g}{N} \sigma_g \mathbf w_g' \mathbf P_g\mathbf\Sigma \mathbf P_g \mathbf w_g
\end{align}

Here $\Bar\lambda_g$ is the average variance of CEO effects in group $g$, and $\Bar\lambda$ is the overall average variance of CEO effects across all groups. Without the bias, the sample covariance and variance would identify the average treatment effect $\Bar\lambda$ even with heterogeneous groups.

The bias terms are complicated group averages that depend on the design matrices $\mathbf P_g$, the weight vectors $\mathbf w_g$, and the shock structure $\sigma_g\mathbf\Sigma$. It is often impractical to estimate these components directly, especially when $\mathbf\Sigma$ is complex (e.g., persistent shocks) and when there are many groups with different designs.

\clearpage
\bibliographystyle{apalike}
\bibliography{../../lib/references.bib}
\clearpage

\section*{Tables and Figures}


\begin{figure}[htbp]
\centering
\includegraphics[width=0.9\textwidth]{figure/figuremc.pdf}
\caption{Monte Carlo event studies under six scenarios.} \label{fig:mc}
\vspace{.2cm}

\begin{minipage}{0.9\textwidth}
\footnotesize Notes: ????
\end{minipage}
\end{figure}

\begin{figure}[htbp]
\centering
\includegraphics[width=0.9\textwidth]{figure/application.pdf}
\caption{Revenue Dynamics Around CEO Transitions in Hungarian Firms}
\label{fig:application}
\vspace{.2cm}

\begin{minipage}{0.9\textwidth}
\footnotesize
Notes: Event studies of various second moments around CEO transitions. All panels normalize to year $t=-1$ (one year before CEO arrival). Panel A shows the variance of revenue changes from baseline. Panel B plots the covariance of revenue change with the change in CEO fixed effect. Panel C displays the fraction of revenue variance explained by CEO fixed effects (R-squared). Panel D shows the variance of CEO fixed effects by firm age, decomposed into revenue shocks and CEO contributions. Panel E shows regression coefficients from projecting revenue on CEO effects, and Panel F shows the coefficients from regressing exporter status on CEO effect. Red lines show naive OLS estimates; blue lines show placebo-corrected debiased estimates. Shaded regions indicate 95\% confidence intervals. The placebo group consists of matched non-changing firms from the same birth cohort and sector, assigned pseudo CEO transitions at the same calendar year as treated firms. Event window spans $[-4, +3]$ years around CEO arrival.
\end{minipage}
\end{figure}

\begin{table}[t]
\centering
\caption{Monte Carlo parameters by scenario}
\label{tab:mc_params}
\begin{threeparttable}
\begin{tabular}{l*{6}{>{\centering\arraybackslash}p{1.8cm}}}
\toprule
\textbf{Scenario} & \textbf{Baseline} & \textbf{Long Panel} & \textbf{Persistent Errors} & \textbf{Unbalanced Panel} & \textbf{Excess Variance} & \textbf{All Complications} \\
\midrule
\textbf{Parameters} & & & & & & \\
\addlinespace
$N_{\text{treated}}$ & \multicolumn{6}{c}{50,000} \\
$N_{\text{control}}$ & \multicolumn{6}{c}{50,000} \\
$\sigma(\Delta z)$ & \multicolumn{6}{c}{1.00} \\
$\sigma(\epsilon_{\text{control}})$ & \multicolumn{6}{c}{0.71} \\
\addlinespace
 $T_{\max}$ & 5 & 20 & 5 & 5 & 5 & 5 \\
 $\rho$ & 0.00 & 0.00 & 0.90 & 0.90 & 0.00 & 0.90 \\
 $\sigma(\epsilon_{\text{treated}})$ & 0.71 & 0.71 & 0.71 & 0.71 & 1.00 & 1.00 \\
 CEO change & --- & --- & --- & 0.20 & --- & 0.20 \\
 hazard & & & & & & \\
\midrule
\textbf{Estimates} & & & & & & \\
\\ $\sigma(\Delta \hat z)$ (OLS) & $0.104^{}$ & $0.101^{}$ & $0.123^{}$ & $0.118^{}$ & $0.109^{}$ & $0.137^{}$\\ $\sigma(\Delta \hat z)$ (debiased) & $0.100^{}$ & $0.100^{}$ & $0.100^{}$ & $0.100^{}$ & $0.105^{}$ & $0.122^{}$\\ \addlinespace $ R^2$ (OLS) & $0.738^{}$ & $0.686^{}$ & $0.787^{}$ & $0.857^{}$ & $0.584^{}$ & $0.786^{}$\\ $ R^2$ (debiased) & $0.662^{}$ & $0.668^{}$ & $0.593^{}$ & $0.598^{}$ & $0.417^{}$ & $0.267^{}$\\ \addlinespace$\hat \beta_2$ (OLS) & $1.007^{}$ & $1.004^{}$ & $0.933^{***}$ & $1.003^{}$ & $1.015^{}$ & $1.008^{}$\\ $\hat \beta_2$ (debiased) & $0.996^{}$ & $1.002^{}$ & $0.993^{}$ & $0.986^{}$ & $0.892^{***}$ & $0.660^{***}$\\ \addlinespace$\hat \beta_{-2}$ (OLS) & $-0.003^{}$ & $0.004^{}$ & $-0.089^{***}$ & $-0.085^{***}$ & $-0.004^{}$ & $-0.137^{***}$\\ $\hat \beta_{-2}$ (debiased) & $-0.004^{}$ & $0.003^{}$ & $-0.007^{}$ & $-0.001^{}$ & $-0.005^{}$ & $-0.000^{}$\\
\bottomrule
\end{tabular}
\begin{tablenotes}[flushleft]\footnotesize
\item Notes: Every Monte Carlo simulation assumes 50,000 treated firms and 50,000 control firms. The standard deviation of true CEO effect changes is 1.00 and the standard deviation of errors in the control group is 0.71 in all scenarios. The table lists scenario-specific parameters: maximum spell length $T_{\max}$, error autocorrelation $\rho$, error standard deviation in the treated group $\sigma(\epsilon_{\text{treated}})$, and the annual hazard of CEO change in the unbalanced panel scenarios. The \emph{baseline} calibration assumes short balanced spells and i.i.d. errors. Columns (2) to (5) introduce one complication at a time. The \emph{long panel} scenario extends the length of spells. The \emph{persistent errors} scenario adds strong autocorrelation. The \emph{unbalanced panel} scenario introduces firm-specific spell lengths drawn from a geometric distribution with a constant hazard of CEO change. The \emph{excess variance} scenario adds a 50\% increase in error volatility in the treated group relative to the control group. The \emph{all complications} scenario combines unbalanced spells, persistent errors, and excess variance. Parameter values are chosen to represent realistic moments from our application where CEO tenures are often short, revenue shocks are persistent, and treated firms are more volatile than control firms.
\end{tablenotes}
\end{threeparttable}
\end{table}

\end{document}
